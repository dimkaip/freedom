%ΠΡΟΤΑΣΗ ΠΡΟΣ ΤΟΝ ΕΛΛΗΝΙΚΟ ΛΑΟ ΠΡΟΚΕΙΜΕΝΟΥ ΝΑ ΦΕΡΟΥΜΕ Ως ΠΟΛΙΤΕΥΝΑ ΤΗΝ ΑΜΕΣΗ ΔΗΜΟΚΡΑΤΙΑ.
%ΓΡΑΜΜΕΝΟ ΣΕ xelatex
% ΒΑΣΙΣΜΕΝΟ ΣΤΗΝ ΥΠΕΡΤΑΤΗ ΑΡΧΗ.

\documentclass[a4paper]{article}
\usepackage[top=10mm, bottom=10mm, left=12mm, right=12mm, textwidth=210mm,textheight=297mm]{geometry}
\usepackage{xunicode}
\usepackage{xltxtra}
\usepackage[ancientgreek]{xgreek}
\usepackage{parskip}
\usepackage{pdfpages}
\usepackage{verbatim}
\usepackage{multicol}
\setmainfont[Mapping=tex-text,Ligatures=Common,Scale=1]{Tahoma}

\makeatletter
\newcommand{\verbatimfont}[1]{\def\verbatim@font{#1}}%
\makeatother

\begin{document}

\textbf{\large \centering Εἰς τὸ ὄνομα τῆς Ἁγίας καὶ Ὀμοουσίου καὶ Ἀδιαιρέτου Τριάδος\\}\par
\textbf{\large \centering ὡς τῆς Ὑπερτάτου Ἀρχῆς τοῦ Συντάγματος\\}\par

\begin{flushright}
{\scriptsize   Τάδε λέγει Κύριος: "Ἑγὼ εἰμι τὸ Ἄλφα καὶ τὸ Ὠμέγα, ὁ Πρῶτος καὶ ὁ Ἔσχατος, ἡ Ἀρχὴ καὶ τὸ Τέλος. Ἐμοῦ ἀκούετε."\\}
\end{flushright}

\textbf{\large \centering ΕΝΩΠΙΟΝ ΠΑΣΗΣ ΑΡΧΗΣ ΔΙΚΑΙΟΥ \\ ΕΙΣ ΤΟ ΟΝΟΜΑ ΤΟΥ ΕΛΛΗΝΙΚΟΥ ΛΑΟΥ\\}\par


\textbf{\Large \centering Συμβολαιογραφική Πράξη Παραχωρήσεως Δικαιώματος Ποινικής και Αστικής Ευθύνης στον Ελληνικό Λαό.\\}\par
%\vspace{1cm}

\setlength{\parindent}{5mm}
%\setlength{\parskip}{3mm plus4mm minus3mm}

Σήμερα στις ............................ εμφανίστηκαν οι γνωστοί μου όπως φαίνονται κατωτέρω με στοιχεία ταυτότητός των όπως φαίνεται στο τέλος της Συμβολαιογραφικής πράξεως στο Συμβολαιογραφείο εμού ........................................................................................................ επί της οδού .......................... στο νομό .............. και μου καταθέσανε το παρακάτω κείμενο το οποίο σκοπό έχει \textbf{να δημοσιευθεί προς τον Ελληνικό Λαό ώστε να δώσει στον κάθε πολίτη το δικαίωμα ποινικής και αστικής ευθύνης ενάντια στους εμφανισθέντες.}\par
Το κείμενο έχει ούτως:\par

Το \textbf{Καλιέλαιον Κίνημα} δημιουργήθηκε από την ανάγκη απελευθερώσεώς μας από τους Ρότσιλντ, τα όργανά τους που είναι οι τραπεζίτες και τα υποχείρειά τους που είναι οι πολιτικοί και οι σφουγκοκωλάριοί τους γύρω από αυτούς. \\ Επειδή είναι πλήρως σαφές ότι οι μυστικές υπηρεσίες -CIA DIA Mossad MI5 κλπ - βρίσκονται πίσω από κάθε προπαγάνδα και κακό που γίνεται στον πλανήτη --\textbf{και που εκπορεύεται από το City του Λονδίνου όπου κατοικοεδρεύουν οι εχθροί μας και εχθροί κάθε έθνους}-- προκειμένου να διαλυθεί η έννομος τάξις και να κατακτηθούν από τους τραπεζίτες οι λαοί και να εκποιηθούν προς χάριν των δικών τους πολυεθνικών όλος ο πλούτος, δια τούτο είναι ανάγκη να δημιουργηθούν θεσμοί εξ αρχής απαράβατοι και μη διαχειρίσιμοι και δομές τέτοιες στις οποίες δεν θα μπορούν οι ένοχοι πολιτικοί, εισαγγελείς και δικαστικοί και πράκτορες να πράξουν τη θέληση των πολυεθνικών και των τραπεζιτών και των Ρόθσιλντ και των συνοδοιπόρων τους.

Για όλα αυτά δίνω δημοσίως και ανεγκλήτως το δικαίωμα σε κάθε πολίτη να προβεί σε μήνυση και αγωγή \textbf{εναντίον μου} για \textbf{όλη την κινητή και ακίνητο περιουσία μου}, όπως επίσης και \textbf{να διαπράξει την εκτέλεση μου καθ οιοδήποτε τρόπο θέλει και δύναται} όταν βρισκόμενος σε θέση εξουσίας δεν πράξω ή ζητήσω δημόσια και συνεχώς και ψηφίσω δημόσια υπέρ των κάτωθι \textbf{σε οιοδήποτε κόμμα ανήκω κάθε φορά μη παραδίδοντας καμμία έδρα} και με κάθε τρόπο τα ακόλουθα, γιατί είναι φανερό ότι η εποχή της αθωότητος πέρασε.\par

Βασικές αρχές της Ελληνικής Δημοκρατίας είναι οι εξής:
\begin{itemize}
\bf
\item Ψήφος μόνο από Έλληνες και επιτίμους πολίτες τώρα και για πάντα.
\item Αιρετότης και Ανακλητότης σε κάθε επίπεδο Δημοσίου ζωής.
\item Κατάλυσις της Δημοκρατίας τιμωρείται με εκτέλεση δι απαγχονισμού δημόσια.
\item Βασική και απαράβατη αρχή του Καλιελαίου κινήματος είναι οτιδήποτε αποτελεί και είναι δημόσιο αγαθό απαγορεύεται να βρίσκεται σε χέρια ιδιωτών ή να ελέγχεται από πολιτικούς. Νερά, υπέδαφος, θάλασσες, μέταλλα, χρήμα, συγκοινωνίες, ενέργεια, υπηρεσίες, ασφάλειες, κλπ. Όλων αυτών η χρήσις καθορίζεται από τους πολίτες μέσω των Δήμων και των Μηνιαίων Συνελεύσεων.
\item Δεν νοείται η έννοια της μονιμότητος σε κανένα τομέα.
\item Οι Υπάλληλοι της Δημοκρατίας είναι υποχρεωμένοι στην άριστη γνώση του Συντάγματος και της Ιστορίας του Δήμου τους και της Επικρατρείας.
\item Οι Υπάλληλοι της Δημοκρατίας δίνουν τον λόγο της Τιμής και της Υπολήψεώς των ότι θα προασπίζονται το Σύνταγμα και την Δημοκρατία με την ζωή τους.
\item Οι πολιτικοί ΔΕΝ ΑΠΟΖΗΜΙΩΝΟΝΤΑΙ, ΚΑΤΑΓΡΑΦΕΤΑΙ Η ΚΙΝΗΤΗ ΚΑΙ ΑΚΙΝΗΤΟΣ ΠΕΡΙΟΥΣΙΑ ΤΩΝ ΚΑΙ ΚΑΤΑ ΤΗ ΘΗΤΕΙΑ ΤΩΝ ΔΕΝ ΠΡΟΣΘΕΤΟΥΝ ΤΙΠΟΤΕ ΣΤΗΝ ΠΕΡΙΟΥΣΙΑ ΤΟΥΣ.
\item Πολιτικοί που αθετούν το ΣΥΝΤΑΓΜΑ εκτελούνται δημόσια.
\item Όσοι καθ οιονδήποτε τρόπο λειτουργήσουν ενάντια στο Καλιέλαιο Κίνημα είναι σαν να λειτουργούν ενάντια στον Ελληνισμό, ενάντια στο Έθνος, ενάντια σε εμάς τους γηγενείς και ιδιοκτήτες όλης της Ελληνικής γης εξ αδιαιρέτου, ενάντια στη Λαϊκή Κυριαρχία και την αυτοδιάθεση των Ελλήνων, ενάντια στην Ηθική και το Δίκαιο, ενάντια στην Άμεση Δημοκρατία, εκτελούνται χωρίς κανένα δικαστήριο, όποια θέση και αν κατέχουν ως προδότες των Ελλήνων και του Έθνους και υπέρ των οχτρών μας των φράγκων και των σιωνιστών που τους ελέγχουν πλήρως. Όσοι εξάγουν λόγο ενάντια στο κίνημα, το οποίο είναι κίνημα Δικαίου και αποκαταστάσεως της Δημοκρατίας  ενάντια στην πολιτική των εγκληματιών χιτλερικών και σιωνιστών, οι οποίοι θέλουν να καταστρέψουν κάθε έθνος, κάθε κρατικό μηχανισμό και να ελέγχουν τα πάντα με τις πολυεθνικές τους, με ιδιωτικά δικαστήρια, με τα μεταλλαγμένα τους, με τους πούστηδες και τις λεσβίες που έχουν προωθήσει παντού και με τους κρυφούς εβραίους σε όλα τα μέρη της εξουσίας και τα δουλάκια τους τους μασσώνους, ως τους κατ εξοχήν συμμορίτες ενάντια σε κάθε κράτος και κάθε κρατικό μηχανισμό, και μέσω αυτών των μηχανισμών καταστρέφουν τη γλώσσα, τα ήθη και έθιμα, την παράδοση, προωθώντας τις δικές τους πολιτικές και σκέψεις και ιδέες οι οποίες λειτουργούν υπέρ των συμφερόντων τους, εκτελούνται άνευ ουδεμίας απολογίας.
\end{itemize}

\textbf{Όλα αυτά επειδή .....}

\begin{itemize}

\item Επειδή μετά την υπογραφή των μνημονίων, την εξαπάτηση των πολιτικών, την αλλοίωση των εκλογών από την \textbf{εβραϊκή Singular Logic}, τα \textbf{μνημόνια}, την \textbf{εσχάτη προδοσία} που διαρκώς ποιούν οι \textbf{πολιτικάντηδες}, η δημόσια αποκάλυψη ότι η πολιτικοί είναι \textbf{όργανα της CIA} λειτουργώντας ως \textbf{έμμισθα όργανά της}, αφού η \textbf{CIA} με ειδική υπηρεσία \textbf{μισθώνει τους πολιτικούς}, και αφού έχει ήδη \textbf{καταλυθεί το Σύνταγμα της χώρας} και ενεργοποιήθηκε ήδη το \textbf{άρθρο 120 του Συντάγματος}, και αφού \textbf{η Δημοκρατία ετρώθη}, και αφού η \textbf{προδοσία των πολιτικών} συνεχίζεται ενώ δημόσια \textbf{οι δωσίλογοι και οι σφουγγοκωλάριοί τους συνεχίζουν να μας δουλεύουν}, και επειδή πρέπει να \textbf{πληρώσουν ποινικά και αστικά} όλοι τούτοι \textbf{με πρώτους τους εισαγγελείς και δικαστές του Αρείου Πάγου} που \textbf{συνεργούν διά της απραξίας στην κατάλυση του πολιτεύματος}.
\item Επειδή οι \textbf{πολιτικοί εισαγγελείς δικαστές ΜΜΕ δημοσιογράφοι} και διάφοροι παπαρολόγοι είναι συνεργοί σε όλα αυτά και οι \textbf{σφουγγοκωλάριοι} δικοί τους πρέπει να \textbf{χάσουν κινητή και ακίνητη περιουσία}.
\item Επειδή η \textbf{ΕΚΤ, το ΔΝΤ και οι τοκογλύφοι} \textbf{πρέπει να πληρώσουν} τη ζημιά που έχουμε υποστεί.
\item Επειδή κάθε ξένος εισβολέας αλλόφυλος \textbf{πρέπει να γυρίσει στην πατρίδα του} με \textbf{νόμο του κράτους χωρίς καμμία ιθαγένεια, δικαιώματα , αλλοίωση του πληθυσμού κλπ}.
\item Επειδή η \textbf{Ορθοδοξία είναι το χαρακτηριστικό του Έθνους} βάσει της οποίας απαγορεύονται οι λοιπές δοξασίες να έχουν δημόσια προβολή και να είναι \textbf{δημοσίου δικαίου}.
\item Επειδή πρέπει άμεσα να καταργηθεί το διπλό φύλλο κυβερνήσεως του Παπανδρέου και το \textbf{Κεντρικό Ισραηλιτικό Συμβούλιο έχει ισχύ κράτους} και δύναται να παίρνει φόρους διαφορετικά από όλους εμάς τους υπόλοιπους.\par
\item Επειδή όλοι τούτοι οι πολλιτικάντηδες έχουν δημιουργήσει ανήκεστη βλάβη σε εμάς τους \textbf{ντόπιους γηγενείς Έλληνες, πρέπει δε να πληρώσουν με κατάσχεση κινητής και ακίνητης περιουσίας και φυλάκιση επί εσχάτη προδοσία}.\par
\item Επειδή όλα αυτά που γίνονται, \textbf{γίνονται για την διάλυση του Ελληνισμού και την εξάλειψη της Ελληνικής φυλής και οι οικονομικές επιβαρύνσεις γίνονται για να εξυπηρετηθούν οι πολυεθνικές και οι τραπεζίτες σε βάρος ημών των γηγενών Ελλήνων.}\par
\item Επειδή \textbf{οι τραπεζίτες έχουν υπεξαιρέσει το ποσό των διακοσίων τριάντα τριών δις ευρώ} όπως βλέπουμε σε έγγραφα δημιουργώντας ανοικτές ανήκεστο βλάβη στη χώρα και σε κάθε Έλληνα πολίτη.
\item Επειδή τα ΜΜΕ λειτουργούν προδοτικά και προς το σκοπό της νέας τάξης πραγμάτων η οποία θέλει τα παρακάτω ...

\end{itemize}

\begin{itemize}

\item Να πάρει τη γη μας.
\item Να πάρει τα νερά μας.
\item Να χαλάσει με ψεκασμούς τη γη μας προκειμένου να μας επιβάλλει τα μεταλλαγμένα του και τις κρεατόμαζες.
\item Να μας καθίζει στο σκαμνί αν δεν του υπακούομε πλήρως με μυστικά και ιδιωτικά δικαστήρια.
\item Να πάρει οποιαδήποτε πλουτοπαραγωγική πηγή.
\item Να αλλοιώσει τον πληθυσμό της χώρας.
\item Να εκποιήσει με ψεύτικες χρεώσεις δημόσιο και ιδιωτικό πλούτο.
\item Να αλλοιώσει την θρησκεία μας ως Ορθοδόξων.
\item Να μας μετατρέψει σε ανταλλακτικά.
\item Να μας εξευτελίσει πλήρως και με την γενοκτονία να πάψουμε να υπάρχουμε.
\item Να πάρει όλους μας τους επιστήμονες ώστε να πάψει σε βάθος χρόνου να υπάρχει Πατρίδα.
\item Και πλείστα άλλα όσα οι ειδικοί έχουν αποκαλύψει.
\end{itemize}


Για όλα αυτά δίνω δημοσίως και ανεγκλήτως το δικαίωμα σε κάθε πολίτη να προβεί σε μήνυση και αγωγή \textbf{εναντίον μου} για \textbf{το σύνολο κινητής και ακινήτου περιουσίας μου}, όπως επίσης και \textbf{να διαπράξει την εκτέλεση μου καθ οιοδήποτε τρόπο θέλει και δύναται} εάν δεν ενεργήσω, ζητήσω δημόσια και συνεχώς και ψηφίσω δημόσια υπέρ των κάτωθι \textbf{σε οιοδήποτε κόμμα ανήκω κάθε φορά μη παραδίδοντας καμμία έδρα σε περίπτωση διαγραφής} και με κάθε τρόπο τα ακόλουθα και όλα αυτά επειδή εγώ προσωπικά δεν είμαι ομοφυλόφιλος, δεν είμαι ρουφιάνος, δεν είμαι μασσώνος, δεν είμαι εβραίος, δεν είμαι εβραίος με αλλαγμένο όνομα, δεν είμαι όργανο της νέας τάξης, των ρόθτσιλντ, των τραπεζιτών, των τοκογλύφων, των απατεώνων γενικά.\par

\textbf{Δηλώνω ότι δεν θα ακουμπήσω ούτε ένα ευρώ από τις αποζημιώσεις του βουλευτή αν υπάρχει, δίδοντάς τες σε έργα υπέρ του Έθνους.} Ζητήσω από τη βουλή άρση ασυλίας ώστε να δικαστούν οι δωσίλογοι βουλευτές. Ζητήσω πλήρη σεισάχθεια εκτός των απατεώνων που θα καθορίσουν ορκωτά δικαστήρια. Μηνύσω και προχωρήσω σε αγωγές ενάντια σε κάθε δωσίλογο, οι δαπάνες δε των μηνύσεων και αγωγών θα προέρχονται από το μισθό του βουλευτού και τα χρήματα των αγωγών από κατασχέσεις θα διατεθούν άμεσα για τη δημιουργία μη κερδοσκοπικού συνεταιρισμού με σκοπό την εκδίωξη των πολυεθνικών, τη δημιουργία θέσεων εργασίας, την ενέργηση της οικοτεχνίας και στον οποίον ο πρόεδρος και το διοικητικό συμβούλιο αλλάζει κάθε χρόνο. Επίσημη τοποθέτηση για άμεση κατάργηση του ΦΠΑ. Άμεσες ενέργειες για \textbf{κατάργηση των ΕΝΦΙΑ και λοιπών προδοτικών ανοησιών των αληταράδων του Υπουργείου Οικονομικών και ακύρωση οποιοδήποτε χρεών των πολιτών προς το κράτος}, αυτό δε \textbf{θα το πληρώσουν οι πολιτικάντηδες η ΕΚΤ και το διεθνές νομισματικό ταμείο} ως αποζημίωση για διαφυγόντα κέρδη, ηθική βλάβη και λοιπά... Άμεση κατάργηση του Υπουργείου Οικονομικών, ΤΑΧΙΣ, αποδείξεων, ταμειακών μηχανών, pos, πλαστικού χρήματος και λοιπές απατεωνίστικες φορολογήσεις, και \textbf{φορολόγηση στο ακαθάριστο} από τους Δήμους. \textbf{Δικαίωμα δωρεάν ραφιού} των παραγωγών \textbf{στις πολυεθνικές}. \textbf{Απαγόρευση μεταλλαγμένων}. \textbf{Απαγόρευση κάθε είδους ηλεκτρονικής καταγραφής που σκέφτηκαν οι σιωνιστές και τα όργανά τους οι μασσώνοι, δηλαδή κάρτες πολίτη, βιομετρικές, πλαστικά χρήματα και άλλες προδοτικές ανοησίες των γνωστών τοις πάσιν αληταράδων και των συν αυτών σφουγκοκωλάριων}. Απαγόρευση ισχύος κάθε συμβάσεως της οιαδήποτε ενώσεως ενισχύει τις πολυεθνικές, ιδιωτικούς στρατούς, μεταλλαγμένα, ειδικά φορολογικά καθεστώτα, ζώνες δουλοπαροίκων εργαζομένων, πόλεις αλλοεθνών και λοιπές \textbf{ανοησίες των διεθνών και εγχωρίων προδοτών}. Αμεση αποκατάσταση ομολογιούχων και της ζημίας τους \textbf{από τις περιουσίες των πολιτικών}. \textbf{Αιρετότητα και ανακλητότητα} των εισαγγελέων και δικαστών και \textbf{μόνο ορκωτά δικαστήρια μιας στάσεως} και τελειώματος των υποθέσεων \textbf{εντός τριών μηνών το αργότερο} και ακόμη με \textbf{κανέναν χρονικό περιορισμό παραγραφής οιασδήποτε ευθύνης}. \textbf{Διεθνής διεκδίκηση} των κλεμμένων από τους γνωστούς παροικούντες την Ιερουσαλήμ απατεώνες και τους λοιπούς σφουγγοκωλάριους με \textbf{Ελληνικό Δημόσιο Συνέδριο Νομικών}. Άμεση κατάργηση υπουργείου παιδείας και δημιουργία νέου υπουργείου παιδείας με \textbf{αιρετούς υπαλλήλους από κάθε Δήμο}. \textbf{Άμεση κατάργηση των δωσιλογικών ΜΜΕ και δημιουργία δημοσίων ΜΜΕ σε κάθε Δήμο και απαγόρευση αλλοφύλλων, μασσώνων κλπ να έχουν διαχείρηση καναλιών, αιρετότητα και ανακλητότητα στους δημοσιογράφους κάθε Δήμου, με ψήφισμά τους από τους Δήμους όπως και οι πολιτικοί, όπως επίσης και πληθώρα καναλιών των Ελλήνων πολιτών χωρίς φορολογικά βάρη για το απρόσκοπτο και το ακατεύθυντο των ειδήσεων.}\\
\textbf{Ακόμη δηλώνω την συμβολαιογραφική κατοχύρωση της Φωνής τη Αληθείας μέσω αυτής της πράξεως και του Καλλιελαίου κινήματος το οποίον είναι προσωπικό μου δημιούργημα και η βάσις κάθε πολιτικής μου σκέψεως και ενεργεία βασικές αρχές του οποίου είναι αιρετότης και ανακλητότης σε κάθε επίπεδο δημοσίου ζωής και η βάσις του οποίου είναι τα Θέσφατα, βάσις αναλοίωτος σε κάθε χρόνο.}

\hspace{2cm}{\large \textbf{Οἱ πολίτες ὅπως εἰς τὴν αρχή τῆς Συμβολαιογραφικῆς Πράξεως.}}\\

%Παρακαλείσθε Σε ότι δε κόμμα μπω ή δημιουργήσω και όσοι έχουν υπογράφει \textbf{ανάλογες συμβολαιογραφικές πράξεις και δεν ανήκουν σε υπάρξαντας πολιτικούς} αυτούς να ψηφίζετε ώστε να μπορέσουμε να ξεκαρφιτσώσουμε απ την πλάτη μας \textbf{το αλληταριό και τους σφουγκοκωλαρίους τους} και φέρουμε ειρήνη και ευημερία στον τόπο.\\
\textbf{Απαγορεύτεται υπάρξαντες πολιτικοί, εβραίοι και αλλοεθνείς να εισέλθουν στο κίνημα επί ποινής θανάτου.}\\
Τέλος καὶ Χριστῷ τῷ Θεῷ Δόξαν.

%\includepdf[pages=-]{kal}

%\includepdf[pages=-]{am}

\textbf{Δηλώνω ανεπιφύλακτα και με κάθε ευθύνη ποινικής και αστικής διώξεως για όλα μου τα περιουσιακά στοιχεία και εν τέλει της εκτελέσεώς μου από τον οποιαδήποτε εάν αυτά που παρακάτω γράφονται δεν γίνουν άμεσα ενεργήσιμα μετά από την κατάκτηση της εξουσίας.}

\textbf{Οι αρχές του κινήματός μας για την Ελευθερία, τη Δημοκρατία, το Δίκαιο, την Ισονομία και Ισοπολιτεία.}


\textbf{Χρηματοδότηση...}
\begin{itemize}
\item \textbf{ΔΗΛΩΝΩ ΚΑΙ ΑΠΟΔΕΧΟΜΑΙ ΟΤΙ Η ΦΩΝΗ ΤΗΣ ΑΛΗΘΕΙΑΣ ΑΝΗΚΕΙ ΑΠΟΚΛΕΙΣΤΙΚΑ ΣΕ ΟΛΟΥΣ ΑΠΟ ΚΟΙΝΟΥ ΤΟΥΣ ΥΠΟΓΡΑΦΟΝΤΑΣ ΤΩΡΑ ΑΛΛΑ ΚΑΙ ΣΤΟ ΜΕΛΛΟΝ, ΣΕ ΟΛΑ ΔΗΛΑΔΗ ΤΑ ΜΕΛΗ ΤΟΥ ΚΙΝΗΜΑΤΟΣ ΚΑΙ Η ΔΙΑΧΕΙΡΙΣΗ ΤΩΝ ΧΡΗΜΑΤΩΝ ΤΗΣ ΓΙΝΕΤΑΙ ΑΠΟ ΟΛΟΥΣ ΜΑΣ ΚΑΙ ΙΣΟΠΟΣΑ ΜΟΙΡΑΖΟΝΤΑΙ ΣΕ ΟΛΟΥΣ ΜΑΣ ΓΙΑ ΤΗΝ ΔΙΑΧΕΙΡΙΣΗ ΤΟΥΣ.}
\item Το κίνημα δεν χρηματοδοτείται ποτέ και για κανέναν λόγο. Κανείς δεν δίνει ούτε 1 λεπτό. Η χρηματοδότηση γίνεται μόνο από δωρεές αντί αγοράς εφημερίδων και περιοδικών, κανένας δε μα κανένας δεν έχει δικαίωμα καθ οιονδήποτε τρόπο να κατακρατήσει χρήματα --\textbf{ΕΚΤΟΣ ΤΗΣ ΒΑΣΙΚΗΣ ΔΙΚΑΣΤΙΚΗΣ ΕΞΑΙΡΕΣΕΩΣ}--  από τις δωρεές οι οποίες υποχρεωτικά πάνε για αύξηση της ενημερωτικής δραστηριότητος και ότι περισσεύει στα ευαγή ιδρύματα και τη διαχείρηση των χρημάτων την κάνει η εκάστοτε κεντρική επιτροπή.
\item \textbf{Δημιουργείται ορκωτό δικαστήριο 33 ορκωτών πολιτών με τυχαία επιλογή από τα μέλη του κινήματος προκειμένου όσοι έχασαν χρήματα από το υπάρχον δωσιλογικό σύστημα και κατά σειράν προτεραιότητος στο κίνημα να τους επιστραφούν τα χρήματα αν υπάρχουν κέρδη από την δραστηριότητα του κινήματος εξ αιτίας διοχετεύσεως περιοδικών και εφημερίδων}.
\item Υποχρεωτική η αλλαγή κεντρικής επιτροπής κάθε τρίμηνο με κλήρωση και όχι με ψηφοφορία.
\item Οι επόμενοι βγαίνουν με κλήρωση και όχι ψηφοφορία.
\item Υποχρεωτικό το ετήσιο συνέδριο δικτυακά ή φυσικά.
\end{itemize}

\textbf{Δομή}

\begin{itemize}
\item Δεν έχει προέδρους, αντιπροέδρους κλπ. Μόνο μία επιτροπή 7μελή η οποία συνεδριάζει μία φορά την εβδομάδα, σκέφτεται και ενεργεί, ανακοινώνοντας δημόσια τι πρόκειται να κάνει στους υπολοίπους. Δεν υπάρχει γραμματεία κλπ. Υπάρχουν καθρεπτικές διεργασίες(mirror servers και blog) ώστε να μην μπορεί να πέσει κάποιο μέρος του συστήματος. 
\item Η κάθε τοπική οργάνωση έχει δικό της δικτυακό τόπο, στον οποίο \textbf{υποχρεούται να προσθέτει ότι και όλοι οι άλλοι έχουν} και μπορεί να διακινεί τα δικά της έντυπα, βιβλία κλπ.
\item Υποχρεωτικά τα συνέδρια δύο φορές το μήνα σε κάθε τοπική οργάνωση.
\item Υποχρεωτική αλλαγή επιτροπής κάθε μήνα και παράδοση στους επομένους με πραγματικά τυχαία επιλογή η οποία καθορίζεται από τους υπολοίπους. Απαγορεύεται πάνω από 3 φορές να είναι κάποιος στην επιτροπή, εκτός της ελλείψεως μελών.
\end{itemize}


\textbf{Σε περίπτωση που αποφασίσει το κίνημα σε ενέργειες με ψήφους άνω του 50\% είναι υποχρεωτικό να συμμετέχουν όλοι σε αυτές τις ενέργειες.}

\textbf{{\Large ΣΕ ΟΛΑ ΤΑ ΑΝΩΤΕΡΩ ΣΕ ΠΕΡΙΠΤΩΣΗ ΥΠΕΡΚΑΛΥΨΕΩΣ ΙΣΧΥΟΥΝ ΤΑ ΒΑΡΥΤΕΡΑ ΑΠΟ ΤΗΝ ΑΠΟΨΗ ΤΩΝ ΠΟΙΝΩΝ.}}


\textbf{Οι νόμοι που υποχρεωτικά θα περάσουν άμεσα και υποχρεωτικά.}
\begin{itemize}

\item  ΠΛΗΡΗΣ ΑΡΝΗΣΗ ΧΡΕΟΥΣ
\item  ΔΗΜΟΨΗΦΙΣΜΑ ΓΙΑ ΕΞΟΔΟ ΑΠΟ Ε.Ε. ΚΑΙ ΕΥΡΩΖΩΝΗ 
\item  ΑΠΟΡΡΙΠΤΕΤΑΙ Η ΣΥΝΘΗΚΗ ΤΗΣ ΛΙΣΣΑΒΟΝΑΣ ΚΑΙ ΟΣΟΙ ΤΗΝ ΠΡΟΣΥΠΕΓΡΑΨΑΝ ΧΑΝΟΥΝ ΚΙΝΗΤΗ ΚΑΙ ΑΚΙΝΗΤΗ ΠΕΡΙΟΥΣΙΑ.
\item  ΔΗΜΟΨΗΦΙΣΜΑ ΓΙΑ ΚΡΑΤΙΚΟΠΟΙΗΣΗ ΤΗΣ ΤΡΑΠΕΖΑΣ ΤΗΣ ΕΛΛΑΔΟΣ.
\item  ΔΗΜΟΨΗΦΙΣΜΑ ΓΙΑ ΚΡΑΤΙΚΟΠΟΙΗΣΗ ΟΛΩΝ ΤΩΝ ΙΔΙΩΤΙΚΩΝ ΤΡΑΠΕΖΩΝ ΚΑΙ ΕΝΟΠΟΙΗΣΗ ΤΟΥΣ ΣΕ ΜΙΑ ΤΡΑΠΕΖΑ ΤΟΥ ΕΛΛΗΝΙΚΟΥ ΕΘΝΟΥΣ ΚΑΙ ΤΟΥ ΕΛΛΗΝΙΚΟΥ ΛΑΟΥ.
\item  ΥΠΟΧΡΕΩΤΙΚΗ Η ΔΗΜΙΟΥΡΓΙΑ ΣΥΝΕΤΑΙΡΙΣΤΙΚΗΣ ΤΡΑΠΕΖΗΣ ΣΕ ΚΑΘΕ ΔΗΜΟ ΚΑΙ ΜΕΣΩ ΑΥΤΗΣ ΟΠΟΙΑΔΗΠΟΤΕ ΣΥΝΑΛΛΑΓΗ ΜΕ ΤΟ ΔΗΜΟΣΙΟ ΜΟΝΟ ΜΕ ΤΗΝ ΧΡΕΩΚΡΑΤΙΚΗ ΔΡΑΧΜΗ ΚΑΙ ΑΠΑΓΟΡΕΥΣΗ ΕΛΕΓΧΟΥ ΤΗΣ ΙΔΙΩΤΕΣ ΕΠΙ ΠΟΙΝΗΣ ΘΑΝΑΤΟΥ.
\item  ΑΝΑΔΡΟΜΙΚΗ ΕΛΕΓΧΟΣ ΚΑΙ ΦΟΡΟΛΟΓΗΣΗ ΟΛΩΝ ΤΩΝ ΕΒΡΑΙΩΝ ΠΟΥ ΚΑΤΟΙΚΟΥΝ ΣΤΗΝ ΕΛΛΑΔΑ ΣΥΜΦΩΝΑ ΜΕ ΤΑ ΑΡΘΡΑ ΤΟΥ ΣΥΝΤΑΓΜΑΤΟΣ ΣΥΜΦΩΝΑ ΜΕ ΤΟ ΟΠΟΙΟ ΕΙΜΑΣΤΕ ΟΛΟΙ ΙΣΟΙ ΚΑΙ ΦΟΛΟΛΟΓΟΥΜΕΘΑ ΑΝΑΛΟΓΑ ΜΕ ΤΗ ΔΥΝΑΜΗ ΜΑΣ.
\item  ΥΠΑΛΛΗΛΟΙ, ΔΗΜΟΣΙΟΙ ΛΕΙΤΟΥΡΓΟΙ ΚΑΙ ΑΝΘΡΩΠΟΙ ΤΟΥ ΣΗΜΕΡΙΝΟΥ ΚΡΑΤΟΥΣ ΠΟΥ ΔΗΜΟΣΙΩΣ ΕΧΟΥΝ ΜΙΛΗΣΕΙ ΥΠΕΡ ΤΗΣ ΠΟΛΥΠΟΛΙΤΙΣΜΙΚΟΤΗΤΟΣ, ΤΗΣ ΠΑΝΘΡΗΣΚΕΙΑΣ Ή ΤΗΣ ΓΕΝΝΕΤΗΣΙΑΣ ΠΡΟΣΒΟΛΗΣ ΑΠΑΓΟΡΕΥΕΤΑΙ ΝΑ ΕΧΟΥΝ ΟΠΟΙΑΔΗΠΟΤΕ ΔΗΜΟΣΙΑ ΘΕΣΗ ΣΤΗ ΔΗΜΟΡΑΤΙΑ.
\item  ΔΗΜΟΣΙΟΠΟΙΗΣΗ ΟΛΩΝ ΤΩΝ ΟΝΟΜΑΤΩΝ ΤΩΝ ΕΒΡΑΙΩΝ, ΚΑΘΩΣ ΑΥΤΑ ΕΧΟΥΝ ΤΡΟΠΟΠΟΙΗΘΕΙ ΩΣΤΕ ΝΑ ΠΡΟΣΕΓΓΙΖΟΥΝ ΤΑ ΕΛΛΗΝΙΚΑ ΟΝΟΜΑΤΑ.
\item  ΑΠΟΠΟΜΠΗ ΤΟΥ ΚΙΣ ΑΠΟ ΤΗΝ ΕΛΛΑΔΑ ΚΑΙ ΚΑΘΕ ΕΒΡΑΪΚΟΥ ΣΧΗΜΑΤΟΣ ΣΤΥΣΤΑΣΕΩΣ ΣΥΜΜΟΡΙΑΣ.
\item  ΑΠΟΛΥΣΗ ΟΛΩΝ ΤΩΝ ΕΒΡΑΙΩΝ ΑΠΟ ΘΕΣΕΙΣ ΕΞΟΥΣΙΑΣ ΚΑΙ ΑΠ' ΟΛΗ ΤΗΝ ΔΙΚΑΙΟΣΥΝΗ.
\item  ΚΛΕΙΣΙΜΟ ΟΛΩΝ ΤΩΝ ΣΤΟΩΝ, ΜΗΔΕΜΙΑΣ ΕΞΑΙΡΟΥΜΕΝΗΣ.
\item  Επαναφορά της θανατικής καταδίκης, για ΤΟ ΕΓΚΛΗΜΑ ΤΗΣ ΕΣΧΑΤΗΣ ΠΡΟΔΟΣΙΑΣ ΚΑΙ ΕΓΚΛΗΜΑΤΑ ΓΕΝΕΤΗΣΙΑΣ ΠΡΟΣΒΟΛΗΣ.
\item  Επαναφορά της θανατικής καταδίκης, για ΤΟ ΕΓΚΛΗΜΑ ΤΗΣ ΑΠΟΚΡΥΨΕΩΣ ΣΤΟΙΧΕΙΩΝ, ΣΚΛΗΡΩΝ ΔΙΣΚΩΝ ΚΛΠ ΑΠΟ ΤΟΥΣ ΑΝΘΡΩΠΟΥΣ ΤΩΝ ΥΠΟΥΡΓΕΙΩΝ ΠΡΟΚΕΙΜΕΝΟΥ ΣΤΗ ΔΙΑΛΥΣΗ ΤΟΥ ΥΠΟΥΡΓΕΙΟΥ ΟΙΚΟΝΟΜΙΚΩΝ, ΤΟΥ ΥΠΟΥΡΓΕΙΟΥ ΠΑΙΔΕΙΑΣ, ΤΟΥ ΥΠΟΥΡΓΕΙΟΥ ΔΙΚΑΙΟΣΥΝΗΣ ΚΑΙ ΛΟΙΠΩΝ ΥΠΟΥΡΓΕΙΩΝ ΝΑ ΜΗΝ ΑΠΟΚΡΥΒΟΥΝ ΣΤΟΙΧΕΙΑ ΓΙΑ ΤΗΝ  ΜΗ ΑΠΟΔΩΣΗ ΕΥΘΥΝΩΝ, ΕΠΕΙΔΗ ΑΥΤΗ Η ΕΝΕΡΓΕΙΑ ΕΙΝΑΙ ΣΥΝΕΡΓΕΙΑ ΣΕ ΕΣΧΑΤΗ ΠΡΟΔΟΣΙΑ.
\item  ΔΗΜΕΥΣΗ ΠΕΡΙΟΥΣΙΩΝ ΟΛΩΝ ΤΩΝ ΜΚΟ ΚΑΙ ΕΚΔΙΩΞΗ ΤΟΥΣ ΑΠΟ ΤΗΝ ΕΛΛΑΔΑ.
\item  ΞΥΡΙΣΜΑ ΚΑΙ ΑΠΟΠΟΜΠΗ ΟΛΩΝ ΤΩΝ ΙΕΡΑΡΧΩΝ ΚΑΙ ΕΝ ΓΕΝΕΙ ΡΑΣΟΦΟΡΩΝ, ΠΟΥ ΥΠΟΣΤΗΡΙΖΟΥΝ Ή ΕΧΟΥΝ ΔΗΜΟΣΙΩΣ ΥΠΟΣΤΗΡΙΞΕΙ ΤΗΝ ΠΑΝΘΡΗΣΚΕΙΑ.
\item  ΕΘΝΙΚΑ ΔΙΚΑΣΤΗΡΙΑ, ΓΙΑ ΟΛΟΥΣ ΟΣΟΥΣ ΥΠΕΓΡΑΨΑΝ ΜΝΗΜΟΝΙΑ ΚΑΙ ΕΒΛΑΨΑΝ ΚΑΘ' ΟΙΟΝΔΗΠΟΤΕ ΤΡΟΠΟ ΚΑΙ ΑΠΟ ΟΠΟΙΑΔΗΠΟΤΕ ΘΕΣΗ, ΤΗΝ ΠΑΤΡΙΔΑ.
\item  ΔΙΑΛΥΣΗ ΤΟΥ ΚΚΕ ΚΑΙ ΠΛΗΡΗΣ ΕΛΕΓΧΟΣ ΣΤΑ ΟΙΚΟΝΟΜΙΚΑ ΤΟΥ.
\item  ΚΑΤΑΡΓΗΣΗ ΚΑΙ ΕΚΔΙΩΞΗ ΑΠΟ ΤΗΝ ΕΛΛΑΔΑ ΤΗΣ ΕΒΡΑΪΚΗΣ SINFULAR LOGIC, ΚΑΤΑΣΧΕΣΗ ΠΕΡΙΟΥΣΙΩΝ ΚΑΙ ΠΕΡΑΣΜΑ ΑΠΟ ΟΡΚΩΤΟ ΔΙΚΑΣΤΗΡΙΟ ΜΕ ΤΗΝ ΚΑΤΗΓΟΡΙΑ ΤΗΣ ΕΣΧΑΤΗΣ ΠΡΟΔΟΣΙΑΣ.
\item  ΟΙ ΔΙΚΑΣΤΕΣ ΠΟΥ ΔΕΝ ΠΗΡΑΝ ΑΠΟΦΑΣΗ ΑΚΥΡΩΣΕΩΣ ΤΩΝ ΤΣΙΠΡΟΕΚΛΟΓΩΝ ΧΑΝΟΥΝ ΚΙΝΗΤΗ ΚΑΙ ΑΚΙΝΗΤΟ ΠΕΡΙΟΥΣΙΑ, ΤΟ ΔΙΚΑΙΩΜΑ ΑΣΚΗΣΕΩΣ ΤΟΥ ΕΠΑΓΓΕΛΜΑΤΟΣ ΤΟΥ ΔΙΚΑΣΤΟΥ ΚΑΙ ΤΟΥ ΔΙΚΗΓΟΡΟΥ ΚΑΙ ΠΕΡΝΟΥΝ ΑΠΟ ΟΡΚΩΤΟ ΔΙΚΑΣΤΗΡΙΟ ΜΕ ΤΗΝ ΚΑΤΗΓΟΡΙΑ ΤΗΣ ΣΥΝΕΡΓΕΙΑΣ ΣΕ ΚΑΚΟΥΡΓΗΜΑΤΙΚΕΣ ΠΡΑΞΕΙΣ ΚΑΙ ΕΧΑΤΗ ΠΡΟΔΟΣΙΑ.
\item  ΟΙ ΔΙΚΑΣΤΕΣ ΠΟΥ ΠΉΡΑΝ ΜΕΤΡΑ ΣΕ ΒΑΡΟΣ ΠΟΛΙΤΩΝ ΚΑΙ ΥΠΕΡ ΤΩΝ ΤΡΑΠΕΖΩΝ ΧΑΝΟΥΝ ΚΙΝΗΤΗ ΚΑΙ ΑΚΙΝΗΤΟ ΠΕΡΙΟΥΣΙΑ ΚΑΙ ΤΟ ΔΙΚΑΙΩΜΑ ΤΟΥ "ΕΙΝΑΙ ΔΙΚΑΣΤΑΙ".
\item  ΟΠΟΙΟΔΗΠΟΤΕ ΜΕΛΟΣ ΤΗΔ ΔΙΚΑΙΟΣΥΝΗΣ ΕΠΡΑΞΕ ΥΠΕΡ ΤΩΝ ΤΡΑΠΕΖΩΝ ΚΑΙ ΚΑΤΑ ΤΩΝ ΠΟΛΙΤΩΝ ΧΑΝΕΙ ΚΙΝΗΤΗ ΚΑΙ ΑΚΙΝΗΤΟ ΠΕΡΙΟΥΣΙΑ.
\item  Εισαγγελείς και δικαστές που πήραν αποφάσεις ενάντια πολιτών και υπέρ τραπεζιτών ή πολιτικών αδίκως καθώς εισαγγελείς που το κατώτερο κείμενο παρά θεώρησαν κατάσχεται η κινητή και ακίνητη περιουσία και παίρνουν από ορκωτό δικαστήριο με την κατηγορία της συνέργειας σε έσχατη προδοσία ενάντια σε κάθε δίκαιο τίμιο και ηθικό.
\item  Όσοι δημοσίως αντιταχθούν με δόλο ή με σκοπό να μην επιτύχει το ζητούμενο θα περάσουν από ορκωτό δικαστήριο 33 ατόμων με το ερώτημα της κατασχέσεως κινητής και ακινήτου περιουσίας επειδή στην ουσία εναντιώνονται στην πρόοδο της χώρας και θέλουν να κυβερνάται η χώρα από κλίκες και συμμορίες, από εβραιομασσώνους και όργανα του σιωνισμού.
\item  Όσοι εισαγγελείς δεν εφήρμοσαν την εν γένει παρακάτω εντολή κατάσχεται κινητή και ακίνητος περιουσία ενώ απαγορεύεται να λειτουργούν υπό αυτύη την ιδιότητα, χάνουν την ιδιότητα του Έλληνος πολίτου, απαγορεύεται να συμμετέχουν στα κοινά.

\begin{quote}
	{\centering
	\textbf{Εἰς τὸ ὄνομα τῆς Ἁγίας καὶ Ὀμοουσίου καὶ Ἀδιαιρέτου Τριάδος \\ ὡς τῆς Ὑπερτάτου Ἀρχῆς τοῦ Συντάγματος}


Τάδε λέγει Κύριος:

Ἑγὼ εἰμι τὸ Ἄλφα καὶ τὸ Ὠμέγα, ὁ Πρῶτος καὶ ὁ Ἔσχατος, ἡ Ἀρχὴ καὶ τὸ Τέλος. Ἐμοῦ ἀκούετε.

ΕΝΩΠΙΟΝ ΠΑΣΗΣ ΑΡΧΗΣ ΔΙΚΑΙΟΥ

ΕΙΣ ΤΟ ΟΝΟΜΑ ΤΟΥ ΕΛΛΗΝΙΚΟΥ ΛΑΟΥ

}

Κε εισαγγελεύ.

Όποιος εισαγγελεύς λέει ότι μπορεί μια \textbf{συμμορία} που ενεργεί σε εγκλήματα κατά των Ελλήνων και κατά της Πατρίδος, να προστατεύεται από το νόμο, τότε ή πρόκειται περί περί συνεργού και οργάνου της συμμορίας.

Όποιος εισαγγελεύς λέει ότι δεν μπορεί να εκδώσει εντολή συλλήψεως των δωσιλόγων, συμμοριτών  και συνομωτών επειδή προστατεύονται από το Σύνταγμα της χώρας, ερμηνεύει λάθος και κατά το δοκούν το Σύνταγμα καὶ παραθεωρεί τὴν Ὑπέρτατο Ἀρχή, "μασσωνικῷ τῷ τρόπῳ".

Όποιος εισαγγελεύς λέει ότι δεν μπορεί να εκδώσει εντολή συλλήψεως των δωσιλόγων, συμμοριτών  και συνομωτών επειδή προστατεύονται από το Σύνταγμα, κατηγορεί το Σύνταγμα και τον Συνταγματικό Νομοθέτη ότι σχεδίασε τον καταστατικό Χάρτη τῆς Χώρας "ὡς ἑπομένου τῆς Ὑπερτάτου Ἀρχῆς", τον Υπέρτατο Νόμο του Κράτους και της Ελληνικής Πολιτείας τέτοιο ώστε να μπορούν συμμμορίες να κυβερνούν ανενόχλητα.

Αυτές όμως οι ερμηνείες είναι για τους αφελείς, ώστε να έχουμε να κάνουμε με συμμορία και συνομωσία η οποία περιλαμβάνει και τους εισαγγελείς οι οποίοι παραβαίνουν το καθήκον τους.

Και εξηγούμαι.

Το άρθρο 86 του Συντάγματος προστατεύει τους νομοθέτες, δηλαδή τους βουλευτές και ορθώς. Γιατί η προστασία της νομοθεσίας από κάθε εμπαθή πολίτη είναι δικαίωμα του βουλευτή. Είναι δικαίωμα να είναι ελεύθερος να αποφασίσει για το Δίκαιο και την Τάξη στην Πολιτεία, γι αυτό το λόγο και η ίδια η Βουλή αποφασίζει για το αν θα αίρει την Βουλευτική ασυλία.

Όμως δεν αφαιρεί τη δυνατότητα συλλήψεως από τις Αστυνομικές και Εισαγγελικές Αρχές όταν έχουμε προφανές κακούργημα. Εκεί δεν υπάρχει βουλευτική ασυλία ειδικά στα ποινικά αδικήματα, πράγμα το οποίο είναι προφανές, είναι δε τόσο προφανές στην κοινή λογική ώστε ο εισαγγελεύς κος Τέντες, τέως Πρόεδρος του Αρείου Πάγου, ερμηνεύοντας το Σύνταγμα δίνει οδηγίες περί της συλλήψεως των βουλευτών σε περίπτωση προφανούς κακουργήματος, γιατί εκεί δεν υπάρχει βουλευτική ασυλία.

Όμως η βουλευτική ασυλία ωφείλεται στο άρθρο 86 του Συντάγματος, το οποίο είναι τρεπτό και μεταβλητό και υπολείπεται του άρθρου 4 το οποίο δεν μεταβάλλεται και σύμφωνα με το οποίο \textbf{όλοι οι Έλληνες είμαστε ίσοι ενώπιον του νόμου}. 

Αυτό σημαίνει ότι οι \textbf{συμμορίτες που παραβιάζουν το Σύνταγμα της χώρας} πρέπει να \textbf{συλλαμβάνονται πάραυτα}, γιατί διαφορετικά όποιος εισαγγελέας λέει και πράττει το αντίθετο, σφάλλει ενώπιον του νόμου, παραβαίνει το καθήκον του και \textbf{πράττει το μείζον έγκλημα. ΚΑΤΗΓΟΡΕΙ ΤΟ ΣΥΝΤΑΓΜΑ ΤΗΣ ΧΩΡΑΣ}. Λέει με άλλα λόγια ότι το \textbf{Σύνταγμα της χώρας προστατεύει απατεώνες, δωσιλόγους και συμμορίτες,} πράγμα το οποίο δεν ισχύει, ενώ θα έπρεπε με το άρθρο 4 να έχει δώσει εντολή συλλήψεως. 

Λέει ότι το Σύνταγμα προστατεύει τους προδότες, όταν την ίδια στιγμή το καταπατούν, πράγμα άτοπο.

\textbf{ΔΕΝ ΥΠΑΡΧΕΙ ΚΕΙΜΕΝΟ ΣΕ ΧΑΡΤΙ ΓΡΑΜΜΕΝΟ ΤΟ ΟΠΟΙΟ ΟΤΑΝ ΚΑΤΑΠΑΤΕΙΤΑΙ ΝΑ ΕΦΑΡΜΟΖΕΤΑΙ. ΕΙΝΑΙ ΑΤΟΠΟ ΩΣ ΛΟΓΙΚΗ. \\ ΔΗΛΑΔΗ ΔΕΝ ΜΠΟΡΕΙ ΤΟ ΣΥΝΤΑΓΜΑ ΝΑ ΤΟ ΚΑΤΑΠΑΤΟΥΝ ΚΑΙ ΤΗΝ ΙΔΙΑ ΣΤΙΓΜΗ ΝΑ ΤΟ ΕΝΕΡΓΟΥΝ ΩΣ ΜΗ ΚΑΤΑΠΑΤΗΜΕΝΟ. \\ ΔΕΝ ΓΙΝΕΤΑΙ ΝΑ ΚΑΝΕΙΣ ΧΡΗΣΗ ΑΥΤΟΥ ΠΟΥ ΚΑΤΑΠΑΤΑΣ ΩΣ ΜΗ ΚΑΤΑΠΑΤΗΜΕΝΟ. \\ ΜΕ ΑΛΛΑ ΛΟΓΙΑ ΚΑΝΕΝΑ ΣΥΝΤΑΓΜΑ ΔΕΝ ΠΡΟΣΤΑΤΕΥΕΙ ΚΑΝΕΝΑΝ Ο ΟΠΟΙΟΣ ΤΟ ΚΑΤΑΠΑΤΑ. }

Πολλές φορές μέσα στη Βουλή των Ελλήνων βουλευτές και υπουργοί αναφέρονται στο Σύνταγμα της χώρας ενώ την ίδια στιγμή το καταπατούν, εφαρμόζουν αποφάσεις των εχθρών μας και εφαρμόζουν τα μνημόνια τα οποία έχουν απορρίψει οι Έλληνες, αποδεικνύοντας ότι πρόκειται περί \textbf{πρακτόρων των ξένων}.

Εις επίρρωσιν τούτων κε εισαγγελεύ υπάρχει το «ιδιότυπο αδίκημα» της \textbf{εθνικής αναξιότητος}.

Η εθνική αναξιότητα αποτελεί στην Ελλάδα «ιδιότυπο αδίκημα», που προσδιορίστηκε με τη Έκτη και τη Δωδέκατη Συντακτική Πράξη του 1945 και με τον Α.Ν. 5333/1945 και τους νόμους 217, 271, 295 και 332 του 1945 με τους οποίους και προσδιορίζονταν επακριβώς η υπόσταση των εγκλημάτων της εθνικής αναξιότητας.

Ο όρος «εθνική αναξιότητα», αφορά στη \textbf{συμπεριφορά πολίτη μιας χώρας που στρέφεται ενάντια της πατρίδας του}, ενάντια της εθνικής αξιοπρέπειας ή την θίγει καθ΄ οιονδήποτε τρόπο.

Ο νόμος προβλέπει ότι τιμωρούνται:

Όσοι κατέχοντες δημόσια θέση έγιναν \textbf{συνειδητά όργανα του εχθρού ή διευκόλυναν το έργο του}.

Όσοι έγιναν \textbf{συνειδητά όργανα του εχθρού προς διάδοση της προπαγάνδας αυτού}.

Όσοι προέβησαν σε \textbf{πράξεις βίας κατά Ελλήνων ένεκα δράσης αυτών κατά του εχθρού}.

Όσοι διετέλεσαν \textbf{αρχηγοί ή οδηγοί κινήσεων κατά της ακεραιότητας της χώρας}.

Όσοι δια της \textbf{οικονομικής τους συνεργασίας με τον εχθρό προκάλεσαν ζημιές στον ελληνικό λαό, ή αποκόμισαν οικονομικά οφέλη} από τον εχθρό.

\textbf{Οσοι συνεργάσθηκαν με τον εχθρό κατά τρόπο ανάξιο Έλληνος πολίτη και έθιξαν την εθνική αξιοπρέπεια}.

Αυτό σημαίνει ότι το \textbf{σύνολο εν γένει των εισαγγελέων πρέπει να καταδικαστεί για εθνική προδοσία} μέσω της αναξιότητας και ήδη Γκουτζαμάνη, Θάνου και λοιποί εισαγγελείς έχουν ήδη καταμηνυθεί για πλήθος ποινικών αδικημάτων που απορρέουν από τα πιο πάνω εγκλήματα κατά του Ελληνικού Λαού.

Τελικά είναι προφανές ότι αν οι εισαγγελείς της χώρας δεν έπρατταν \textbf{παράβαση καθήκοντος}, η χώρα δεν θα βρισκότανε σε τέτοια δεινά, ώστε αβίαστα να βγαίνει το συμπέρασμα ότι \textbf{οι εισαγγελείς ανήκουν σε γενικές γραμμές στην ίδια συμμορία}.


\end{quote}


\item  Οι ένορκοι εξατάζονται πρώτα από τους δικηγόρους των διαδίκων και κοινή συναινέσει επιλέγονται μετά την τυχαία επιλογή τους.
\item  Απογορεύεται στα Δημοτικά η χρήση βαθμολογίας μέχρι την 4η Δημοτικού. Μέχρι την 4η Δημοτικού τα παιδιά εκπαιδεύονται στη σωστή συμπεριφορά σύμφωνα με το Ιαπωνικό πρότυπο.
\item  Στην εκπαίδευση εφαρμόζεται καλλιγραφία, χορός, μη βαθμολογία, μέχρι και την 4η Δημοτικού, εθνικός ύμνος, πρωϊνή προσευχή.
\item  Κεφάλαια άνω του 3πλασίου του μέσου όρου κατά Δήμο κατά κεφαλήν απαγορεύεται να κατέχονται και κατάσχονται από τους δήμους για επενδύσεις στη βάση των λαϊκών Συνεταιρισμών. Με αυτό τον τρόπο θα αναπτυχθούν άμεσα εταιρείες παραγωγικές και εξαγωγικές και οι Έλληνες θα αναπτύξουν μια ταχύτατη και μη ελεγχόμενη από την εβραιομασσωνία και το σιωνισμό ανάπτυξη, στη βάση ηθικού και τελείου συστήματος στο οποίο δεν θα μπορούν πλέον οι διεθνείς απατεώνες και λοποδύτες με τους ανθρώπους τους να εισέλθουν ως διοίκηση. \\ Θα πρέπει να θυμηθούμε ότι η πτώση του Βυζαντίου ξεκινούσε από την απληστία των πλουσίων οι οποίοι ήθελαν όλα να τα έχουν στα χέρια τους ενώ ο ίδιος ο λαός υπέφερε Όπως και το ίδιο συνέβαινε σε κάθε εποχή και σε κάθε πτώση αυτοκρατορίας. Να θυμηθούμε ότι η αρχαία Αθήνα επέζησε εξαιτίας του γεγονότος ότι ο Σόλων μπόρεσε και έπεισε τους πλουσίους να δώσουν τα κτήματά τους 
στους πτωχούς και έτσι αναπτύχθηκε η Αθήνα αναπτύχθηκε η Δημοκρατία έγινε πολύ δυνατό το κράτος και μπόρεσε και άντεξε και γλύτωσε τον παντελή αφανισμό από τους Πέρσες αυτό είναι βάσις κάθε σκέψεως, να ενεργηθούν οι δυνάμεις των πολιτών με την ελευθερία και τη γνώση και η συνεργατικότητα μεταξύ των πολιτών, πλέον ως θεσμός. \\ Γιατί δεν θα μπορεί κάποιος να έχει πολλά αγαθά αν ο διπλανός του μένει αδρανής, και αποκτώντας ο διπλανός του αγαθά ανεβάζει τον μέσο όρο αγαθών, οπότε δεν κινδυνεύει σε κατάσχεση. \\Ακόμη οι συνταξιούχοι έχουν συμφέρον να προωθούν σε παραγωγικότητα και γνώση τους νέους ώστε περισσότερα αγαθά σημαίνει και μεγαλύτερες συντάξεις.\\Με αυτό τον τρόπο και αυτές τις σκέψει δημιουργείται ένας μηχανισμός Δημοκρατικός, λίαν αναπτυξιακός και παραγωγικός προς το συμφέρον όλων και των πλουσίων και των πτωχών.\\ Γιατί κινδυνεύοντας το κράτος από ξένες επιβουλές δεν γλυτώνει ούτε ο φτωχός ούτε ο πλούσιος. \\ Ο σχεδιασμός αυτης της λειτουργίας του Έθνους είναι αμεσοδημοκρατική, εντελώς αναπτυξιακή και στη βάση της δημιουργίας πραγματικά πανίσχυρου ΑΜΕΣΟΔΗΜΟΚΡΑΤΙΚΟΥ ΜΗ ΕΛΕΓΧΟΜΕΝΟΥ ΚΡΑΤΙΚΟΥ ΜΗΧΑΝΙΣΜΟΥ.\\Η προοπτική η παιδεία και όλα τα καλά μπορούν να προκύψουν όταν η κοινωνία λειτουργήσει κοινοτικά.
\item  Τα κεφάλια υποχρεωτικά να βρίσκονται εντός της επικράτειας.
\item  \textbf{ΤΟ ΦΟΡΟΛΟΓΙΚΟ ΚΑΘΕΣΤΩΣ ΤΟΥ ΚΑΛΛΙΕΛΑΙΟΥ ΚΙΝΗΜΑΤΟΣ ΙΣΧΥΕΙ ΜΟΝΟ ΓΙΑ ΤΑ ΜΕΛΗ ΤΟΥ ΚΙΝΗΜΑΤΟΣ. ΟΙ ΥΠΟΛΟΙΠΟΙ ΟΙ ΟΠΟΙΟΙ ΔΕΝ ΕΝΔΙΑΦΕΡΘΗΣΑΝ ΓΙΑ ΤΗΝ ΕΛΕΥΘΕΡΙΑ ΤΗΣ ΠΑΤΡΙΔΟΣ ΤΩΝ, ΔΕΝ ΕΝΔΙΑΦΕΡΘΗΣΑΝ ΓΙΑ ΤΟ ΜΕΛΛΟΝ ΤΟΥ ΤΟΠΟΥ ΤΟΥΣ ΚΑΙ ΑΦΗΣΑΝ ΤΟΝ ΕΒΡΑΪΚΟ ΣΧΕΔΙΑΣΜΟ ΚΑΙ ΤΗΝ ΑΠΑΤΗ ΠΟΥ ΕΣΤΗΣΑΝ ΟΙ ΕΒΡΑΙΟΙ ΣΕ ΣΥΝΕΡΓΑΣΙΑ ΜΕ ΤΟΥΣ ΧΙΤΛΕΡΙΚΟΥΣ ΝΑ ΣΥΝΕΧΙΖΕΤΑΙ, ΟΙ ΟΠΟΙΟΙ ΔΕΝ ΕΝΑΝΑΝ ΤΟ ΧΡΕΟΣ ΤΟΥΣ ΑΠΕΝΑΝΤΙ ΣΤΗΝ ΠΑΤΡΙΔΑ ΘΑ ΕΧΟΥΝ ΤΟ ΙΔΙΟ ΦΟΡΟΛΟΓΙΚΟ ΚΑΘΕΣΤΩΣ ΜΕ ΑΥΤΟ ΤΩΝ ΜΝΗΜΟΝΙΩΝ ΚΑΙ ΕΩΣ 99 ΧΡΟΝΙΑ ΕΝΦΙΑ ΚΛΠ, ΕΣΟΔΑ ΤΑ ΟΠΟΙΑ ΘΑ ΠΗΓΑΙΝΟΥΝ ΣΤΗΝ ΠΑΤΡΙΔΑ ΚΑΙ ΟΧΙ ΣΤΟ ΕΞΩΤΕΡΙΚΟ.}

\item  \textbf{ΟΙΚΟΝΟΜΙΑ}

\item Εντὸς 3 μηνών δημοσίευση λογιστικού ελέγχου του χρέους και αναφορές στα φυσικά, νομικά κλπ πρόσωπα που εμπλέχθηκαν και απόδωση ευθυνών με Ορκωτό Πανελλαδικό Δικαστήριο με ορκωτούς από κάθε νομό.
\item \textbf{Τα μέλη του κινήματος δικαιούνται να δημιουργούν οποιαδήποτε ΕΠΙΧΕΙΡΗΣΗ ΘΕΛΟΥΝ ΜΕ ΑΡΙΘΜΟ ΤΑΥΤΟΠΟΙΗΣΗΣ ΤΗΝ ΗΜΕΡΟΜΗΝΙΑ ΚΑΤΑΘΕΣΕΩΣ ΤΗΣ ΣΥΜΒΟΛΑΙΟΓΡΑΦΙΚΗΣ ΠΡΑΞΕΩΣ ΚΑΙ ΤΗΝ ΩΡΑ ΚΑΤΑΘΕΣΕΩΣ ΚΑΙ ΜΕ ΤΕΛΙΚΟ ΣΤΟΙΧΕΙΟ ΤΟΝ ΑΥΞΟΝΤΑ ΑΡΙΘΜΟ ΤΟΥ ΠΟΛΙΤΟΥ ΣΤΗΝ ΠΡΑΞΗ.} 
\item \textbf{Τα μέλη του κινήματος που δημιουργούν Ελληνική Επιχείρηση καταθέτουν σε συμβολαιογράφο την έναρξη και τον τίτλο της επιχειρήσεώς των καθώς ζητούν από τον Συμβολαιογράφο Σφραγίδα και Υπογραφή στο βιβλίο καταγραφών προκειμένου να φορολογηθούν στο μέλλον όταν ελευθερωθεί η Πατρίδα και επανέλθει η Δημοκρατία, συντάσσεται δε πράξη με τον συμβολαιογράφο σχετικά με το γεγονός. Στη συνέχεια καταθέτουν αντίγραφο στην εισαγγελία με δικαστικό επιμελητή προκειμένου να ενημερωθεί η υπάρχουσα αρχή ότι οι πολίτες δεν αναγνωρίζουν αντισυνταγματικά μέτρα και πολιτικές στοχεύσεις του εβραιοσιωνισμού με τους ντόπιους δωσιλόγους και προδότες, αλλά και να γνωρίζουν οι διάφοροι υπάλληλοι των δωσιλόγων ότι στο μέλλον θα περάσουν από ορκωτά δικαστήρια}.
\item Καταργοῦνται \textbf{για τα μέλη} τὰ ΤΑΞΙΣ, Φ.Π.Α, ἀποδείξεις, ταμειακὲς μηχανές, Ὑπουργεῖο Δικαιοσύνης, οἰκονομικὲς ἐφορίες, παρακαταθηκῶν καὶ δανείων καὶ ὅλος αὐτὸς ὁ ἀντιπαραγωγικὸς καὶ γραφειοκρατικὸς συρφέτος.
\item Αίρονται τα capital controls και τιμωρούνται με δήμευση περιουσιών όσων τα δημιούργησαν.
\item Απογορεύεται καθ οιονδήποτε τρόπον οποιοσδήποτε φόρος σε εμπράγματο υλικό(ακίνητη και κινητή περιουσία), κι επιπρόσθετος φόρος σε προϊόντα(οινόπνευμα, κρασί κλπ κλπ), όπως και σε μεταβιβάσεις ακινήτων όπως ακόμη και σε πρόστιμα. ΔΕΝ ΝΟΟΥΝΤΑΙ ΠΡΟΣΤΙΜΑ ΓΙΑ ΚΑΝΕΝΑ ΛΟΓΟ. ΤΑ ΧΡΩΣΤΟΥΜΕΝΑ ΠΑΝΕ ΔΙΚΑΣΤΙΚΑ ΑΜΕΣΑ.
\item Απογορεύεται η εισαγωγή προϊόντων από τὸ εξωτερικό αν πρώτα δεν έχουν εξαντληθεί τα διαθέσιμα, εκτός της δικαστικής αποφάσεως διά ειδικούς λόγους(παράδειγμα κάποια υλικά του τόπου δεν είναι της τεχνολογικής αξίας που ζητείται για ένα εργοστάσιο, οπότε εντός ημερών μπορεί ο ενδιαφερόμενος να έχει δικαστική εντολή εισαγωγής, ή αν κάποιος τα δίνει πολύ ακριβά), η οποία απόφασις δημοσιεύεται στην εφημερίδα του Δήμου.
\item Κάθε Δήμος υποχρεούται σε δημιουργία ηλεκτρονικού \textbf{ΔΗΜΟΠΡΑΤΗΡΙΟΥ} μέσα από το οποίο πωλούνται τα προϊόντα του Δήμου στους εμπόρους και μόνο μέσα από τις συνεταιριστικές τράπεζες. 
\item Κάθε Δήμος έχει τον δικό του \textbf{Οργανισμό Φυτοφαρμάκων και Ζωῖκής Παραγωγής} ενώ απαγορεύεται η αγορά φυτοφαρμάκων από πολυεθνικές και λιπασμάτων μέσω τραπεζών με επιτόκια κλπ. Σταματά πλέον η απάτη με τα λιπάσματα, τους τραπεζίτες και τους συνεταιρισμούς.
\item Δημιουργείται ο \textbf{ΟΡΓΑΝΙΣΜΟΣ ΕΚΔΟΣΕΩΣ ΔΙΔΑΚΤΙΚΩΝ ΒΙΒΛΙΩΝ} με σήμα την κουκουβάγια όπως παλαιά και ο οποίος εκδίδει τα βιβλία των σχολείων και τα συγγράμματα των πανεπιστημίων. Η έκδοση γίνεται εντός της επικρατείας. Οι υπάλληλοι είναι αιρετοί στο χρόνο της κυβερνήσεως και επιλέγονται από την κυβέρνηση. Είναι υποχρεωμένοι να είναι απόφοιτοι Ηλεκτρολόγων Μηχανικών και Τεχνολογίας υπολογιστών, Μηχανολόγοι Μηχανικοί, Ναυπηγοί ή Αεροναυπηγοί, να έχουν άπταιστη γνώση του \LaTeX, της c, c++, python, scribus, inkscape, apache, Libre office, Wordpress, και ανάλογες τεχνολογίες \textbf{ΕΞΑΙΡΟΥΜΕΝΩΝ ΤΩΝ ΤΕΧΝΟΛΟΓΙΩΝ MICROSOFT ΚΑΙ ΚΛΕΙΣΤΟΥ ΚΩΔΙΚΑ ΤΕΧΝΟΛΟΓΙΩΝ}.
\item \textbf{Καταργείται το Χρηματιστήριο και κάθε είδους απάτη με μη εμπράγματο υλικό}. 
\item Εταιρείες που έχουν απομειώσει τα δικαιώματα των Ελλήνων Πολιτών με εξαναγκασμό σε αναγκαστική και υποχρεωτική συνεργασία με τους τραπεζίτες κατάσχεται κινητή και ακίνητος περιουσία.
\item Η μέγιστη θητεία που μπορεί κάποιος να έχει θέση σε συνεταιρισμό είναι 3 έτη.
\item Απογορεύεται οιαδήποτε ποσόστωση εκτός δικαστικής αποφάσεως. Ελιές, σταφύλια παύουν πλέον να έχουν ποσόστωση όπως οι πράκτορες της ΕΕ κάνανε. Τώρα παύει η ΕΕ να τρώει τον κόπο μας που είναι το ΦΠΑ το οποίο παίρνει και αποδίδει πίσω ένα μικρό ποσοστό. Η κομπίνα των χιτλερικών και εβραίων που στήσανε την Ε.Ε. τελειώνει.
\item Επειδή πρέπει να πάψουμε να συνεχίζουμε να ταΐζουμε τους Γερμανούς θα δημοσιοποιηθεί πλήρης έλεγχος της απάτης που στήθηκε και από ποιούς εντός 3 μηνών από την ανάληψη της κυβερνήσεως.
\item Άμεση απαίτηση για απόδωση εξηγήσεων από την Ε.Ε, για το αν διοικείται από εβραίους, διεκδίκηση των ποσών του ΦΠΑ που τους ταΐζουμε τόσα χρόνια, μέσω ορκωτού πανελλαδικού δικαστηρίου ως ένα από τα πρώτα μελλήματα της νέας κυβερνήσεως.
\item Απαγορεύεται οποιοσδήποτε φόρος σε εισαγωγές καθ οιονδήποτε τρόπο.
\item Απαγόρευση ισχύος κάθε συμβάσεως της οιαδήποτε ενώσεως ενισχύει τις πολυεθνικές, ιδιωτικούς στρατούς, μεταλλαγμένα, ειδικά φορολογικά καθεστώτα, ζώνες δουλοπαροίκων εργαζομένων, πόλεις αλλοεθνών και λοιπές \textbf{ανοησίες των διεθνών και εγχωρίων προδοτών και σφουγκοκωλαρίων αυτών}.
\item Κάθε Δήμος υποχρεούται στη δημιουργία ΤΡΙΩΝ τουλάχιστον διαφορετικών συνεταιριστικών τραπεζών ανά (\textbf{ΔΕΚΑ ΧΙΛΙΑΔΩΝ ΚΑΤΟΙΚΩΝ ΑΝΑ ΥΠΟΚΑΤΑΣΤΗΜΑ ΤΡΑΠΕΖΗΣ}) με την παράλληλη δημιουργία καταπιστεύματος. Οἱ τράπεζες εἶναι κρατικές λαϊκής βάσεως, δηλαδή ανήκουν στους κατοικους με την επίβλεψη του κρατικού μηχανισμού του ΔΗΜΟΥ και την υψηλή επίβλεψη της κυβερνήσεως της ΔΗΜΟΚΡΑΤΙΑΣ. Ἐπιβλέπονται ἀπὸ τοὺς τοπικοὺς εἰσαγγελεῖς καὶ τοῦ ὄργανα τοῦ Δήμου.
\item Τὸ νόμισμα τῆς Ἑλλάδος καὶ ἡ πηγὴ τῆς ἐλευθερίας της εἶναι ἡ ΔΡΑΧΜΗ. Νόμισμα χρεωκρατικὸ τὸ ὁποῖο ἐκδίδεται μόνο ἀπὸ τὴν Ἑλληνικὴν Κυβέρνησιν καὶ ποτέ, μὰ ποτὲ εἰς τὸν αἰώνα δὲν ἐπιτρέπεται νὰ δοθεῖ ἄδεια σὲ οἱονδήποτε ἰδιώτη νὰ τὸ τυπώσει καθ᾿ οἱονδήποτε τρόπο. Το νόμισμα αυτό δεν πωλείται, δεν μπαίνει στο χρηματιστήριο και δεν αποτελεί μέσον συναλλαγής εκτός της επικρατείας.
\item Ὅσοι Ἕλληνες διαφύγουν στὸ ἐξωτερικὸ γιὰ νὰ διασώσουν τὰ λεφτά τους ὥστε αὐτὰ νὰ μὴν μποῦνε στὴν παραγωγικὴ διαδικασία καὶ πρόοδο τῆς Πατρίδος, νὰ ἐκδίδεται ἄμεσα καὶ ὑποχρεωτικὰ ἀπὸ τὴν κυβέρνησιν ἔνταλμα συλλήψεως καὶ κατασχέσεως τῶν περιουσιῶν των. Αὐτοὶ θεωροῦνται προδότες.
\item Ἑταιρεῖες οἱ ὁποῖες ἔχουν σχεδιάσει προϊόντα ὥστε τὰ ὑποσυστήματά τους νὰ ἐξαπατοῦν, νὰ ἀποκλείονται ἀπὸ τὴν Ἑλλάδα.
\item Μετὰ τὰ ἑξήντα χρόνια ὁ καθένας λειτουργεῖ συμβουλευτικά. Μπαίνει σὲ σύνταξη εἴτε ἔχει δουλέψει καὶ ὄχι εἴτε συνεχίζει νὰ δουλεύει.
\item Οἱ πολυεθνικὲς φορολογοῦνται στὸ 12\% ἐπὶ τοῦ ἀκαθαρίστου καί ἀναδρομικά.
\item Ἀπαγορεύεται ἡ χρήση ἀξιοκρατικῶν νομισμάτων καθ᾿ οἱονδήποτε τρόπο εντός της επικρατείας.
\item Τὰ κέρδη τῶν ἐξαγωγῶν ἀπαγορεύεται νὰ γίνονται λογιστικὲς μονάδες, ἀλλὰ μόνο ἐμπράγματα δικαιώματα, μὲ πρώτη προτίμηση τὴν ἐναπόθεση μετάλλων.
\item Ἡ ἐναπόθεσις μετάλλων θὰ γίνεται σὲ καταπίστευμα ποὺ ὑποχρεοῦται νὰ διατηρεῖ ἡ τράπεζα τοῦ Δήμου.
\item Κάθε Δῆμος ὑποχρεοῦνται σὲ δημιουργία συνεταιριστικῆς τραπέζης, μὲ ἐπόπτες τοὺς εἰσαγγελεῖς καὶ τὴν ἐπιτροπὴ ἐντίμων πολιτῶν, καὶ τὸν ἀρχηγὸ τῆς ἀστυνομίας καὶ τὸν οἰκεῖο ἐπίσκοπο.
\item Κάθε συνεταιριστικὴ τράπεζα δύναται νὰ προβεὶ σὲ ἐκτύπωση χρεοκρατικοὺ τοπικοῦ νομίσματος, συνδεδεμένο μὲ τὸ ἐθνικό, τὸ ὁποῖο εἶναι καὶ αὐτὸ χρεωκρατικό, καὶ αὐτὴ ἡ διαδικασία θὰ ἐκτινάξει τὴν παραγωγικότητα καὶ τὶς ἐξαγωγές, θὰ μηδενίσει τὴν ἀνεργία καὶ μὲ τὴ χρήση τῆς κυκλικῆς μεταναστεύσεως τῶν 4 ἐτῶν, θὰ δημιουργήσει τεράστια ἀνάπτυξη. Γιὰ νὰ γίνει αὐτὸ πρέπει νὰ περάσει από τοπικὸ Δημοψήφισμα.
\item Γιὰ δὲ τὴν τεράστα ἀνάπτυξη πρέπει καὶ ὑποχρεοῦται ἡ συνεταιριστικὴ τράπεζα νὰ δημιουργεῖ καταπίστευμα, τὰ προϊόντα δὲ ποὺ θὰ εἰσῤῥέουν σὲ αὐτό, θὰ εἶναι ἀφενὸς μὲν φορολογημένα, ἂφ ἑτέρου δὲ πλέον μὴ κινδυνεύοντα. Ἡ ἐναπόθεσις προϊόντων ὅπως παράδειγμα τὰ μέταλλα ἀπὸ διαδικασίες ἐξαγωγικὲς δὲν φορολογοῦνται.
\item Καταργεῖται ὁ ΓΟΚ, τὸ ΤΕΕ, ἡ Πολεοδομία, ἡ ἐφορία(\textbf{ἐκτὸς ἑνὸς τμήματος γιὰ τὰ μὴ μέλη}), οἱ λογιστές, οἱ γραφιάδες τοῦ Δημοσίου καὶ ἁπλοποιεῖται ἡ φορολογικὴ νομοθεσία ἡ ὁποία ἀνατίθεται στοὺς Δήμους. Δὲν πρέπει νὰ ξεχνοῦμε ὅτι ὅλοι αὐτοὶ μόνο κακὸ κάνουν στὴν κοινωνία, ὄντας ἀντιπαραγωγικοί, ἀντιερευνητικοί, ἀντιεξαγωγικοί, ἀντιλειτουργικοί, καὶ πρώτιστα λαδωμένοι. Νὰ βγάλουμε τὴν πλάτη μας τὸ μοίασμα τῶν κρατικοδίαιτων ἀπατεώνων.
\item Οἱ διευθυντὲς τῶν τραπεζῶν ὁρίζονται ἀπὸ τὴν κυβέρνηση μὲ τὴ συμφωνία τῶν τοπικῶν κοινωνιῶν, εἶναι αἱρετοὶ γιὰ ένα χρόνο, μὲ τὴ σύμφωνο γνώμη τῶν αἱρετῶν ἀρχόντων τοῦ Δήμου.
\item Ἀπεγκλωβίζεται ἡ δυνατότης κατοχῆς καὶ τὸ δικαίωμα τοῦ κράτους ἐπὶ τῆς προσωπικῆς περιουσίας καθ᾿ οἱονδήποτε τρόπον. Ἡ προσωπικὴ περιουσία εἶναι ἀναπαλλοτρίωτη, ἀφορολόγητη καθ᾿ οἱονδήποτε τρόπο, εκτός της δικαστικής οδού και όταν υπερβαίνει κατά πολύ τον μέσο όρο. Βασική αρχή κανείς δεν διαθέτει προσωπική περιουσία πάνω από το 3πλάσιο του μέσου όρου του τόπου. Αυτός ο όρος είναι ο πιο δίκαιος από όλους καθότι με βάση αυτό τον όρο μια κοινωνία προοδεύει όχι ατομικά και από μόνη της αλλά όλοι μαζί, ενώ με αυτόν τον βασικότατο όρο δεν μπορούν άνθρωποι δωσίλογοι, προδότες, άρρωστοι να αναρριχηθούν και να ελέγχουν τους άλλους. Ο περιουσιακός έλεγχος εξετάζεται από τους αιρετούς τοπικούς εισαγγελείς και οι αποφάσεις λαμβάνονται στο δικαστήριο του Δήμου.
\item Οἱοδήποτε χρεοκρατικὸ νόμισμα δὲν μπορεῖ νὰ μετατραπεῖ σὲ ἐνυπόθηκο καθ᾿ οἱονδήποτε τρόπο. Ὁτιδήποτε μὲ εὐρὼ καὶ δολάρια ἔχει ἀγοραστεῖ ἐπιστρέφει στὸ δημόσιο χωρὶς νὰ ἐπιστραφοῦν οἱ λογιστικὲς μονάδες.
\item Ἀπαγορεύεται ἡ χρηματοδότηση τῶν σχολικῶν βιβλίων ἀπὸ τὸ ἐξωτερικό.
\item Τὰ βιβλία τῶν μαθητῶν, σπουδαστῶν, φοιτητῶν εἶναι πολυτονικὰ καὶ ἀγοράζονται. Τώρα μποροῦν νὰ κάνουν χρεοκρατικὴ χρήση τῶν βιβλίων, ἔχουν τὴν δυνατότητα νὰ τὰ πωλοῦν ὅσα δὲν θέλουν.
\item Τὰ βιβλία τῶν σπουδαστῶν - φοιτητῶν προτείνεται νὰ εἶναι δίχρωμα.
\item Θεσμοθετεῖται ἡ ὁμάδα ἀποζημιώσεων ἡ ὁποία ἀποτελεῖται ἀπὸ ἕναν εἰσαγγελέα, ἐὰν οἰκονομολόγο, ἕναν ἱστορικὸ καὶ φισιοδίφη ἀπὸ κάθε Δῆμο, οἱ ὁποῖοι ἐκλέγονται κάθε δυὸ ἔτη, συντηροῦνται ἀπὸ τὸ Δῆμο, καὶ οἱ ὁποῖοι θὰ λάβουν τὸ ἕνα τοῖς χιλίοις τοῦ χρυσοῦ καὶ τῶν χρημάτων ποὺ θὰ ἀποφέρουν στὴν Ἑλληνικὴ Πολιτεία. Ἀπαραίτητη προϋπόθεση τῆς ὁμάδος εἶναι ἡ ἄριστη γνώση τοῦ Διεθνοῦς Δικαίου. Σκοπὸς ὁμάδος, οἱ ἐπιστροφὲς χρυσοῦ ἀπὸ τὴ Γερμανία, Τουρκία, Ἀγγλία κ.λ.π, ἀποτίμηση καταστροφῶν καὶ διεθνεῖς διεκδικήσεις. Τὸ σύνολο τοῦ ἀνθρώπων τῆς ὁμάδος εἶναι 3 ἐπὶ τοὺς νόμους τῆς χώρας σὺν 13 εἰδικοὺς στοὺς Η/Υ, διαδίκτυο κλπ
\item Ἡ ἑλληνικὴ πολιτεία δημιουργεῖ ἀξιοκρατικὸ νόμισμα μὲ βάση τὶς ἀποφάσεις τῆς ἀνεξάρτητης πλέον Δικαιοσύνης σχετικὰ μὲ τὶς ἀποζημιώσεις τῆς ὁμάδος τοῦ σώματος Ἀποζημιώσεων, νόμισμα τὸ ὁποῖο ἀναφέρει ρητὰ τὸ ἐμπράγματο δικαίωμα(χρυσός, γῆ κλπ..), το οποίο δεν διακινείται εντός της Επικρατείας αλλά γίνεται χρήση του για εξωτερικές συναλλαγές.
\item Καμμία διακρατικὴ συμφωνία ἡ ὁποία μετατρέπει χρεοκρατικὸ νόμισμα σὲ ἐμπράγματη ὑποχρέωση δὲν ἰσχύει καὶ ὅσοι ἔχουν κάνει συμφωνίες κατάσχεται ἄμεσα κινητὴ καὶ κινητὸς περιουσία καὶ νὰ ἀποπέμπονται ἐκ τῆς ἐπικράτειας ὡς προδότες.
\item Συστήνεται ὑπηρεσία μὲ αἱρετοὺς καὶ ἕναν ἀπὸ κάθε Δῆμο κατ᾿ ἔτος στελόμενο, οἱ ὁποῖοι ἔχουν πλῆρες δικαίωμα σὲ κάθε φάκελο λογαριασμὸ κ.λ.π. πολιτῶν καὶ πολιτικῶν προκειμένου νὰ προβαίνουν σὲ ὁποιοδήποτε ἔλεγχο ἀπατών κλπ. Ὅλοι αὐτοὶ εἶναι εἰσαγγελεῖς καὶ ἀστυνομικοί.
\item Οἱ Ἕλληνες ποὺ γιὰ ἱδεολογικοὺς λόγους ἑντάσσονται στὸ κίνημα δὲν φορολογοῦνται για 2 χρόνια.
\item Ὅσοι Ἕλληνες πᾶνε νὰ κατοικήσουν στὸ Καστελόριζο καὶ στὴ Ῥῷ δημιουργώντας οἰκισμὸ θὰ ἔχουν ὅτι θελήσουν καὶ ἔχουν χρεία ἀπὸ τὴν κυβέρνησιν.
\item Ὅσοι πῆραν ὁμόλογά του ἑλληνικοῦ δημοσίου στὴ δευτερογενῆ ἀγορὰ καὶ τὰ πληρήθησαν κανονικὰ, ἐπιστρέφουν τὴ διαφορὰ στὴν Ἑλλάδα ὅσοι δικοί μας τοὺς βοηθήσανε νὰ χάνουν κινητικὴ καὶ ἀκίνητη περιουσία. Ἡ διαφορὰ τῶν ὁμολόγων εἶναι στὸ 6,7\% κέρδους.
\item Ὅλοι οἱ ὑπουργοὶ καὶ ὑφυπουργοὶ οἰκονομικῶν ἐπὶ ΠΑΣΟΚ, Ν.Δ. χάνουν κινητὴ καὶ ἀκίνητο περιουσία καὶ ὅσοι ἔχουν μοιραστεῖ ἐξ αὐτῶν ὁμοίως.
\item Αναιρείται κάθε υπαγωγή στο υφιστάμενο άρθρο 99, και για κάθε περίπτωση ξεχωριστά περνά από ορκωτό δικαστήριο τουλάχιστον 30 ενόρκων από την εισαγγελία του Δήμου, με αναδρομική ισχύ.
\item Ὃ μασσῶνος Κωνσταντῖνος Καραμανλής, ὁ παλιὸς εἶχε ὑπογράψει μὲ τὴ σατανικὴ εβρΕΟΚ, Ε.Ε, νὰ μὴν καῖμε μέταλλο γιὰ εἴκοσι χρόνια πάνω ἀπὸ 1200 Κελσίου καθυστερώντας ὡς ὄργανο τῶν ξένων τὴν ἀνάπτυξη γιὰ τουλάχιστον εἴκοσι χρόνια, ἐνῶ παράλληλα νομιμοποίησε μὲ τοὺς ἄλλους προδότες ὁμόλογα ἄκυρα μὲ προσαύξηση μάλιστα 60\% τὰ ὁποία διεκδικοῦμε σὺν τὴν ἠθικὴ βλάβη σὺν διαφυγόντα κέρδη. Ὅσοι στήσαν αὐτὴ τὴν κομπίνα πρέπει νὰ πληρώσουν. Ἡ Ἑλλάδα μετατρέπεται πλέον στὸ κέντρο τῶν τεχνολογιῶν. Ἡ Ἀθήνα μετατρέπεται σὲ μία τεράστια πανεπιστημιούπολη ἀπ᾿ ὅπου θὰ ἔρχονται ἀπὸ τὰ πέρατα τοῦ κόσμου γιὰ νὰ φοιτοῦν ἐνῶ στὰ ἰδιωτικὰ πλέον πανεπιστήμια θὰ διδάσκουν οἱ καλύτεροι καθηγητὲς τοῦ κόσμου. Καμμία φορολογία στὰ ἰδιωτικὰ πανεπιστήμια καὶ σχολὲς ποὺ ἀνοίγονται ἀπὸ Ἕλληνες. Γίνεται ἄμεση σύνδεση παραγωγῆς μὲ τὰ δημόσια πανεπιστήμια, ἐνῶ ὑποχρεοῦνται σὲ ἀῤῥωγὴ καὶ τὰ ἰδιωτικὰ πρὸς ἐπιχειρήσεις.
\item Τὰ ἔργα Κορέ, γρανάζι, τὰ ἀνιχνευτικὰ νέα ραντάρ, οἱ ἀπίστευτες Ἑλληνικὲς ἐφευρέσεις ἀνήκουν μόνο στὴν Ἑλλάδα καὶ χρηματοδοτοῦνται ἄμεσα τοῦ Πέτρου Ζωγράφου ἡ πιὸ μεγάλη ἐφεύρεση ποὺ ἔγινε ποτὲ καὶ ἡ ὁποία λύνει τὸ ἐνεργειακὸ μία καὶ γιὰ πάντα ἀνήκουν μόνο στὴν Ἑλλάδα χρηματοδοτοῦνται ἄμεσα γιὰ τὴν ἀκρίβεια μὲ τὴν πρώτη κυβέρνησιν δημιουργοῦνται τρία ἐργοστάσια αὐτοκινήτων καὶ ἄμεσα ἡ Νέα ΔΕΗ μὲ τὴν ἐφεύρεση τοῦ Πέτρου Ζωγράφου, κατανεμημένη σὲ κάθε Δῆμο καὶ σὲ κάθε block τετραγώνων γιὰ ἀνεξαρτησία ἐνεργειακὴ ἀπὸ κάθε ξένη ἐπιβολή.
\item Τὸ βιβλίο εἰσερχομένων γιὰ τὶς μεταποιητικὲς καὶ ἐξερχόμενων γιὰ τὶς παραγωγικὲς ἐπιχειρήσεις, συμβολαιογραφικὰ δημιουργούμενο καὶ ἀπὸ τὸν οἰκεῖο εἰσαγγελέα καὶ διευθυντὴ τῆς ἀστυνομίας ἐπιτηρούμενο ἀντικαθιστὰ τὶς τοπικὲς ἐφορίες μὲ τὰ χαρτιά τους.
\item Δικαίωμα προσβάσεως στὰ βιβλία εἰσαγόμενων ἐξερχόμενων ἔχουν μόνο τὸ τμῆμα ἐσόδων τοῦ Δήμου. Ὁ Δῆμος κρατάει το 2/3 των φόρων γιὰ τὴν τοπικὴ κοινωνία καὶ ἀποδίδει το 1/3 στὴν κυβέρνησιν.
\item Ἡ φορολόγηση γίνεται ἐπὶ τοῦ ἀκαθαρίστου προϊόντος.
\item Οἱ πολυεθνικὲς εἶναι ὑποχρεωμένες νὰ δίνουν ράφι στοὺς παραγωγοὺς δωρεάν.
\item Ἐπιτρέπεται καὶ ἐπιβάλλεται ἡ οἰκοτεχνία.
\item Κάθε Ἕλλην δύναται νὰ συμμετέχει στὶς μεταφορὲς μὲ ἰδιωτικὰ ὀχήματα, ἀρκεῖ μία ἐπιβεβαίωση ὅτι τὸ ἐν λόγῳ ὄχημα ἔχει τὴ δυνατότητα ἀπὸ μηχανολόγο.
\item Καταργοῦνται οἱ ἰδιωτικὲς ἀσφάλειες. Ὁ λόγος δομήθηκε αὐτὴ ἡ προδοτικὴ νομοθεσία εἶναι γιὰ νὰ μαζεύονται τὰ χρήματα τοῦ κόσμου ἀπὸ λίγους καὶ στὴ συνέχεια μὲ διάφορες κομπίνες νὰ τὰ μαζεύουν, ὅπως παράδειγμα οἱ μεγάλες καταστροφὲς τῶν Ἑλληνικῶν καταστημάτων καταστημάτων προκειμένου νὰ μποῦν οἱ πολυεθνικὲς ἐξ οὐ καὶ ἡ προδοτικὴ φορολογικὴ νομοθεσία ὑπὲρ τῶν πολυεθνικῶν.
\item Κάθε Δῆμος εἶναι ὑποχρεωμένος νὰ ἔχει τμῆμα ἀσφαλίσεως αὐτοκινήτων, καθὼς καὶ συνεργεῖο μὲ αἱρετὸ πρόσωπικὸ κατ᾿ ἔτος.
\item Μεταφορά, ἀλλαγὴ κ.λπ. ἐνεργεῖται ἀπὸ τὸ τμῆμα ἀσφαλίσεως αὐτοκινήτων τοῦ Δήμου.
\item Ὅλοι οἱ τόποι τοῦ Δήμου ἀνήκουν σὲ αὐτὸν καὶ στοὺς κατοίκους. Καὶ καμμιὰ κυβέρνηση δὲν τοὺς ἐποπτεύει.
\item Ἀγρότες, κτηνοτρόφοι, βιοτέχνες, ἀποτελλοῦν τὴν καρδιὰ τῆς ὑπάρξεως τῆς Ἑλλάδος. Οἱ Δῆμοι πρέπει μὲ ἰδιαίτερο τρόπο νὰ τοὺς προσέχουν, γιατί ἀποτελοῦν τὴν παραγωγὴ βάσης τῆς χώρας.
\item Τιτλοποιοῦνται τὰ ἐκ τῆς Δικαιοσύνης πρόστιμα πρὸς τὴ ΤΡΟΪΚΑ ἔπ' ἐσχάτη προδοσία, ἐκβιασμό, ἀπάτη, τρομοκρατία, τοκογλυφία, διεθνῶν ἐγκλημάτων, διεθνοῦς ἀπάτης ὡς ἀξιοκρατικὸ μέγεθος εἰσαγόμενο στὰ διεθνὴ χρηματιστήρια στὴν ἀξία τοῦ 75\% τῶν ἀποφάσεων.
\item Τιτλοποιοῦνται τὰ χρέη ἀντίστοιχως, ἑταιρειῶν, ὀργανισμῶν κ.λπ. τότε ἔχουν κάνει ἔγκλημα ἐνάντια στοὺς Ἕλληνες καὶ τὸν Ἑλληνισμὸ καὶ εἰσάγονται στὰ χρηματιστήρια.
\item Ἀπαγορεύεται τὸ ἐπαγγελματικὸ ψάρεμα ἀπὸ ξένους, συλλογὴ γόνου ἐντὸς τῶν ἑλληνικῶν ὑδάτων.
\item Καταργοῦνται οἱ συντάξεις τῶν βουλευτῶν. Δὲν νοεῖται ἐπάγγελμα βουλευτής. Αὐτὸ εἶναι προσφορὰ στὸ ἔθνος.
\item Πρῶτο μέλημα τῆς πρώτης κυβερνήσεως τοῦ κινήματος καὶ κάθε κυβερνήσεως εἶναι ἡ ἐπιστροφὴ τῶν Ἑλλήνων ἐπιστημόνων καὶ ἡ βοήθειά τους σὲ παραγωγικὲς διαδικασίες μὲ ἐπανένταξη τῶν στὴν ἑλληνικὴ πολιτεία.
\item Εἰδικὴ ἔρευνα καὶ χρήση τῶν ἀποβλήτων, κατάργηση τῶν ΧΥΤΑ καὶ μετατροπῆς σὲ ΧΥΤΗ, δημιουργία ἐτειρειῶν ἀποβλήτων καὶ ἐπανάκαμψη σύμφωνα μὲ τὶς ἤδη γνωστὲς καὶ ἐφαρμόσιμες τεχνολογίες. Τὸ πρόβλημα τῶν ἀποβλήτων εἶναι πολὺ ἁπλό. Δὲν χρειαζόμαστε λίγα ἀπόβλητα ἄλλα πολλά, καὶ δὲν λύνουν τὸ πρόβλημα γιὰ τὶς μίζες. Τὸ μεγάλο πρόβλημα, ξεκινάει μέσα ἀπὸ τοὺς εὐρωπαίους. Σήμερα δὲν ὑπάρχει πρόβλημα ἐπιβίωσης δὲν ὑπάρχει κανένα μὰ κανένα πρόβλημα, ὅλα εἶναι τεχνητά.
\item Κατάργηση του ΦΠΑ και κακούργημα η αναφορά του και οποιαδήποτε προσπάθεια επιστροφής του.
\item Υποχρεωτικές συνεταιριστικές τράπεζες λαϊκής βάσεως σε κάθε Δήμο και χρήση εσωτερικά Δραχμής.
\item Εσωτερικό νόμισμα Δραχμή ίσης αξίας με το ευρώ και απαγόρευση χρήσεώς της από ξένες δυνάμεις χρηματιστήρια και αλλού. Απαγόρευσις Ελληνικού νομίσματος σε κατοχή τραπεζιτών.
\item Η Δραχμή τυπώνεται από την Ελληνική Κυβέρνηση ως χρεωκρατικό νόμισμα και απαγορεύεται η χρήση της εκτός χώρας και σε χρηματιστήρια.
\item Υποχρεωτική παραμονή κεφαλαίων εντός της χώρας.
\item Δὲν πρέπει νὰ ξεχαστοῦν τὰ 20 ἢ 25 χιλιάδες χρυσὲς λίρες ἢ τάληρα(θὰ τὸ δείξει ἡ ἔρευνα καὶ τὸ εἶδος καὶ τὴν ποσότητα) ποὺ ἔχασε λόγω τοῦ πολέμου ὁ συνεταιρισμὸς τῶν Ἀμπελακίων, νὰ ἀποτιμηθοῦν νὰ ἐκδικαστοῦν ἀπὸ τὴν κεντρικὴ κυβέρνηση καὶ νὰ ἀπαιτηθοῦν.
\item Ἐπιβάλλεται ἡ δημιουργία συνεταιρισμῶν ἐλαιοπαραγωγῶν καὶ ὁ 5πλασιαμός τῆς φυτικῆς καὶ ζωικῆς παραγωγῆς μὲ βίκο-τριφύλι καὶ κότες, κουνέλια, γαλοποῦλες στοὺς ἐλαιῶνες, ἐνῶ μὲ συστηματικὸ δολωματικὸ ψεκασμὸ παύει ὁ δάκος(χωρὶς φάρμακα).
\item Ἀπαγορεύεται ἡ προσωπικὴ περιουσία ἑνὸς ἀτόμου νὰ ἐκτινάσσεται 3 φορές επάνω τοῦ μέσου ὄρου. Ο όρος αὐτὸς εἶναι βασικὸς γιὰ τὴν εὐτυχισμένη διαβίωση ἑνὸς λαού.
\item Αὐτὴ αὕτη ἡ περιουσία νὰ κατανέμεται σὲ πολλοὺς μετὰ ἀπὸ δικαστήριο ὁρκωτὸ τὸ ὁποῖο πραγματοποιεῖ ὁ Δῆμος. Ἐδῶ στὸ χῶρο ποὺ κατοικῶ, ἔρχομαι σὲ ἐπαφὴ μὲ ἐργοστάσια ποὺ ἀνήκουν στὰ παιδιὰ τῶν δωσιλόγων. Αὐτοὶ οἱ δωσίλογοι πήρανε χρήματα ἀπὸ τοὺς γερμανοὺς στὴν κατοχὴ καὶ φτιάξανε περιουσίες. Ὅλα αὐτὰ ἀνήκουν πλέον στὸ σύνολο, καὶ ἂν μὲ εἶναι ἐργοστάσια σὲ συνεταιρισμοὺς ποὺ πρέπει νὰ δημιουργηθοῦν σπὸ τοὺς φτωχότερους ὥστε νὰ ἐργαστοῦν. Ὅλα ὅμως αὐτὰ περιέχονται ὡς περιουσία στοὺς Δήμους, καὶ ὅσοι παίρνουν μέρος τούτων τὸ παίρνουν πρόσκαιρα, ἀπὸ τοὺς πιὸ φτωχοὺς ὥστε νὰ δημιουργήσουν κάτι στὴ ζωή τους. Αὐτὸ εἶναι ἕνα δύσκολο κομμάτι ἀποφάσεων τὸ ὁποῖο ἐνεργεῖ ὁ Δῆμος καὶ οἱ ἀποφάσεις λαμβάνονται στὸ ὁρκωτὸ δικαστήριο, ἐνῶ μὲ δημοψήφισμα ἐπικυρώνεται τὸ ποιοὶ καὶ γιὰ πόσο χρονικὸ διάστημα θὰ ἀναλάβουν μέρη περιουσιακὰ καὶ γιὰ ποιὸ λόγο. Ὁ εἰσαγγελέας τοῦ Δήμου, οἱ αἱρετοὶ ἄρχοντες, ὁ Διευθυντὴς τῆς Ἀστυνομίας εἶναι οἱ ἐπιβλέποντες αὐτῶν τῶν διαδικασιῶν.
\item Μεγάλη προσωπικὴ περιουσία εἶναι αὐτὴ ποὺ εἶναι μεγαλύτερη ἀπὸ τὸ τριπλάσιό του μέσου ὄρου κατὰ κεφαλήν στον τόπο διαμονής.
\item Κάθε Έλλην πολίτης φορολογείται ξεχωριστά \textbf{εκτός από από τους παραγωγούς οι οποίοι δεν φορολογούνται}.
\item Οι ιδιώτες τραπεζίτες φορολογούνται στο 100\% της αξίας των ακινήτων τους συν τα πρόσθετα έσοδα.
\item Φόρος σε ακίνητα δεν υπάρχει, τέλη κυκλοφορίας, μεταβιβάσεις κλπ, κρατικές υπηρεσίες όπως η ΜΟΜΑ, δεν φορολογούνται. 
\item Απαγορεύονται τα τέλη κάθε είδους.
\item Καταργούνται πάσης φύσεως εταιρείες και όλοι φορολογούνται ως φυσικά πρόσωπα. Ειδικά ανώνυμες και κάθε είδους εταιρείες απαγορεύονται. Τώρα λειτουργούν συνεταιρισμοί μόνον.
\item \textbf{Τα παραγόμενα προϊόντα απαγορεύεται ρητά να διατίθενται από τους παραγωγούς με εξαίρεση τους συνεταιρισμούς.}
\item Οι πολυεθνικές απαγορεύεται να δημιουργούν θυγατρικές εντός της χώρας.
\item Οι πολυεθνικές απαγορεύεται να είναι παραγωγοί προϊόντων εντός της επικρατείας.
\item Οι πολυεθνικές είναι υποχρεωμένες να παρέχουν δημόσια τα πρόσωπα που βρίσκονται πίσω από αυτές, δηλ. τους μετόχους και ιδιοκτήτες.
\item \textbf{Οι τιμές των παραγομένων προϊόντων καθορίζονται αποκλειστικά από τους παραγωγούς εκτός δικαστικής αποφάσεως.}
\item \textbf{Ἀπαγορεύεται ἡ εἰσαγωγὴ προϊόντων ἀπὸ ἄλλο νομό εκτός δικαστικής αποφάσεως}. Ἐκτὸς καὶ ἂν δὲν ὑπάρχουν αὐτὰ τὰ προϊόντα. Ὁ σύν/σμὸς θΈΣ/ΓΑΛΑ καὶ ΘΕΣ/ΓΗ ἁπλώνεται σὲ ὅλη τὴν Ἑλλάδα. Ἀπαγορεύονται τὰ προϊόντα βασισμένα στὴν ἀσπαρτάμη. Ἀπαγορεύονται τὰ προϊόντα βασισμένα στὴν ὑδρογόνωση ἐλαίων. Ἀπαγορεύονται τὰ πλαστικὰ βούτυρα, ὑδρογονωμένα κλπ Ἀπαγορεύεται ἡ προσθήκη ζαχάρεως στὰ τρόφιμα. Ἀπαγορεύονται οἱ ὀδοντόκρεμες νὰ περιέχουν φθόριο. 
\item Τὰ φυσικὰ πρόσωπα ποὺ πήρανε θαλασσοδάνεια τὰ ἐπιστρέφουν μὲ κατάσχεση κινητῆς καὶ ἀκινήτου περιουσίας. Παρομοίως καὶ τὰ νομικά. 
\item Οἱ τιμὲς τῶν ἐνοικίων καθορίζονται ἀπὸ τὸ Δῆμο σὲ συνεργασία μὲ τὸν εἰσαγγελέα τοῦ Δήμου καὶ τοὺς δημογέροντες, ὥστε νὰ μὴν ὑπάρχει αἰσχροκέρδεια καὶ πλεονεξία.
\item Οἱ πολεμικὲς ἀποζημιώσεις εἶναι 1,5 τρὶς εὐρὼ καὶ ἐκδίδεται ἔνταλμα διεθνὲς κατὰ τῶν ἐγκληματιῶν τῆς Γερμανίας(Σόϊμπλε κλπ.).
\item Aπαγορεύεται τα ιδιωτικά ΚΤΕΟ. Όλες οι υποδομές μεταφέρονται στους Δήμους και οι υπάλληλοι είναι αιρετοί κάθε δύο έτη.
\item Κατάργηση μνημονίων αυτοδίκαια.
\item Καταργοῦνται ἐντὸς τῆς ἐπικράτειας τὰ λεγόμενα πνευματικὰ δικαιώματα όταν δεν ενεργούνται, εκτός δικαστικής αποφάσεως. Μὲ τὸ σύστημα αὐτὸ ἡ γέννηση μίας ἰδέας, τεχνικῆς, τεχνογνωσίας, ὁδηγεῖ σὲ παραγωγὴ τῆς χωρὶς καθυστέρηση. Μὲ αὐτὸ τὸν τρόπο εὐνοοῦνται ὅλοι καὶ εἰδικὰ ἐφευρέτες.
\item Καταργοῦνται οἱ ἑταιρεῖες πνευματικῶν δικαιωμάτων. Ἀνοίγονται δημόσια τὰ βιβλία των. Τιμωροῦνται ἰδιαίτερη οἱ πολιτικοὶ ποὺ ψήφισαν αὐτοὺς τοὺς νόμους μὲ ἄμεση κατάσχεση κινητῆς καὶ ἀκινήτου περιουσίας.
\item Δὲν ὑπάρχουν πνευματικὰ δικαιώματα. Ἀπὸ τὴ στιγμὴ ποὺ κάποιος δημιουργεῖ πνευματικά, αὐτὸ εἶναι κοινὸ μνημεῖο, σκαλοπάτι γιὰ ἀνάπτυξη δική του καὶ τῶν ἄλλων, σκαλοπάτι γιὰ γρήγορη ἀνάπτυξη τεχνῶν καὶ ἐπιστημῶν. Μπορούν να διατηρηθούν από τον δημιουργό μόνο μέσω δικαστικής αποφάσεως.
\item  Ἡ Ἑλληνικὴ Βιομηχανία κάνει ὑποχρεωτικὰ χρήση μόνο open source hardware (π.χ. μὲ βάση τὸ arduino), μὲ τὴν ὑποχρεωτικὴ καὶ ἄμεση βοήθεια τῶν Πανεπιστημίων ἐκτὸς Δικαστικῆς ἀποφάσεως.
\item Ἀπαγορεύεται ἡ κυκλοφορία μηχανῶν μὲ ἀπαγορευτικά δικαιώματα στὴν ἀντιγραφὴ ἀναταλλακτικῶν hardware.
\item Ὁ νέος φορολογικὸς κώδικας γιὰ τοὺς Δήμους \textbf{μόνο γιὰ τὰ μέλη τοῦ κινήματος} ἀπαγορεύεται νὰ εἶναι περισσότερο ἀπὸ 5 σελίδες Α5 μὲ 12ρια γράμματα στὴ γραμματοσειρὰ Ubuntu Normal. Ὁ νέος φορολογικὸς κώδικας ὑποχρεοῦται νὰ ἐφαρμόσει τὸ Σύνταγμα ποὺ λέει ὅτι ὅλοι ἔχουν τὸ δικαίωμα σὲ ἐργασία(οἱ Ἕλληνες ἐννοεῖται) καὶ συνεισφέρουν ἀνάλογα μὲ τὴ δύναμή τους. Τὸ νέο αἱρετὸ ὑπουργεῖο οἰκονομικῶν εἶναι αὐτὸ ποὺ καλεῖται σὲ συνεργασία μὲ τοὺς Δήμους καὶ τοὺς δημογέροντες νὰ δημιουργήσουν τὸ νέο φορολογικὸ κώδικα, ὁ ὁποῖος ἐπιβαρύνει τοὺς τεμπέληδες καὶ εἰσοδηματίες καὶ ἐλαφρύνει τοὺς ἐνεργοὺς πολίτες καὶ κυρίως αὐτοὺς ποὺ παράγουν. \\ Βασικὴ ἔννοια, ἡ ὁποία χρησιμοποιεῖται πλέον εἶναι ἡ ἔννοια τοῦ ἔργου. Ἄλλη ἡ ἐργασία τοῦ μισθωτοῦ, ἄλλη τοῦ ἐπιχειρηματία, τοῦ παραγωγοῦ κλπ. Ὁ νέος φορολογικὸς κώδικας μὲ τὴν ἰδιαιτερότητα τοῦ ἔργου καλεῖται νὰ δώσει δίκαιη λύση ὅπως τὸ Σύνταγμα ἀπαιτεῖ, τὸ "σύμφωνα μὲ τὶς δυνάμεις του". Ἰδιαίτερη προσοχὴ χρειάζεται γιὰ τὶς παραγωγικές, μεταποιητικές, ἐξαγωγικὲς ἑταιρεῖες καὶ τοὺς κινδύνους των, καθότι αὐταὶ αὖται πρέπει πρώτιστα νὰ ἐνεργηθοῦν, ἐνισχυθοῦν καὶ βέβαια αὐτὸ δὲν γίνεται μέχρι στιγμῆς μὲ τοὺς πράκτορες καὶ δωσιλόγους καὶ ὀργάνων τῶν ξένων πού μᾶς κυβερνοῦν. Ἀπὸ ἐμᾶς ἐξαρτᾶται ἡ σωτηρία μας καὶ ἡ ἐπιβίωσή μας. Αὐτοὶ προσπαθοῦν νὰ μετατρέψουν τὴ χώρα σὲ μπουρδέλο, ὥστε πάντα νὰ τοὺς κουβαλᾶμε στὴν πλάτη μας μέσα ἀπὸ ἕνα πρόστυχο ἐβραιομασσωνικὸ σύστημα ποὺ δημιουργήσανε καὶ κατσικωθήκανε στὴν πλάτη μας.

\item  \textbf{ΔΙΚΑΙΟΣΥΝΗ}

\item Καταργείται το ΣΤΕ και οι υπάλληλοί του περνούν από ορκωτό δικαστήριο από τους Δήμους με την κατηγορία συνεργείας σε εσχάτη προδοσία. Κατήγοροι είναι οι πολίτες. Όποιος πολίτης προσκομίσει τα απαραίτητα στοιχεία στον Δήμο ξεκινά η δίκη τους με ορκωτά δικαστήρια άνω των 33 πολιτών. Η ποινή είναι κατάσχεση κινητής και ακινήτου περιουσίας και η δημόσια εκτέλεση αν αποδειχθεί ότι  ενήργησαν με δόλο.
\item Οι εισαγγελείς και οι δικαστές περνούν από έλεγχο γνώσεων από επιτροπή ειδικών, Ελλήνων και ξένων, που θα καθορίζει η κυβέρνηση, κάθε τρία χρόνια και με δημόσιες εξετάσεις. Το ίδιο και οι πανεπιστημιακοί με 8ωρες εξετάσεις κάθε 3 χρόνια. Δεν υπάρχουν καθηγητές, επίκουροι κλπ. Όλοι είναι καθηγητές και επιβραβεύεται η αριστεία, οι θέσεις τους δε είναι ανάλογα με τη γραπτή βαθμολογία. Οι άριστοι των αρίστων αποτελούν την 7μελή επιτροπή των Πρυτάνεων. Οι φοιτητές απαγορεύεται καθ οιονδήποτε τρόπο να δημιουργούν φατρίες, τηρείες κλπ. Για την πολιτική εκτόνωση των φοιτητών υπάρχει σε κάθε πανεπιστήμιο ειδικός χώρος πολιτικής με ευθύνη των πρυτάνεων. Απαγορεύεται η αφισοκόλληση, καταστροφή, βαφή των πανεπιστημιακών χώρων. Επιβάλλεται η ύπαρξη περιοδικού της σχολής όπου οι φοιτητές μπορούν να γράφουν τις απόψεις τους δημόσια με ευθύνη των Πρυτάνεων. Σε κάθε Πανεπιστήμιο υπάρχει Σώμα Ασφαλεία του Πανεπιστημιακού χώρου με ευθύνη των Πρυτάνεων.
\item Ἡ ποινὴ τοῦ θανάτου ἐνεργεῖται μὲ κώνειο.
\item Ὅσοι λειτούργησαν ἐνάντια στὰ συμφέροντα τῆς Ἑλλάδος, χάνουν κινητὴ καὶ ἀκίνητο περιουσία, ἐνῶ ποτὲ δὲν δικαιοῦνται νὰ κατεβοῦν στὴν πολιτικὴ σκηνή.
\item Οι εισαγγελείς είναι ελεύθεροι επαγγελματίες με κανονικά γραφεία όπως οι πολιτικοί μηχανικοί, ψηφίζονται δε κάθε χρόνο. Κάθε Δήμος έχει τους δημοσίους αιρετούς δικούς του εισαγγελείς. Οι Δικαστές δεν κάνουν ελεύθερο επάγγελμα και δεν αποφασίζουν τις δίκες, παρά μόνον οι ορκωτοί πολίτες. Όλοι αποζημιώνονται ωριαία.
\item Οι λαθρέποικοι επιστρέφουν άμεσα και διά νόμου στους τόπους των, καταργείται η ιθαγένεια των μη Ελλήνων. Ιθαγένεια δίνεται μόνο σε 7ης γενιάς τέκνα αφού πρώτα είναι πλήρεις γνώστες της Ελληνικής παιδείας και ιστορίας, έχει δηλαδή καλλιεργηθεί σε αυτούς η Εθνική, Θρησκευτική και Πολιτική συνείδηση.
\item Δημιουργεῖται τὸ Ἀνώτατον Συνταγματικὸν Δικαστήριο. Οἱ δικαστὲς εἶναι αἱρετοὶ καὶ ἀλλάζουν κατ᾿ ἔτος. Εἶναι ὅσοι οἱ νομοὶ τῆς Χώρας καὶ Ἑλληνικῆς καταγωγῆς. Κάθε Δῆμος ἀποστέλλει καὶ ἕναν αἱρετὸ δικαστή. Κάθε Δικαστὴς τοῦ Ἀνωτάτου Συνταγματικοῦ Δικαστηρίου δύναται νὰ ἔχει μέχρι 12 ὑπαλλήλους βοηθοὺς καὶ οἱαδήποτε ὑποδομὴ ἐπιθυμεῖ. Κανένας δὲν ἔχει δικαίωμα νὰ ἐγερθεῖ ἐναντίον του κατὰ τὴ διάρκεια τῆς λειτουργίας τῶν καθηκόντων του. Ἀπαγορεύεται ρητῶς οἱαδήποτε συμμετοχὲς σὲ σέκτες. 
\item Η κυβέρνηση είναι οι επί 10 νομούς της χώρας εκλεγμένοι κάθε δύο χρόνια από τις τοπικές κοινωνίες, δεν υπάρχει πρωθυπουργός, μόνον ομάδες εργασίας των 7 ατόμων, τα θέματα των εργασιών των τα καθορίζουν στη βουλή και αποδίδουν το ειδικό νομοθετικό έργο τους βασισμένο σε αυτά που εδώ περιγράφονται. Το έργο τους υπόκειται στον έλεγχο της Συνταγματικότητος από αποκεκαθαρμένο Σύνταγμα(όπως η απατεωνιά του άρθρου 86) το οποίο προτείνεται και με Δημοψήφισμα επί των άρθρων του επικυρώνεται.
\item Καταργεῖται τὸ Ὑπουργεῖο Δικαιοσύνης. Καταργεῖται ἡ διαδικασία δικῶν ὅπως ἔχουμε γνωρίσει μέχρις σήμερον. Οἱ εἰσαγγελεὶς εἶναι αἱρετοὶ ὅπως καὶ οἱ δικαστές. Ἐπιλέγονται κάθε χρόνο, ὅπως καὶ οἱ τοῦ Δήμου. Δὲν ἀποφασίζουν οἱ εἰσαγγελεῖς γιὰ τὴν προώθηση τῶν φακέλων. Τὰ δικαστήρια εἶναι μόνο ὁρκωτά. Ἡ ἐργασία πλέον τῶν δικαστῶν καὶ τῶν εἰσαγγελέων εἶναι ἐλεύθερη σὲ ὁποιοδήποτε μέρος θέλει ὁ ἐνάγων. Οἱ ὁρκωτοὶ πολίτες καὶ οἱ εἰσαγγελεῖς ὅπως καὶ οἱ δικαστὲς θὰ ἀποζημιώνονται μὲ τὴν ὥρα, ὅπως καὶ οἱ δικηγόροι. Αὐτὸς ὁ ὁποῖος χάνει πληρώνει ἄμεσα τὰ ἔξοδα μὲ ἄμεση κατάσχεση κινητῆς καὶ ἀκινήτου περιουσίας. Τὰ δικαστήρια γίνονται μίας στάσεως, δηλαδὴ παράγεται ἄμεσο καὶ γρήγορο ἀποτέλεσμα. Σὲ κάθε ἐτήσια ψηφοφορία ἐπιλέγονται οἱ αἱρετοὶ εἰσαγγελεῖς, δικαστές, ἔνορκοι. Οἱ ἔνορκοι μποροῦν νὰ διακρίνονται σὲ τάξεις ἀνάλογα μὲ τὴ μόρφωση ἂν γιὰ παράδειγμα ἔχουμε εἰδικὲς δικές. Ὅλοι ἀποζημιώνονται ὡριαία. Σὲ κάθε δίκη εἶναι ἀπαραίτητο εἰς στενογράφος. Ἡ ἀπόφαση ἐκτελεῖται ἄμεσα ἀπὸ τὴν τοπικὴ ἀστυνομία. Σὲ περίπτωση ποὺ ἐνάγων ἢ ἐναγόμενος δὲν διαθέτει προσωπικὴ περιουσία γιὰ νὰ καλύψει τὰ ἔξοδα ὑποχρεοῦται σὲ ἐργασία ὥστε ὁ ἐργοδότης νὰ τὰ καλύψει. Ὁ ἐνάγων ἔχει τὴν δυνατότητα νὰ καλέσει ὅσους ἐνόρκους θέλει. Μὲ τὸ σύστημα αὐτὸ θὰ πάψουμε νὰ ἔχουμε ἀδιάφορους εἰσαγγελεῖς, λαδωμένους δικαστές, καθυστερήσεις, ἀδικίες κλπ. Οἱ ἔνορκοι, οἱ δικαστές, εἰσαγγελεῖς κ.λπ. ἐπιλέγονται τυχαῖα, ἐνῶ οἱ δίκες μποροῦν νὰ γίνουν σὲ ὁποιαδήποτε αἴθουσα. Ἐλαχιστοποιοῦνται ἔτσι ἀδικίες κλπ. Χωρὶς δικαιοσύνη δὲν ὑπάρχει δημοκρατία.
\item Καταργοῦνται οἱ φυλακὲς καὶ ἀντικαθίστανται μὲ κοινωνικὴ ἐργασία μὲ βραχιολάκι. Βασικὸς λόγος εἷναι ὅτι ἁποτελοῦν πηγὴ διαφθορᾶς. Ἡ κοινωνία χρειάζεται πολίτες καὶ ὄχι κακούργους. Μένουν μόνο φυλακὲς για ἑπικίνδυνους.
\item Οἱ φυλακισμένοι πληρώνουν μὲ ἐργασία τὴν ἐπιτήρησί τους καὶ τὸ φαγητό τους. Σὲ καμμιὰ περίπτωση δὲν ἐπιβαρύνεται ὁ Δῆμος.
\item Οἱ δάσκαλοι καὶ καθηγητὲς πληρώνονται ἀπὸ τὸ Δῆμο καὶ ψηφίζονται κατ᾿ ἔτος.
\item Ορκωτά δικαστήρια εντός 3 ημερών για ειδικές περιπτώσεις --πνευματικά δικαιώματα, εισαγωγές κλπ--.
\item \textbf{ΤΑ ΟΡΚΩΤΑ ΔΙΚΑΣΤΗΡΙΑ ΕΧΟΥΝ ΤΙΣ ΕΞΕΙΣ ΔΙΑΒΑΘΜΙΣΕΙΣ. ΤΑ 7ΜΕΛΗ, ΤΑ 13ΜΕΛΗ ΤΑ 33ΜΕΛΗ ΚΑΙ ΤΑ ΠΑΝΕΛΛΑΔΙΚΑ ΜΕ 10*ΤΟΥΣ ΔΗΜΟΥΣ ΤΗΣ ΧΩΡΑΣ ΚΑΙ ΤΑ ΕΙΔΙΚΑ ΓΙΑ ΤΟΥΣ ΠΟΛΙΤΙΚΟΥΣ ΚΑΙ ΔΙΚΑΣΤΙΚΟΥΣ ΜΕ 99 ΟΡΚΩΤΟΥΣ ΠΟΛΙΤΕΣ.} 

\begin{quotation}
\textbf{ΔΙΑΔΙΚΑΣΙΑ ΜΗΝΥΣΕΩΝ ΚΑΙ ΑΓΩΓΩΝ}\\
Σε κάθε Δήμο της Δημοκρατίας διατηρούνται σάκοι με τις ταυτότητες των φυσικών προσώπων κατά τάξη πτυχίων.\\
Κάθε Έλλην Πολίτης της Δημοκρατίας έχει δύο μεταλλικές ταυτότητες. Η πρώτη μπαίνει στο σκάκο που περιέχει τους εκλεγμένους ενόρκους πολίτες της Δημοκρατίας. \\
Η μεταλλική ταυτότητα περιέχει το εθνόσημο.\\
Η μεταλλική ταυτότητα περιέχει επί πλέον ηρρωϊκό πρόσωπο του Δήμου, πράγμα που σημαίνει ότι κάθε Δήμος έχει διαφορετικής μορφής ταυτότητα.\\
Σε κάθε ταυτότητα υπάρχουν τα στοιχεία του πολίτου της Δημοκρατίας.\\
Ανάλογα τη δίκη ο καταγγέλλων ή μυνύων πολίτης είναι αυτός που επιλέγει από το σάκο τυχαία τις ταυτότητες των ορκωτών πολιτών ενώπιον οργάνων του Δήμου.\\
Οι ταυτότητες παραδίδονται στους πολίτες και μετά το πέρας της δίκης επανατοποθετούνται.\\
Για ειδικού τύπου δίκες που χρειάζονται εξειδικευμένοι πολίτες η επιλογή γίνεται από τους σάκκους με τους εξειδικευμένους πολίτες.\\
\end{quotation}

\item Οἱ παιδεραστὲς ἐκτελοῦνται δημοσίως \textbf{ΜΕ ΑΠΑΓΧΟΝΙΣΜΟ}.
\item Οἱοδήποτε ξένο κράτος ποὺ δὲν συμμορφώνονται μὲ τὶς ἀποφάσεις τῶν Ἑλληνκῶν Δικαστηρίων κηρύσσεται παραχρήμα ἐχθρικὸ καὶ ὑποχρεοῦται σὲ ἐγκατάληψη τῆς πρεσβείας του ὅπως καὶ τῶν κινητῶν καὶ ἀκινήτων κατέχει.
\item Κάθε εἰσαγγελέας μπορεῖ νὰ ἔχει μέχρι 12 στελέχη τῆς ἐπιθυμίας του καθὼς καὶ κάθε δικαστής.
\item Ἡ ἐλευθερία τοῦ λόγου εἶναι δεδομένη καὶ ἐπιβάλλεται.
\item Βασικὰ καὶ ἀρχικὸ μέλημα τῆς πρώτης κυβερνήσεως ἡ σύστασις τοῦ πρώτου, ἀρχικοῦ καὶ μεγάλου δικαστηρίου χωρὶς δικαστές, μόνο ἐνόρκους, 100 ἀπὸ κάθε Δῆμο μὲ φυσικὴ παρουσία 10 ἀτόμων, μὲ 3 εἰσαγγελεῖς καὶ δημοσίους κατηγόρους μὲ θέμα τὴν ἐσχάτη προδοσία πολιτικῶν, ἀρχόντων, ΜΜΕ, δικαστῶν εἰσαγγελέων καὶ τραπεζιτῶν, μὲ κατὰ πλειοψηφία ἀπόφαση ἔνοχος ἢ ἀθῶος καὶ μὲ ποινὴ τὴν κατασχέσεως κινητῆς καὶ ἀκινήτου περιουσίας καὶ τὴν ἀποπομπὴ ἐκ τῆς ἐπικρατείας. Τὰ πρῶτα χρήματα παίρνουν μὲ προσαύξηση 300\% οἱ ὁμολογιοῦχοι.
\item Ἡ κατηγοροῦσα ἀρχὴ δὲν εἶναι οἱ εἰσαγγελεῖς ἀλλὰ οἱ πολίτες ὅταν φέρουν ὅλες τὶς ἀπαραίτητες ἐνδείξεις στοὺς Δήμους οἱ ὁποῖοι ὑποχρεοῦνται νὰ ἀποστείλουν ἄμεσα καὶ χωρὶς καμία ἀναβολὴ ἡ καθυστέρηση τὴ δικογραφία. Οἱ ἔνορκοι πολίτες παρακολουθοῦν τὶς διαδικασίες καὶ οἱ ὁποῖες θὰ κυκλοφοροῦν καὶ σὲ DVD, δικτυακὰ καὶ μέσω τῶν ΜΜΕ καὶ οἱ ὁποῖοι ἀποζημιώνονται ὠριαίως.
\item Μὲ αὐτὸν τὸν τρόπο θὰ γλυτώσουμε ἀπὸ τοὺς προδότες, τοὺς μανδαρίνους τῶν ὑπουργείων, τοὺς λαδωμένους εἰσαγγελεῖς καὶ δικαστές, τὰ ὄργανα τῶν ξένων κλπ Γιὰ νὰ ὑπάρξει δημοκρατία εἶναι βασικὸ νὰ λειτουργεῖ ἡ δικαιοσύνη πράγμα ποὺ σήμερα δὲν γίνεται.
\item Κανένα ἀπὸ τὰ ἀδικήματα δὲν ἔχει χρόνο παραγραφῆς.
\item Κάθε ὑπόθεσις δικαστικὴ ἐκδικάζεται ἐντὸς τριῶν μηνῶν.
\item Αὐτὴ αὕτη ἡ ἀντίληψις περὶ δικαίου τοῦ ἁπλοῦ ἀνθρώπου, δηλαδὴ τῶν ἐνόρκων, μόνο αὐτὴ ἀποδίδει τὸ δίκαιον. Μὲ αὐτὸ τὸ σύστημα ἐνόρκων ὑποχρεοῦται καὶ ἡ νομοθεσία καὶ οἱ διατάξεις νὰ εἶναι ἁπλὲς καὶ κατανοητές, καθότι ἔχει γραφεῖ ὅτι τὰ περίπλοκα πράγματα καὶ διατάξεις "ἐκ τοῦ πονηροῦ ἐστίν",δηλονότι, ὅτι ἔχει σκεφτεῖ μέχρι σήμερα εἶναι ἐβραιομασσωνικό, γιὰ τὴ βόλεψη μίας κλίκας καὶ τὴν ὑπαγωγή μας στοὺς σιωνιστὲς μὲ τὶς κωλοτράπεζές τους. Ἔχουμε δὲ χρέος γιὰ τὴν Πίστη μας, τὴν Πατρίδα μας, τὴν οἰκογένειά μας νὰ τοὺς ἐξαφανίσουμε, καθότι ἢ αὐτοὶ ἢ ἐμεῖς.
\item Ανοίγονται οι λογαριασμοί των ανθρώπων του ΣΔΟΕ. Ειδική επιτροπή ελέγχου από την Κυβέρνηση.
\item Εισαγγελείς και δικαστές που ενήργησαν ενάντια στους πολίτες αδίκως και υπέρ των αλητοτραπεζών, χάνουν κινητή και ακίνητο περιουσία και άδεια ασκήσεως επαγγέλματος. Το ορκωτό δικαστήριο που θα καθορίσει το εκβησόμενο είναι 33 ορκωτών πολιτών.
\item Κάθε δικαστήριο ορκωτών πολιτών έχει κατ ελάχιστον 7 ορκωτούς πολίτες.
\item \textbf{Οι νόμοι λειτουργούν γνωστικά, παιδαγωγικά και οδηγητικά απέναντι στα ορκωτά δικαστήρια και ΟΧΙ ΑΝΑΓΚΑΣΤΙΚΑ}. Αυτό σημαίνει ότι αυτή αύτη η συνείδηση των ορκωτών πολιτών \textbf{αποτελεί και είναι το ΔΙΚΑΙΟ}.
\item Οἱ αἱρετοὶ εἰσαγγελεῖς συνέρχονται μία φορᾶ τὸ χρόνο στὴ βουλὴ σὲ δημόσια διαβούλευση γιὰ τοὺς νόμους καὶ τὶς δράσεις των.
\item Ἕνωση ἢ κράτος τὸ ὁποῖο δὲν δέχεται διεθνῆ ἐντάλματα στοὺς ρότσιλτ, σορός, ροκφελερ καὶ λοιποὺς ἐγκληματίες καὶ τοκογλύφους καὶ κάθε παρόμοιο τύπο εἶναι ἀποῤῥιπτέο ἀπὸ τὸ Ἑλληνικὸς Ἔθνος καὶ κράτος, οἱ πρεσβεῖες των νὰ κλείνουν καὶ νὰ ἀναστέλλεται κάθε ἐπαφὴ μαζί τους. Ὁ λόγος εἶναι ἁπλός. Ἐὰν δὲν δέχονται αὐτὰ τὰ ἐντάλματα σημαίνει ὅτι εἶναι μαζί τους, ὅποτε δὲν μποροῦν νὰ ἔχουν καμία σχέση μὲ ἐμᾶς. Πάνω ἀπ' ὅλα εἶναι ἡ ἀλήθεια καὶ τὸ Δίκαιο.
\item Θάνατος στοὺς ὁμοφυλόφιλους δικαστές.

\item  \textbf{ΔΗΜΟΙ}

\item Απαγορεύονται οι κεραίες κινητής τηλεφωνίας εντός πόλεων και οικισμών.
\item Οι Δήμοι είναι υποχρεωμένοι να δημιουργούν εταιρείες οι θέσεις των οποίων θα καλύπτονται μόνο από ΑΜΕΑ. Αυτή η υποχρέωση θα αποδώσει στους Δήμους, στα παιδιά με ειδικές ανάγκες χρήματα, επιβεβαίωση, εργασία, εκτόνωση. Το έχει κάνει ήδη το Ισραήλ με μεγάλη επιτυχία για τα ειδικά αυτά άτομα, θα το κάνουμε και εμείς.
\item Δημοψήφισμα διενεργεῖται ὑποχρεωτικά, ὅταν τὸ 5\% τοῦ ἑλληνικοῦ λαοῦ κατ᾿ ἐλάχιστον ἀποφασίσει πρὸς τοῦτο.
\item Ὑποχρεωτικὴ ἡ κυκλικὴ μετανάστευση στὸ ὅριο τῶν 4 ἐτῶν. Μὲ διακρατικὴ συμφωνία μισθώνονται ἔποικοι, οἱ ὁποῖοι ὅμως τώρα δὲν ἀποκτοῦν δικαιώματα, καὶ παραμένουν γιὰ τέσσερα ἔτη στὴν ἐπικράτεια. Ἀποβάλλονται οἱ ἔποικοι οἱ ὁποῖοι μετώκισαν στὴ χώρα μας ἤ μας τοὺς φέραμε ὡς πέμπτη φάλαγγα. Ἡ συνθήκη Δουβλίνο ΙΙ εἶναι προδοτικὴ καὶ πλέον ἐνίσχυρη και όσοι την υπέγραψαν τιμωρούνται επί εσχάτη προδοσία και κατάσχεση κινητής και ακινήτου περιουσίας.
\item Κάθε Δῆμος εἶναι ὑποχρεωμένος νὰ συντηρεῖ, νὰ χρηματοδοτεῖ κινηματογραφικὸ ὅμιλο, μὲ ἔμφαση καὶ πρώτιστη ἀνάγκη στὰ ἔργα τῶν ἠῤῥώων τοῦ 21, τῶν ἀρχαίων ἠῤῥώων καὶ τῆς πραγματικῆς ἱστορικῆς ἀληθείας, ὡς τὰ γεγονότα, τὰ γραπτά, οἱ ἱστορικοὶ ἀποκαλύπτουν.
\item Τὰ ἔργα κάθε Δήμου προβάλλονται παγκόσμια μὲ ἀρχὴ τὴν ὁμογένεια.
\item Οι παρελάσεις μ εμβατήρια, εθνικό ύμνο, με τη συμετοχή των σωμάτων ασφαλείας είναι υποχρεωτική και γίνεται μετά από μνημόσυνο δέηση υπέρ των πεσόντων και των ηρώων του γένους. Τη σημαία κρατούν Έλληνες και άριστοι.
\item Στις παρελάσεις είναι υποχρεωτική η παρουσία του σώματος φουστανελάδων, των \textbf{ΑΤΑΚΤΩΝ} τους οποίους υποχρεωτικά ο Δήμος διατηρεί.
\item Οι \textbf{ΑΤΑΚΤΟΙ} αποτελεί \textbf{σώμα εκστρατευκτικό, παραδοσιακής τοπικής φορεσιάς, με πλήρη οπλισμό παλαιό και σημερινό }και έχει πολλαπλό σκοπό. Την ανύψωση του ηθικού των σημερινών ραγιάδων, \textbf{την επιτήρηση του Δικαίου και του Συντάγματος, εφ όσον το Σύνταγμα της χώρας έχει τα επαναστατικά στοιχεία των Συνταγμάτων του 2011, 1955 και 1975, στοιχεία τα οποία υιοθετεί και επιβλέπει. Ορκίζεται τιμή και δόξα στην Πατρίδα, να φυλάττει τους όρους και τη στάση των προηγουμένων ηρρώων, να ελέγχει κάθε κακό δημόσια ΧΩΡΙΣ ΝΑ ΕΠΙΤΡΕΠΕΤΑΙ ΚΑΝΕΙΣ ΝΑ ΠΡΟΒΕΙ ΣΕ ΜΗΝΥΣΕΙΣ ΚΛΠ ΠΡΟΣ ΑΥΤΟ ΤΟ ΣΩΜΑ ΟΤΑΝ ΕΚΦΡΑΖΕΤΑΙ ΕΝ ΣΥΝΟΛΩ ΚΑΙ ΟΧΙ ΜΕΜΟΝΩΜΕΝΑ.}
\item \textbf{Το σώμα των ΑΤΑΚΤΩΝ αποτελεί κατ ελάχιστον το 10\% των πολιτών κάθε Δήμου υποχρεωτικά.}
\item \textbf{ΔΗΜΙΟΥΡΓΕΙΤΑΙ ΔΗΜΟΒΟΥΛΙΟ ΠΟΛΙΤΩΝ ΣΕ ΚΑΘΕ ΠΟΛΗ ΣΤΟ ΟΠΟΙΟ ΜΙΛΑ ΜΕ ΤΗ ΣΕΙΡΑ ΟΠΟΙΟΣ ΘΕΛΕΙ ΑΝΑΛΥΟΝΤΑΣ ΚΑΙ ΑΝΑΠΤΥΣΣΟΝΤΑΣ ΤΟ ΛΟΓΟ, ΤΗ ΡΗΤΟΡΙΚΗ ΚΑΙ ΤΗΝ ΠΟΛΙΤΙΚΗ.}
\item Κάθε Δῆμος ἐπιβάλλεται νὰ καλλιεργεῖ τὴν κάνναβη γιὰ τὸ ὕφασμά της καὶ τὶς ἀντικαρκινικές της ἰδιότητες ποὺ βρίσκονται στὸ λάδι της. Ἀκόμη ἐπιβάλλεται ἡ καλλιέργεια τῆς σόγιας.
\item Κάθε Δῆμος ὑποχρεοῦται νὰ καταγράφει τὴν ἐμπειρία τῶν παλαιότερων σχετικὰ μὲ τὰ βότανα, τὶς θεραπεῖες, τὶς καλλιέργειες, τὴν παραγωγή, τὶς ἐξαγωγές, νὰ ἐνημερώνει τοὺς καλλιεργητές, νὰ παρέχει πλήρη ἐνημέρωση μέσω τῶν καναλιῶν καὶ τῶν ἐφημερίδων, ἀκόμη ἠλεκτρονικὰ μέσω internet στὴν ἰστοσελίδα τοῦ Δήμου, καὶ νὰ παραδίδει αὐτὰ τὰ πράγματα στοὺς ἑπομένους ὡς κόρην ὀφθαλμοῦ. Ἀκόμη δὲ νὰ μεταφέρει τὴν τεχνογνωσία καὶ ἐμπειρία τῶν πανεπιστημίων ὅσον ἀφορᾶ τὴ ζωικὴ καὶ φυτικὴ παραγωγὴ στοὺς κτηνοτρόφους καὶ καλλιεργητὲς καὶ ἀγρότες.
\item Κάθε Δῆμος ὑποχρεοῦται νὰ διατηρεῖ ὑπὸ τὴν ἐπίβλεψη τοῦ τοπικοῦ αἱρετοῦ εἰσαγγελέως τοῦ Δήμου, συνεταιρισμὸ μὴ κερδοσκοπικό, μὲ ἐτήσια ἀλλαγὴ τοῦ Δ.Σ., καταναλωτικοῦ γιὰ τοὺς ἄπορους, συνταξιούχους καὶ ΑΜΕΑ. Οἱ θέσεις τοῦ Δ.Σ. θὰ εἶναι σὲ συνταξιούχους οἱ ὁποῖοι θὰ ψηφίζονται ἀπὸ τὴ γενικὴ συνέλευση.
\item Καταργείται ο ΓΟΚ. Βασική αρχή είναι ότι η δημιουργία νέου κτιρίου ΔΕΝ ΚΡΥΒΕΙ ΤΟΝ ΗΛΙΟ. Τα νέα κτίρια επιτρέπονται στο μηνιαίο τοπικό δημοψήφισμα.
\item Κάθε μήνα γίνεται ο απολογισμός των αιρετών αρχόντων και η επικύρωση με ψήφο και όχι με ανάταση χειρός όλων των θεμάτων που έχουν εκτεθεί στην ιστοσελίδα του Δήμου όπως και στην εφημερίδα του Δήμου.
\item Απαγορεύεται ο Δήμος να λειτουργεί ως επιχείρηση νοικιάζοντας πεζοδρόμια, κλπ. Όλα αυτά αποφασίζονται στην μηνιαία συνέλευση του Δήμου και παραχωρούνται. Οι αιρετοί του Δήμου δεν αποζημιώνονται, όλα εντάσσονται στην προσφορά προς το σύνολο.
\item Ἀπαγορεύονται κτίρια ἄνω τῶν δυὸ ὀρόφων. Μόνο τὰ δημόσια κτίρια καὶ οἱ μονὲς καὶ τὰ εἰδικὰ κτίρια κατὰ παράδοση, ὅπως λ.χ. οἱ πύργοι τῆς Μάνης.
\item Σε κάθε οικιστικό τετράγωνο είναι υποχρεωτική η ὺπαρξη ενός αγάλματος το οποίο θα απεικονίζει έναν από τους αρχαίους σοφούς ή στρατηγό(Πλάτωνα, Αριστοτέλη, Μέγα Αλέξανδρο κλπ), ή έναν Βυζαντινό Αυτοκράτορα, ή κάποιον άγιο ή εφευρέτη ή ήρρωα. Καταργείται η δημόσια ανεικονική τέχνη και η μοντέρνα γλυπτική. Ο λόγος γι αυτό είναι ότι η δημόσια τέχνη επέχει ρόλο παιδεύσεως των πολιτών και γνώσεως των προγόνων τους, αυτό δηλαδή που οι \textbf{προδότες} καταργούν.
\item Τὸ ὕψος, τὸ μέγεθος, τὸ εἶδος τῶν δημόσιων κτηρίων ἐπιλέγεται μὲ δημοψήφισμα τοῦ Δήμου. Για κάθε δημόσιο ἔργο ἀπαιτεῖται δημοψήφισμα.
\item Τὰ δημόσια ἔργα διενεργοῦνται \textbf{ΜΟΝΟ ἀπὸ τὴν ΜΟΜΑ.}
\item Καταργεῖται ἡ Τουρκικὴ πρεσβεία στὴ Θράκη.
\item Ἀπαγορεύεται ἡ οἱασδήποτε μορφῆς ἠλεκτρονικῆς καταγραφῆς. TAXIS, κάρτα πολίτη, κλπ και ιδίας αντιλήψεως προϊόντα. Όσοι επιβάλλουν ή προβάλλουν τέτοια προϊόντα αυτό θεωρείται εσχάτη προδοσία.
\item Ἀπαγορεύονται τὰ αἰολικὰ πάρκα καὶ τὰ συγκροτήματα ἡλιακῶν πάνελς γιὰ συλλογὴ ἐνέργειας.
\item Ἐπιβάλλεται σὲ κάθε παραθαλάσσιο Δῆμο ὁ ἐμπλουτισμὸς θαλασσῶν μὲ ἁπλὲς καὶ γνωστὲς τεχνολογίες τὶς ὁποῖες γνωρίζουμε. Μέσω αὐτῶν τῶν μεθόδων δημιουργοῦμε τὴν πιὸ βαριὰ βιομηχανία τῆς χώρας ἡ ὁποία θὰ ἀποδώσει ἀπρόσμενο πλοῦτο καὶ εὐμάρεια, ἐξαγωγὲς καὶ τουρισμό. Κάθε Δῆμος ὑποχρεοῦται νὰ συντηρεῖ μία τέτοια μονάδα.
\item Κάθε Δῆμος ὑποχρεοῦται νὰ διατηρεῖ τὴ σχολὴ βυζαντινῆς μουσικῆς.
\item Κάθε Δῆμος ὑποχρεοῦται νὰ διατηρεῖ \textbf{ΠΕΡΙΟΥΣΙΟΛΟΓΙΟ} στο οποίο δηλώνονται τα περιουσιακά στοιχεία. Κάθε πολίτης της Δημοκρατίας δηλώνει τα τα παγκόσμια περιουσιακά του στοιχεία \textbf{εκεί που εχει γεννηθεί}. Μόνο τα εντός της επικρατείας ελέγχονται για το 3πλάσιο του μέσου όρου.
\item Κάθε Δῆμος ὑποχρεοῦται νὰ διατηρεῖ \textbf{ΕΒΡΑΙΟΛΟΓΙΟ} στο οποίο δηλώνονται υποχρεωτικά όσοι πολίτες είναι εβραϊκής ή ξένου γένους ή καταγωγής. 
Δεν είναι υποτιμητικό να είναι κανείς Εβραίος, αλλά έχει διαπιστωθεί από παράδοση ότι οι μεγαλύτεροι εχθροί του Ελληνισμού είναι οι Έλληνες προδότες και οι αυτών σφουγκοκωλάριοι και από τους αλλοεθνείς οι Εβραίοι οι οποίοι λειτουργούν όλοι μαζί ως ένα σώμα, το οποίο μολύνει κάθε έθνος με την προδοσία μέσω του χρήματος. Δηλαδή ελέγχει και κατευθύνει του πολιτικούς στο σκοπό τους. \\ Χαρακτηριστικό παράδειγμα προδοσίας είναι η από το 2012 δημόσια αποκάλυψη ότι ο Τσίπρας έχει σχέση με την μυστική υπηρεσία των Εβραίων τη mossad, ότι η εβραϊκή Sigular Logic καταδολίευσε τις εκλογές προκειμένου ο δικός τους άνθρωπος τσίπρας να πάρει την εξουσία και να διαλύσουν τον Ελληνισμό, δηλαδή η γνωστή τακτική των Εβραίων, οι δε Έλληνες εισαγγελείς έχει περίτρανα αποκαλυφθεί ότι είναι δικοί τους άνθρωποι και τους προστατεύουν, αν και υπάρχει η υπόνοια ότι σε όλο το σώμα του δημοσίου έχει διαποτιστεί από \textbf{Εβραίους με αλλαγμένα ονόματα}, πράγμα το οποίο αν αποκαλυφθεί ότι είναι έτσι έχουμε \textbf{συνομωσία σε βάρος μας από τους εβραίους} πράγμα τρομερό και απίστευτο.
\item Οἱ δάσκαλοι ἀπόφοιτοι τῆς βυζαντινῆς μουσικῆς ὑποχρεοῦνται νὰ παρέχουν τὶς ὑπηρεσίες τους ἔναντι ἀμοιβῆς στὰ σχολεῖα.
\item Καταργοῦνται οἱ δημόσιες συχνότητες, καὶ δὴ οἱ πανελλαδικές. Κάθε Δῆμος ὑποχρεοῦται νὰ συντηρεῖ τρία κανάλια τοπικὰ στὰ ὁποῖα θὰ προβάλλονται εἰδήσεις τοῦ τόπου, οἱ ἐνημερωτικὲς καὶ παιδαγωγικὲς ἐκπομπές, στὶς ὁποῖες θὰ συμμετέχουν οἱ τοπικοὶ φορεῖς, ὁ καθένας θὰ ἔχει τὴν ἐκπομπή του.
\item Κάθε Δῆμος ὑποχρεοῦται νὰ διατηρεῖ τράπεζα σπόρων.
\item Οἱ συμμορίτες μὲ τὶς κωλοεταιρεῖες τους, πούχουν σχεδιάσει τὸν CODEX ALLIMENTRIOUS, GATT, τὴ νέα τάξη, τὴν Ἀτζέντα 21, καὶ ὅ,τι ἄλλο προδοτικὸ γιὰ τὴν γῆ, νὰ τὸ βάλουν ἐκεῖ ποὺ ξέρουν. Οἱοσδήποτε ὑπερθεματίζει, ἐνεργεῖ καὶ προβάλλει γιὰ αὐτὰ τὰ πράγματα νὰ δημεύεται η περιουσία του, ὡς ὄργανο τῶν παγκοσμίων προδοτῶν.
\item Τὸ ὕψος, τὸ μέγεθος, τὸ εἶδος τῶν δημόσιων κτηρίων ἐπιλέγεται μὲ δημοψήφισμα τοῦ Δήμου. Κάθε δημόσιο ἔργο ἀπαιτεῖται δημοψήφισμα.
\item Τὸ πρῶτο μέλημα τῆς πρώτης κυβερνήσεως μὲ τὸ νέο ἄμεσο-δημοκρατικὸ σύστημα εἶναι τὸ τάμα τοῦ ἔθνους. \textbf{«Πατρίδα, πατρίδα, ἤσουν ἄτυχη ἀπὸ κυβερνῆτες νὰ σὲ κυβερνήσουν...»}! Αὐτὸ ἀναφώνησε κάποια στιγμή, ἀπογοητευμένος ἀπὸ τοὺς πολιτικοὺς τῆς ἐποχῆς του, ὁ ἥρωας στρατηγὸς Ἰωάννης Μακρυγιάννης. Στίς 31 Ἰουλίου 1829 ἡ Δ΄ Ἐθνική Συνέλευσις στό Ἄργος ἐξέδωσε τό Η΄ Εἰδικό Ψήφισμα γιά τήν ἐκπλήρωσι τοῦ Τάματος τοῦ Ἔθνους: «Ἡ Δ΄ Ἐθνική Συνέλευσις τῶν Ἑλλήνων νομίζει ἐαυτήν εὐτυχή, γενομένη τό ὄργανον δι’ οὖ τό Ἔθνος ἐκπληροῖ τό πλέον ἐφετόν τῶν χρεῶν του, τό νά ἀναπέμψη τήν εὐγνωμοσύνην του πρός τόν Θεόν, ὅστις ἔδειξε τοιαῦτα θαύματα διά τήν σωτηρίαν του… \textbf{Ὅταν ἡ τοπική περιφέρεια τῆς Ἑλλάδος καί ἡ καθέδρα τῆς Κυβερνήσεώς της κατασταθῶσιν ὁριστικῶς καί οἱ οἰκονομικοί πόροι τοῦ Κράτους ἐπιτρέψωσιν, θέλει ἀνεγερθῆ κατά διαταγήν τῆς Κυβερνήσεως εἰς τήν καθέδραν αὐτῆς, Ναός ἐπ’ ὀνόματι τοῦ Σωτῆρος τιμώμενος»}. Τό Ψήφισμα ὑπεγράφη ἀπό τόν Κυβερνήτη \textbf{Ἰωάννη Καποδίστρια} καί δημοσιεύθηκε \textbf{στό ὑπ’ ἀριθμ. 5 Φύλλο τῆς “Ἐφημερίδος τῆς Κυβερνήσεως” ὡς διάταγμα: «Περί ἀνεγέρσεως Ναοῦ τοῦ Σωτῆρος ἐν Ἀθήναις».Τό ἑπόμενο βῆμα ἔγινε ἐπί Βασιλέως Ὄθωνος κατόπιν ὑποδείξεως τοῦ ἰδίου τοῦ Θ. Κολοκοτρώνη. Ὁ Ὄθων ἐξέδωσε δύο Βασιλικά Διατάγματα: Τό πρῶτο στίς 25 Ἰανουαρίου 1834 καί τό δεύτερο στίς 3 Ἀπριλίου 1838 (ΦΕΚ ὑπ’ ἀριθμ. 12/11-4-1838) γιά τήν ἀνέγερσι τοῦ «Τάματος τοῦ Ἔθνους». Παράλληλα, τόν Αὔγουστο τοῦ 1843 ἡ Βασίλισσα Ἀμαλία ἀνέβηκε ἔφιππη στό ὕψωμα τῶν Νοτίων «Τουρκοβουνίων» συνοδευομένη ἀπό τόν Γενναῖο Κολοκοτρώνη (υἱό τοῦ Θεοδώρου). Ἐκεῖ ἀντίκρυσε τόν ὑψηλό βράχο τῆς κορυφῆς τόν ὁποῖον ἀπεκάλεσε «ἀστέρι τοῦ Βορρᾶ» καί ὅρισε αὐτόν τόν χῶρο ὡς τόν κατάλληλο γιά τήν ἀνέγερσι τοῦ Τάματος τοῦ Ἔθνους.} Δυστυχῶς ὅμως οἱ ὅρκοι, τά ψηφίσματα, τά διατάγματα κ.λπ. ἔμειναν γράμματα κενά. Τώρα η πραγματοποίηση του Τάματος του Έθνους είναι \textbf{υποχρεωτική και άμεση}.
\item Οἱ σχολὲς Βυζαντινῆς μουσικῆς τῶν δήμων νὰ ὑλοποιοῦν τὸν ἐθνικὸν ὕμνον.
\item Κάθε μήνα στὶς ἀρχὲς τοῦ μηνὸς τὴν πρώτη Κυριακή, στὶς 11 τὸ πρωί, μετὰ τὴ Θεία Λειτουργία ἐνεργεῖται ἡ Συνέλευσις τῶν Πολιτῶν στὴν κεντρικὴ πλατεία. Ἀπὸ πρὶν οἱ πολίτες πρέπει νὰ ἀποφασίζουν γιὰ θέματα τοῦ Δήμου, ὅπως αὐτὰ ἔχουν ἀπὸ πρὶν ἀνακοινωθεῖ ἀπὸ τὸ δικτυακὸ τόπο τοῦ Δήμου καὶ τὴν ἐφημερίδα τοῦ Δήμου. Τὰ τοπικὰ κανάλια εἶναι ὑποχρεωμένα νὰ καλύπτουν τὴ Συνέλευση τῶν Πολιτῶν.
\item Οἱ ἀποφάσεις λαμβάνονται μὲ ψηφοφορία. Σὲ περίπτωση μεγάλων δήμων οἱ ἐκλεγμένοι στὸ Δῆμο ἂς βρίσκουν λύσεις. Ἡ ἡμέρα τῆς Συνελεύσεως τῶν Πολιτῶν εἶναι πάντα Κυριακή.
\item Οι Έλληνες που σπουδάσανε στο εξωτερικό και εργάζονται σε μεγάλες εταιρείες είναι υποχρεωμένοι να επιστρέψουν στην Πατρίδα προκειμένου να δημιουργήσου και στελεχώσουν την βιομηχανία της χώρας.
\item Κάθε Δήμος υποχρεούται σε Η1-Β για την πρόοδο του τόπου.
\item Ὁ Δῆμος συντηρεῖ τὴν ἀστυνομία, ἡ ὁποία ἐπιλέγεται μὲ ψηφοφορία κάθε δυὸ ἔτη.
\item Κάθε δυὸ χρόνια γίνεται ἡ ἐπιλογὴ ἐκ τῆς τοπικῆς κοινωνίας, τῶν ὁρκωτῶν, τῶν εἰσαγγελέων, τῶν εἰσαγγελέων τοῦ Δήμου, τῶν εἰσαγγελέων τῶν εἰδικῶν ὑπηρεσιῶν, τῶν ἀστυνομικῶν, τῶν δασκάλων, τῶν καθηγητῶν, τῶν ὀργάνων τῶν δικαστηρίων, τῶν ὀργάνων τῶν ὑπουργείων καὶ τῆς πολιτείας, τῆς ἀγροφυλακῆς, ὀργάνων τῶν συνεταιρισμῶν, τῶν τραπεζῶν, τραπέζης σπόρων, καταπιστευμάτων, καθὼς καὶ τῶν εἰδικῶν ὀργανισμῶν, καὶ ὅτι ἄλλο θὰ χρειαστεῖ ὁ Δῆμος.
\item Κάθε Δῆμος συντηρεῖ ἀγροφυλακή, αἱρετὴ κατ᾿ ἔτος, πλήρως ὁπλισμένη καὶ μὲ κάθε δυνατότητα ἐλέγχου.
\item Τὸ συμβούλιο τῶν 7, αἱρετὸ κάθε δυὸ χρόνια, ἡ ὁποῖα εἶναι ἡ πλήρης καὶ ἀπόλυτη ἀρχὴ τοῦ Δήμου. Δὲν ὑφίσταται δήμαρχος ὅπως τὸ γνωρίζουμε σήμερα.
\item Ὁ Δῆμος ὑποχρεοῦται νὰ διαθέτει ἐξστρατευτικὸ σῶμα.
\item Μὲ μία πινακίδα αὐτοκινήτου μπορεῖ κανεὶς νὰ κινήσει ὅλα του τὰ ὀχήματα.
\item Ἐπιβάλλεται ἡ βοήθεια καὶ ἀῤῥωγὴ τοῦ Δήμου μὲ ἀρχιτεκτονικὲς ἐπιτροπές, ἐπιστήμονες, τοὺς ἐπιτίμους πολίτες τῆς πόλεως, μέρη τῶν πανεπιστημίων, ὁμάδες φοιτητῶν γιὰ τὴν ἀνάδειξη ἀρχαιολογικῶν χώρων, καταστημάτων, ξενοδοχείων γιὰ τὴ δημιουργία τουρισμοῦ, κυρίως προσκυνηματικοῦ, μελετητικοῦ, διακοπῶν, σχολικοῦ, ἱστορικοῦ, φυσιοδιφικοῦ, ὡς ἐπίσης καὶ νέες συνταγές, παραγωγές, ἐξαγωγὲς ὅπως συνεταιρισμὸς τῶν Ἀμπελακίων μὲ τὸν τότε Σβάρτς, ἀκόμη καὶ μὲ τὴ δημιουργία ἐταιριῶν ὑψηλῆς τεχνολογίας καὶ γνώσεων ἀπὸ τοὺς ἐπιτίμους πολίτες τοῦ Δήμου.
\item Ὡς ἐπίτιμοι πολίτες ἑνὸς Δήμου θεωροῦνται οἱ ἄνθρωποι ποὺ ἐνεργὰ βοήθησαν, βοηθοῦν καὶ προάγουν ἕνα τόπο μὲ τὴ γνώμη τους, τὴν ἀῤῥωγή τους, τὶς ἐνέργειές τους, τὶς γνώσεις τους, καὶ μὲ ὁποιοδήποτε ἄλλο τρόπο. Οἱ ἐπίτιμοι πολίτες μπορεῖ νὰ εἶναι καὶ ἄλλης ἐθνικότητος.
\item Κάθε Δῆμος εἶναι ὑποχρεωμένος νὰ προσφέρει κατάλυμα στοὺς ἐπιτίμους πολίτες, ἕνα ξενοδοχεῖο ὑποδοχῆς καὶ διαμονῆς γιὰ ἕνα χρονικὸ διάστημα, π.χ. 1 μήνα κατὰ ἔτος. Στὸ ἐν λόγῳ ξενοδοχεῖο θὰ ὑπάρχει ἐπιτροπὴ προσφορᾶς, διοργανώσεως συνεδρίων, καταγραφῆς ἰδεῶν καὶ τεχνολογιῶν, ὅπως ἐπίσης καὶ ὑποδομὲς ἐργαστηριακὲς γιὰ χρήση τους ἀπὸ τοὺς ἐν λόγω πολίτες μὲ συμμετοχὴ ὁμάδων ἐπιστημόνων καὶ φοιτητῶν. Ἡ πρόοδος ποὺ ἔχει νὰ ἐπέλθει ἀπὸ αὐτὴ τὴν ἐνέργεια θὰ δώσει τεράστια προκοπὴ στοὺς Δήμους.
\item Ἐνισχύεται μὲ ἐπιστημονικὲς ἐπιτροπὲς ὁ ἀγροτοτουρισμὸς καὶ οἰκοτουρισμός. Χρέος τοῦ συμβουλίου τῶν 7 νὰ ἔχει τέτοιες ἐπιτροπὲς καὶ δράσεις.
\item Ὑπέδαφος, ἔδαφος, ἀέρας, νερά, ἐνέργεια κλπ, ὅτι ἐντὸς τῆς ἐπικρατείας εὑρίσκεται, δὲν παραχωροῦνται, δὲν ἐνοικιάζονται καὶ ὅποια ἐπένδυση ἔγινε με "παραχώρηση" καὶ ὄλες οἱ λοιπὲς συμφωνίες καταργοῦνται καὶ ὅλα ἐπιστρέφουν στοὺς Δήμους, στὴν κυριὸτητα τῶν Δήμων καὶ κοινοτήτων πρὸς εκμετάλευσιν αὐτοδίκαια, ἄμεσα, χωρὶς δικαστικὲς ἀποφάσεις καὶ λοιπὲς διαδικασίες. Γιὰ δὲ τὶς στρατηγικὲς παραχωρήσεις τῶν κυβερνήσεων ἔχει νὰ συμφωνεῖ ὁ Δῆμος κατ᾿ ἔτος κατὰ τὸ συμφέρον τῆς χώρας.
\item Ἐλέγχονται οἱ ἀμερικανικὲς βάσεις γιὰ πυρηνικά. Σὲ περίπτωση ποὺ βρεθοῦν ἐνδείξεις πυρηνικῶν ὅπως ἰσσότοπα, ραδιενέργεια κλείνει ἡ βάση. Καμμία ἀμερικανικὴ βάση δὲν ἐπιτρέπεται νὰ ἔχει στραμμένα ὄπλα πρὸς τὴ Ρωσία. Ξέρουμε καλὰ ὅτι οἱ Ἀμερικανοὶ ἐλέγχονται πλήρως ἀπὸ τὸν σιωνισμό. Δίδονται άμεσα δύο βάσεις στη Ρωσσία για την επέκτασή της προς τη Μεσόγειο, εφ όσον εγγυούνται τα σύνορά μας, την υποχρεωτική τήρηση του Συντάγματός μας, τη βοήθειά τους με κάθε μέσον προκειμένου το αμεσοΔημοκρατικό Σύνταγμά μας να εφαρμόζεται. Παρόμοια οι ΗΠΑ για να έχουν τη βάση τους δεν πρέπει να έχουν πυρηνικά, εγγυούνται απολύτως τα σύνορά μας και την υποχρεωτική τήρηση του Συντάγματος ημών των Ελλήνων, τη βοήθειά τους με κάθε μέσον προκειμένου το αμεσοΔημοκρατικό Σύνταγμά μας να εφαρμόζεται και χωρίς καμμία εβραιομασωνική ή σιωνιστική ανάμειξη.
\item Ἑνώνεται ἡ Κύπρος μὲ τὴν Ἑλλάδα. Ὁ ἀττίλας ἐπιστρέφει σπίτι του μὲ ἀπειλὴ πολέμου. Οἱ ἰδιοκτησίες τῶν Ἑλλήνων ποὺ κατελήφθησαν ἐπιστρέφονται ἀσχέτως συμβολαίων.
\item Ἑνώνεται ἡ Βόρεια Ἤπειρος μὲ τὴν Ἑλλάδα. Ὁποιαδήποτε ἐνάντια ἀναφορὰ ἢ ἐνέργεια ξένου κράτους τὸ καθιστὰ ἐχθρικὸ καὶ ἀπομακρύνονται - κλείνονται ἄμεσα τὰ προξενεῖα του. Καμμία σχέση μὲ τοὺς ἐχθρούς μας. Αυτό ισχύει και για τη Ρωσσία και για τις ΗΠΑ.
\item Θεσμοθετεῖται ἄμεσα ἡ ΕΟΖ καὶ μὲ τὸ Καστελόριζο μέσα.
\item Ἀπαγορεύεται κανάλια και εκπομπές, ΜΜΕ κλπ, νὰ ἐλέγχονται ἀπὸ ἐβραίους, τούρκους, μωαμεθανοὺς, παρὰ μόνον ἀπὸ Ἕλληνες.
\item Κάθε Δῆμος εἶναι ὑποχρεωμένος νὰ ἀναδεικνύει τοὺς ἤῤῥωες, τὸν πολιτισμό, τὴν ἱστορία τοῦ τόπου του. Αὐτὸ σημαίνει τὴ δημιουργία ταινιῶν ἱστορικοῦ χαρακτήρα γιὰ παγκόσμια προβολή, ἕνα δηλαδὴ τεράστιο ἐργαστήρι διεθνές του πολιτισμοῦ, τῆς ἱστορίας, τῆς φιλοσοφίας κλπ. Μὲ τὰ σύγρονα μέσα εἶναι πλέον εὔκολο, ἀρκεῖ νὰ προσεγγιστοῦν οἱ ξένοι εἰδικοὶ ἐπιστήμονες προκειμένου νὰ δώσουν τὶς τεχνολογικὲς λύσεις γι΄ αὐτὸ τὸ σκοπό. Εἶναι ἡ εὐκαιρία μας ὡς Ἕλληνες νὰ δείξουμε παγκόσμια ἀλλὰ καὶ στὸ ἐβραϊκὸ χόλυγουντ ὅτι πάντα τὰ καταφέρνουμε καλύτερα. Ἀρκεῖ νὰ καταφέρουμε νὰ ἀποβάλλουμε τοὺς προδότες καὶ δωσιλόγους ποὺ πονηρὰ μᾶς ρίξανε νὰ μᾶς κυβερνᾶνε.
\item Δὲν ὑπάρχουν γκρίζες ζῶνες καὶ ἄλλες τέτοιες ἀνοησίες τῶν δωσιλόγων.
\item Ἀπαγορεύεται σὲ ξένους ἡ δωρεὰν παραμονὴ σὲ δημοσίους χώρους. Θὰ πληρώνουν ὅπως στὰ ξενοδοχεῖα.
\item Ἀπαγορεύεται τηλεφωνικὲς ἑταιρεῖες νὰ ἔχουν μετόχους ξένους.
\item Οἱ ἔχοντες ἐκφράσει ἄποψη δημοσίως ὅτι δὲν εἴμαστε Ἕλληνες ὅτι ἔχει χαθεῖ ἡ γενιά μας καὶ λοιπὲς παρόμοιες ἀνοησίες, δὲν ἔχουν καμία θέση σὲ πανεπιστήμια, γυμνάσια, λύκεια, ΜΜΕ καὶ Δημόσιο.
\item Ἀπὸ τὴν κυβέρνησιν δημιουργεῖται μυστικὴ ὑπηρεσία μὲ στόχο τὴν ἔρευνα καὶ ἀναφορὰν τῶν ἐχθρῶν στὴν κυβέρνησιν ἡ ὁποία ἀναφέρει τὰ ἀποτελέσματα τῆς ἐρεύνης στὴν ὁμάδα τῶν εἰσαγγελέων γιὰ περαιτέρω ἔρευνες καὶ δικαστήρια. Κατασκοπεία σὲ βάρος τοῦ ἔθνους ἔχει ποινὴ τὴν ἐκτέλεση.
\item Ὅσοι ἐντάσσονται στὸ κίνημα, δέχονται τὰ θέσφατα, ἄσχετα νὰ πιστεύουν ἢ ὄχι.
\item Ἀπὸ τὴ στιγμὴ ποὺ τὸ κίνημα κυβερνήση, τότε ὅλα τὰ θέσφατα νὰ ἐπέχουν θέση νόμου τὴν ὁποία δὲν χρειάζεται ἐπικύρωση ἐκ τῆς βουλῆς.
\item Ὅσοι δὲν συμφωνοῦν μὲ κάποιο ἀπὸ τὰ θέσφατα, παρακαλοῦνται νὰ μὴ άσχολοῦνται μὲ τὸ κίνημα.
\item Ὁ Δῆμος καθορίζει τὶς τάξεις τῶν ἐργαζομένων γιὰ τὴν ἀποζημίωση κάθε τάξεως. Παράδειγμα κυκλικοὶ μετανάστες, ἀνειδίκευτοι ἐργάτες, εἰδικευμένοι ἐργάτες, βαρέως τύπου ἐργάτες, καθηγητές, πανεπιστημιακοί, ἐρευνητὲς κλπ.
\item Ἀπαγορεύεται ἡ χρήση μισθοφόρων στον Ελληνικό στρατό.
\item Δὲν ὑπάρχουν διόδια. Καταργοῦνται. Δρόμοι ὑποδομὲς καὶ ὅτι ἔχουν δημιουργήσει οἱ ἐργολάβοι, αὐτοδίκαια, χωρὶς ἀνταπόδοση, ἔρχονται στὴν κατοχὴ τῶν Δήμων καὶ τῆς κυβερνήσεως. Ἀπαγορεύονται τὰ διόδια, γιατί δὲν μπορεῖ τέτοιες ἀντιπαραγωγικὲς ἀνοησίες νὰ δημιουργοῦν χρῆμα.
\item Ἀπαγορεύονται οἱ μεταφορικὲς ἑταιρεῖες. Κάθε ὁδηγὸς πρέπει νὰ ἔχει τὸ ὄχημά του, ἐνῶ ἐπιτρέπονται οἱ συνασπισμοί.
\item Ἀπαγορεύονται οἱ κλειστὲς παραλίες.
\item Δημιουργεῖται κεντρικὸ ἐποπτικὸ ὄργανο ἀπὸ τὴν κυβέρνηση ἐλέγχου συνεταιρισμῶν.
\item Στὸν ἐμπλουτισμὸ θαλασσῶν εἶναι μέτοχοι ὅλοι κάτοικοι τοῦ Δήμου.
\item Σὲ κάθε Δῆμο δημιουργεῖται συνεταιρισμὸς βοτάνων.
\item Ἀντικαθίστανται οἱ σακοῦλες ὀνίων μὲ δικτάκια ὥστε νὰ μὴ μολύνουμε τὸ περιβάλλον.
\item Ἀπαγορεύονται οἱ μεταλλαγμένοι σπόροι.
\item Κάθε Δῆμος χρησιμοποιεῖ εἰδικὴ μηχανὴ καὶ σύστημα συλλογῆς κλαδιῶν, ὀργανικῶν ὑλικῶν τὰ ὁποία ἐναποθέτει, ξεραίνει φυσικά, πολτοποιεῖ, παράγοντας ἐνέργεια πρὸς ὤφελος τῶν πολιτῶν.
\item Ἀπαγορεύονται οἱ χρωστικὲς χημικὲς οὐσίες στὰ τρόφιμα, σαπούνια, καθαριστικά, ὥστε νὰ μὴν ἐπιμολύνονται τὰ ὑπόγεια ὕδατα.
\item Οἱ λεκάνες τῶν τουαλετῶν συνδέονται σὲ εἰδικὸ δίκτυο τὸ ὁποῖο καταλήγει σὲ σύστημα δημιουργίας πολὺ ὑψηλῆς ἀξίας λιπάσματος. Ἀπαγορεύεται ὁ χημικὸς καθαρισμὸς λεκανῶν καὶ γίνεται χρήση νέων μεθόδων.
\item Δημιουργεῖται ἀπὸ τὴν κυβέρνησιν εἰδικὸ σῶμα ἐξορύξεως πέτρας, κοπῆς σὲ din, ὡς φθηνὴ πρώτη ὕλη γιὰ δόμηση ἢ ἐπαναδόμηση οἰκιῶν, πράγμα ποὺ θὰ λύσει πλεῖστα ὅσα προβλήματα.
\item Τὰ προϊόντα μπαζῶν δημιουργοῦν στοχευουμένες οἰκίες ψαριῶν, ἔξυπνα, φθηνὰ καὶ σωστὰ ὥστε νὰ δημιουργηθοῦν ταχύτατα οἰκοσυστήματα καὶ νὰ ἐπιτευχθεῖ εὔκολα ὁ ἐμπλουτισμὸς θαλασσῶν.
\item Δημιουργοῦνται ἄμεσα ἀπὸ τὴν κυβέρνησιν τρία ἐργοστάσια ὑάλου βαρέως τύπου, ὥστε νὰ προχωρήσουμε ἄμεσα στὴν ἀντικατάσταση τοῦ ἀλουμινίου, σιδήρου καὶ πλαστικοῦ γιὰ τὴ δημιουργία κουφωμάτων κλπ
\item Οἱ πολίτες κάθε Δήμου συστήνουν συνεταιρισμοὺς μὲ τὴ βοήθεια τοῦ Δήμου καὶ τῆς κυβερνήσεως καὶ τὴν ἀῤῥωγὴ τῶν πανεπιστημίων ὥστε γιὰ ὤφελός τους νὰ διασώσουν τὰ γυμνὰ βουνά.
\item Ἀπαγορεύονται σχέσεις ἐμπορικὲς καὶ συναλλαγὲς μὲ ἰδιωτικὲς τράπεζες ἢ τραπεζίτες απὸ τὸ Δημόσιο.
\item Στὸ δημόσιο ἐπιτρέπονται λειτουργικὰ μόνο open source.
\item Ἐπειδὴ μὲ αὐτὸ τὸ ἀμεσοδημοκρατικὸ σύστημα οἱ Δῆμοι θὰ ἀποκτήσουν τεράστια δύναμη καὶ οἰκονομικὴ ἰσχύ, θὰ εἶναι ὅμως πάντα ἐλεγχόμενοι ἀπὸ τὴν κεντρικὴ κυβέρνηση. Ἰδέες ἀποσχίσεων τιμωροῦνται μὲ ἐκτελέσεις. Ἡ Ἑλλάδα μόνο αὐξάνει, ποτὲ δὲν ἑλλατοῦται καὶ ποτὲ δὲν παραιτεῖται τῶν δικαιωμάτων της(Β. Ἤπειρος, Κύπρος, ἀποζημιώσεις, κλπ
\item Ἡ Ἀθήνα μετατρέπεται σὲ ἕνα τεράστιο ἐργαστήρι ἐπιστημῶν, τεχνολογιῶν, πανεπιστημίων, ἐνῶ οἱ νέοι ἐπιστρέφουν στὰ χωριά τους. Ὁ νέος σχεδιασμὸς εἶναι χωριὸ - μικρὴ ἢ μεγάλη παραγωγικὴ μονάδα, ἰδιωτικὲς ἢ συνεταιρισμοί, κεφαλοχώρια μὲ ἀποθηκευτικοὺς τύπους, πόλεις στὶς ὁποῖες εἰσῤῥέουν τὰ ἀγαθὰ καὶ γράμματα. Στὰ κεφαλοχώρια εἶναι ἐπιβεβλημένο νὰ λειτουργοῦν servers ὥστε ὅλα τὰ χωριὰ νὰ ἔχουν πρόσβαση σὲ ἐνημέρωση κ.λπ. Ὁ νέος ἱστὸς τῆς χώρας θὰ ἀποδώσει καρποὺς ταχύτατα καὶ πρόοδο ἀπίστευτη.
\item Οἱ πόλεις, τὰ πανεπιστήμια, ἡ κυβέρνηση, οἱ Δῆμοι κλπ ἐνισχύουν τεχνολογικὰ, γνωστικὰ, με υποδομὲς τὰ χωριὰ τὰ ὁποία μποροῦν νὰ παρέξουν τεράστια ποσὰ ἁγαθῶν καλύπτοντας τὶς διατροφικὲς ἀνάγκες τῆς χώρας.
\item Ἡ Ἑλληνίδα μάνα μὲ πάνω ἀπὸ τρία παιδιὰ ἐνισχύεται μὲ σιτηρέσιον ἀπὸ τὸ Δῆμο καὶ παίρνει ἐλάχιστη σύνταξη.
\item Δημιουργεῖται ἀπὸ τὴν κυβέρνηση ἡ εἰδικὴ ἐπιτροπὴ ἐλέγχου καυσίμων τῶν Δήμων σὲ σχέση μὲ τὰ σκουπίδια. Ἔχουν παρατηρηθεῖ ἀποκλίσεις τῆς τάξεως τοῦ 70\%, πράγμα ποὺ σημαίνει ὅτι ὑπάρχει δόλος(μίζες). Οἱ ὑπεύθυνοι πληρώνουν ἀπὸ τὴν τσέπη τους.
\item Οἱ ἰδιῶτες μποροῦν νὰ μεταφέρουν προϊόντα ἢ ἀνθρώπους.
\item Απαγορεύονται προϊόντα χωρίς πλήρη εγχειρίδιο στα Ελληνικά.
\item Κατάργηση του Καλλικράτη.
\item Ἀπαγορεύονται οἱ διαφημίσεις καὶ ἐπιτρέπονται ἐνημερώσεις.
\item Ὁ Δῆμος ἐκδίδει ἐφημερίδα στην ὁποία συμμετέχουν ὅλοι οἱ σύλλογοι, σωματεῖα, συνεταιρισμοί. Στὴν ἰστοσελίδα τοῦ Δήμου κάθε κατάστημα δικαιοῦται σελίδος μὲ ἐλάχιστη ἀμοιβὴ γιὰ ἐνημέρωση τῶν πολιτῶν.
\item Μὲ αὐτὸ τὸ σύστημα παύουμε νὰ καταστρέφουμε τὸ περιβάλλον, νὰ φορτώνουμε ἐξόδα στὶς μικρὲς ἐπιχειρήσεις, ἐνῶ οἱ μεγάλες γίνονται μεγαλύτερες μὲ ψεύτικα χρήματα, δῶρα τῶν κλεφτῶν καὶ ἀπατεώνων πολιτικῶν. Στὴν ἐφημερίδα τοῦ Δήμου κατατίθενται μελέτες καὶ ἰδέες ὅσων θέλουν νὰ χρηματοδοτηθοῦν, ἐνῶ ταυτόχρονα παίρνουν καὶ βέβαια ἡμερομηνία.
\item Ὅ,τι παθαίνουμε ὡς ἔθνος, ὅτι θὰ πάθουμε εἶναι ἀπὸ τὰ μυαλά μας, καὶ βεβαίως βεβαίως ἀπ᾿ τὰ γεννήματα τῆς προδοσίας καὶ διαστροφῆς τοὺς πολιτικάντηδες τὰ γεννήματα τῆς ἐβραιουργιᾶς -κατὰ τὸ Μακρυγιάννη-, τὸ χριστοκτόνο λαό, ὄχι ὅμως ἀπὸ ὅλο τὸ λαό, ἀλλὰ τοὺς σιωνιστὲς ποὺ ἔχουνε πάρει χωρὶς νὰ δουλέψουμε τὸ 99\% τοῦ παγκόσμιου πλούτου, αὐτοὺς ποὺ σχεδιάσαμε πολέμους καὶ ἐνεργοποιήσανε τοὺς δυὸ παγκοσμίους πολέμους καὶ κατέχουνε τὶς ΗΠΑ καὶ τὴν Ε.Ε. καὶ Γερμανία πλήρως , καὶ θέλουν νὰ καταστρέψουν ὅλο τὸν πλανήτη καὶ νὰ τὸν ὑποτάξουν.
\item Ὅμως ἐπειδὴ πολὺς κόσμος ἔχει ταλαιπωρηθεῖ, τώρα δίνεται ἡ εὐκαιρία νὰ δουλέψει ὁ καθένας χωρὶς νὰ φοβᾶται μήπως κάποιο "ΤΕΒΕ" τοῦ κλείσει τὸ σπίτι μὲ ὅλες αὐτὲς τὶς ἀντισυνταγματικὲς χρεώσεις καὶ προσαυξήσεις. Ὅλα αὐτὰ καταργοῦνται. Γιατί σύμφωνα μὲ τὸ Σύνταγμα ὅλοι δικαιοῦνται νὰ ἐργάζονται -ἄσχετα ἂν χρωστᾶνε ὄχι- καὶ συνεισφέρουν ἀνάλογα μὲ τὶς δυνάμεις τοὺς δηλαδὴ δὲν ὑπάρχει κάποιο "ΤΕΒΕ"-μπαμπούλας νὰ σὲ ἀπειλεῖ, ἔχεις δὲν ἔχεις κέρδη ἐγὼ θὰ σοῦ τὰ πάρω, βρέξει χιονίσει, καὶ θὰ σοῦ βάζω καὶ κάτι τρελὲς προσαυξήσεις. Ὅλοι ὅσοι ἔχουν δημιουργήσει καὶ ἐνεργήσει αὐτὴ τὴν νομοθεσία νὰ χάνουν κινητὴ καὶ ἀκίνητο περιουσία καὶ μέχρι τὰ γεράματά τους νὰ ξεσκατώνους κώλους γερόντων, γιατί διέφθειραν, ἐπρόδωσαν, ἀλλοίωσαν συνειδήσεις καὶ Σύνταγμα γιὰ τὴν πάρτη τους, τὶς θεσοῦλες καὶ τὴ βόλεψή τους. Κινητὴ καὶ ἀκίνητο περιουσία. Ἔτσι ὅλοι οἱ πολιτικοὶ ποὺ πήρανε χρήματα ἀπὸ τὸ Δημόσιο κουρβανὰ γιὰ τὰ "κόμματά τους", δηλαδὴ τὶς συμμορίες τῶν ἀχρήστων ἀπατεώνων ποὺ βόλεψαν, τώρα νὰ πληρώσουν. Ὑπάρχουν πλέον δικαστήρια.
\item Κάθε Δῆμος εἶναι ὑποχρεωμένος νὰ βοηθᾶ μὲ κάθε τρόπο στὴ δημιουργία συνεταιρισμῶν τῶν κτηνοτρόφων μὲ σκοπὸ τὴ βελτίωση τῆς ποιότητας καὶ τῆς παραγωγῆς καὶ τὴν εἰς καλὲς τιμὲς πώλησή τους, στὴ δημιουργία ὑποπροϊόντων ἀκόμη καὶ στὴν ἐξαγωγὴ τοὺς κατὰ τὸν τύπο τοῦ συνεταιρισμοῦ τῶν Ἀμπελακίων, τῆς ὀργανώσεως τοῦ Σβάρτς, ὅπου κάθε ἔτος θὰ μεταβάλλεται τὸ Δ.Σ. ὅπως καὶ τὰ διεθνῆ καταστήματα θὰ ἀλλάξουν οἱ ὑπάλληλοι ἐτησίως.
\item Ἀπαγορεύεται ἡ χρήση συμβολαιογράφων ἐκτὸς Δήμου γιὰ ἀκίνητα. Δηλαδὴ ὅπου εἶναι τὸ ἀκίνητο γίνεται χρήση τοπικοῦ συμβολαιογράφου.
\item Ἀπαγορεύεται σουπερμάρκετ νὰ λειτουργοῦν ὡς φαρμακεία.
\item Οἱ Δῆμοι ἀπαγορεύεται νὰ κάνουν χρήση τοῦ δημόσιου κεφαλαίου, νὰ λειτουργοῦν ὡς ἐπιχειρήσεις καὶ νὰ ἐνοικιάζουν γιὰ λογαριασμό τους, γιὰ λογαριασμὸ τῶν τοπικῶν ἀρχόντων, γιὰ τὴ δημιουργία θέσεων ἐργασίας οἱοδήποτε τόπο ποὺ κατέχουν δηλαδὴ πεζοδρόμια θέσεις περιπτέρων, παραλίες κ.λπ.
\item Όποιος προσβάλλει την Ελλάδα και τους Έλληνες είναι υποχρεωτική η εκτέλεσή του.
\item Απαγορεύεται η ψήφος κάτω των 24 ετών.
\item Απαγορεύεται η ψήφος στους τεμπέληδες.
\item Απαγορεύεται η ψήφος στους ανέργους.
\item Ἀπαγορεύεται ἡ διαχείρισις τοῦ χρήματος ἀπὸ τοὺς αἱρετοὺς ἄρχοντες. Αὐτὸ τὸ ἀναλαμβάνει εἰδικὴ ὑπηρεσία ἐπιβλεπομένη ὑπὸ τῶν εἰσαγγελέων τοῦ Δήμου καὶ τῆς Ἀστυνομίας καὶ Ἀσφαλείας τοῦ Δήμου.
\item Ἀπαγορεύεται ποινῆς ἐκτελέσεως, δημόσιας ἐκτελέσεως, οἱ αἱρετοὶ ἄρχοντες νὰ διαχειρίζονται δημόσιο χρῆμα.
\item Ὅσοι αἱρετοὶ ἄρχοντες(βουλευτὲς κλπ) μοίρασαν σὲ δικούς τους δημόσιο χρήμα νὰ κατάσχεται κινητὴ καὶ ἀκίνητος περιουσία καὶ μὲ διεθνή εντάλματα νὰ κατάσχονται τὰ πάντα καὶ νὰ ἐκτελοῦν μέχρι τὸ τέλος τῆς ζωής τους κοινωνικὴ ἐργασία.
\item Κουκουλοφόροι, καταληψίες καὶ λοιποὶ μπαχαλάκηδες νὰ συλλαμβάνονται ἄμεσα καὶ νὰ μπαίνουν σὲ δικαστήριο μὲ 100 ἐνόρκους ἀπὸ τὸ Δῆμο και πληρώνουν άμεσα τις καταστροφές που έχουνε κάνει. Τώρα μπορούν να εκφραστούνε άμεσα και όχι να πράτουν τα έργα των εβραιοσιωνιστών υλοποιώντας την πολιτική τους.
\item Οἱ Δῆμοι γίνονται ὅπως παλαιά. \textbf{Δῆμοι καὶ κοινότητες. Καταργεῖται ὁ Καλλικράτης.}
\item Όλοι οι Υπάλληλοι της Δημοκρατίας δεν αποζημιώνονται ανά Δήμο ίσα. Ο Δήμος καθορίζει το ποσό. Ανάλογα την ανάπτυξη του Δήμου είναι και οι αποζημιώσεις και οι συντάξεις.
\item Οι αιρετοί δεν αποζημιώνονται καθ οιονδήποτε τρόπο.
\item Οἱ Δῆμοι καὶ οἱ κοινότητες παραχωροῦν χωρὶς ἀνταπόδοση τόπους, πεζοδρόμια, παραλίες, μὲ ἀπόφαση τοῦ συμβουλίου τῶν 7 καὶ αἰτιολόγηση πρὸς τὸ γενικὸ καλό. Ἂν ἀποδειχθεῖ ὅτι ἡ ἀπόφασή τους ἦταν γιὰ τὸ συμφέρον τους νὰ χάνουν κινητὴ καὶ ἀκίνητο περιουσία. Αὐτὲς οἱ ἀποφάσεις ἔρχονται σὲ δημόσια διαβούλευση καὶ τὸ μηνιαῖο δημοψήφισμα, ἀφοῦ πρῶτα κάθε θέμα ἀναπτυχθεῖ στὴν ἐφημερίδα τοῦ Δήμου.
\item Στὸ κίνημα ποὺ δημιουργεῖται, ὁ συγγραφεὺς καὶ ἱδρυτὴς εἶναι ἁπλὰ ὁ ὑψηλὰ ἐπιβλέπων, προκειμένου οἱ ὁμάδες νὰ ψηφίζουν δημοκρατικὰ . Κάθε ὁμάδα ἔχει μία ψῆφο καὶ μάλιστα δημοσιευμένη. Πάνω ἀπὸ 7 ὁμάδες ἀποτελοῦν τὴ συντονιστικὴ ἐπιτροπὴ τοῦ κινήματος ἀλλὰ καὶ τὴν ἐκτελεστική. Οἱ ὁμάδες αὐτὲς ἀλλάζουν κάθε χρόνο μὲ ψήφισμα ὅλων τῶν ὁμάδων ἠλεκτρονικά. Δὲν εἶναι ἀπαραίτηση ἡ φυσικὴ παρουσία. Ὅταν μαζευθοῦν πάνω ἀπὸ 1000 ὁμάδες, τότε θὰ ἐνεργηθεῖ τὸ πρῶτο συνέδριο. Ἀπαγορεύεται ἡ χρηματοδότηση. Τὸ ὅλο κίνημα εἶναι ἰδεολογικὸ καὶ πνευματικὸ μὲ στόχο τὴν ἐπιβολὴ τοὺς νέου Συντάγματος καὶ τὴν ἀμεσοδημοκρατικὴ λειτουργία τοῦ κράτους.
\item Ἀπαγορεύεται ἡ αἰσχροκέρδεια. Τὰ προϊόντα ἀπαγορεύεται νὰ φθάνουν στὴν κατανάλωση μὲ πάνω ἀπὸ 30\% κέρδος και λιγώτερο από 5\%. Τώρα μὲ αὐτὸ τὸ σύστημα ὅλοι πληρώνουν ἀναλογικὰ \textbf{--ΕΚΤΟΣ ΑΠΟ ΤΟΥΣ ΠΑΡΑΓΩΓΟΥΣ ΟΙΟΔΗΠΟΤΕ ΠΡΟΪΟΝΤΩΝ ΓΙΑ ΤΗΝ ΠΡΩΤΗ ΠΕΝΤΑΕΤΙΑ-} στὰ εἰσερχόμενα. Ὁ βασικὸς αὐτὸς κανόνας ἐξισοῤῥοπεῖ τὴν αἰσχροκέρδεια(ἐνῶ παράλληλα αὐτοδίκαια δίνει τὰ κύρια ποσὰ στοὺς παραγωγοὺς καὶ ἐνισχύει κάθετα τὶς παραγωγικὲς διαδικασίες). Σχετικὰ μὲ αὐτὸ μπορεῖτε νὰ διαβάσετε τὸ σχετικὸ κείμενο τοῦ Τραϊανοῦ σχετικὰ μὲ τὴν καταπολέμηση τῆς φτώχειας. Στὴν πράξη αὐτὸ τὸ κείμενο ἀποκαλύπτει ὅτι γιὰ νὰ βάλουνε χέρι στὸν τόπο οἱ ξένοι χρησιμοποίησαν τοὺς \textbf{προδότες πολιτικούς μας}. Οἱ \textbf{βρωμιαρέοι πολιτικοί μας, ἄχρηστοι κι ἐπικίνδυνοι} δώσανε στὰ δικά τους λαμόγια προνόμια, βάλανε σὲ θέσεις εὐθύνης ἀχρήστους καὶ ἐπιβλαβεῖς δικούς τους, βάλανε στὸ χέρι δημόσια περιουσία καὶ ἐπιτρέψανε στοὺς βλαχοδημαρχέους νὰ ἐκμεταλεύονται τὴ δημόσια περιουσία (πεζοδρόμια, παραλίες, ἀκίνητα κ.λπ.) αὐξάνοντας τὰ χρήματα ἐπιβιώσεως, καταστρέφοντας τοὺς παραγωγοὺς κλπ..
\item Οποιαδήποτε παραγωγική διαδικασία \textbf{ΔΕΝ ΦΟΡΟΛΟΓΕΙΤΑΙ ΓΙΑ ΤΗΝ ΠΡΩΤΗ ΠΕΝΤΑΕΤΙΑ}.
\item Δημιουργείται η \textbf{Υπηρεσία Ελέγχων Δημοσίων Έργων} με ετησίους επιλεγμένους \textbf{Υπαλλήλους της Δημοκρατίας} των οποίων η σύνθεση είναι Ένας Πολιτικός Μηχανικός με αποδεδειγμένη εμπειρία στα στατικά, ένας οικονομολόγος αποδεδειγμένης ικανότητος, ένας Εισαγγελέας, ένας Τοπογράφος Μηχανικός αποδεδειγμένης ικανότητος και επάρκειας, ένας Μηχανολόγος Μηχανικός αποδεδειγμένης ικανότητος και επάρκειας, όλοι \textbf{τυχαία επιλεγμένοι} από κάθε Δήμο. Η υπηρεσία αυτή έχει σκοπό τον έλεγχο όλων των Δημοσίων έργων πριν και μετά και την απόδωση ευθυνών.
\ Οι \textbf{Ὑπάλληλοι Τῆς Δημοκρατίας} είναι υποχρεωμένοι να γνωρίζουν άπταιστα το Σύνταγμα και \textbf{να το υπηρετούν με τη ζωή τους}.
\item Κάθε μορφής Υπάλληλος θεωρείται \textbf{Ὑπάλληλος Τῆς Δημοκρατίας}.
\item Η Ελλάδα \textbf{ΔΕΝ ΑΝΑΓΝΩΡΙΖΕΙ ΤΗ ΣΥΝΘΗΚΗ ΤΗΣ ΛΙΣΣΑΒΩΝΑΣ} την οποίαν έστησαν οι χιτλερικοί με τους εβραίους όπως αποκαλύφθηκε.
\item Απαγορεύεται πρόσωπα προδοτών, βουλευτών του ΠΑΣΟΚ, ΝΔ, ΣΥΡΙΖΑ κλπ να βρίσκονται σε δρόμους, αίθουσες κλπ. Η μνήμη όλων των προδοτών και των σφουγκοκωλάριών των πρέπει να εξαλειφθεί από την μνήμη του Έθνους. Αυτή η περίοδος της ιστορίας θα αποτελεί την \textbf{μαύρη περίοδος της προδοσίας και του δωσιλογισμού}...
\item Στην Ελληνική Δημοκρατία απαγορεύονται τα ιδιωτικά ΜΜΕ επί ποινής Θανάτου, γιατί αυτά αυτά αποτελούν το εργαλείο των προδοτών. Στην Ελληνική Δημοκρατία μπορούν και υποχρεούνται όλοι να έχουν πρόσβαση στα ΜΜΕ στα οποία ανήκουν στη Δημοκρατία, οι δε \textbf{Ὑπάλληλοι τῆς Δημοκρατίας} υποχρεωτικά υπηρετούν με θρησκευτική αφοσίωση το Σύνταγμα τῆς Δημοκρατίας, απόλυτα το Δίκαιο, τη Ισσονομία και Ισσοπολιτεία γι αυτό \textbf{και ψηφίζονται}.
\item Ἀπαγορεύεται ταμπέλες καταστημάτων μὲ ξένους χαρακτῆρες, ὑποχρεωτικὰ Ἑλληνικὲς καὶ πολυτονικές.
\item Ἀπαγορεύεται τὸ κηφηνεῖν. Οἱ κηφῆνες ποὺ ζοῦν ἀπὸ ἐνοίκια καὶ περιουσίες χωρὶς νὰ ἐργάζονται ὅταν εἶναι κάτω ἀπὸ 60 ἐτῶν νὰ χάνουν κινητὴ καὶ ἀκίνητο περιουσία. Τώρα ὅλοι μποροῦν νὰ ἐργάζονται, νὰ εἶναι παραγωγικοί, νὰ συνεισφέρουν στὸ γενικὸ καλό, γιατί τώρα δὲν ἐκβιάζονται μὲ ὅλα αὐτὰ τὰ δεσμά(ἐνοίκια, ΦΠΑ, ταμεῖα κλπ). Μπορεῖ κάποιος ὡς ἐργασία νὰ δηλώσει ἐρευνητής, ἀλλὰ εἶναι ὑποχρεωμένος ἐντὸς 3 ἐτῶν τὸ μέγιστο νὰ ἔχει ἐπιδείξει τὸ ἀποτέλεσμα τῆς ἐρεύνης του ὥστε αὐτὸς ὁ ἴδιος ἢ οἱ ἄλλοι νὰ τὴν ἐκμεταλευθοῦν). Ὁ τοπικὸς εἰσαγγελέας μπορεῖ καὶ ἐπιβάλλεται νὰ προβαίνει στὸν ἔλεγχο τῆς ἐρεύνης ὥστε νὰ μὴν εἶναι ἀπάτη. Οἱ ἐρευνητὲς μὲ τὴν ἔγγριση τῆς Δημογεροντίας μποροῦν νὰ τύχουν ἐνισχύσεως ἂν τὴν ζητήσουν δημόσια μέσω τῆς ἐφημερίδος τοῦ Δήμου, καὶ αὐτὴ ἡ ἔρευνα εἶναι σημαντικὴ γιὰ τὸ κοινωνικὸ σύνολο.
\item Ἀπαγορεύονται οἱ ἐπιδοτήσεις ἀπὸ τὸ ἐξωτερικὸ καὶ οἱ ἐπιδοτήσεις γενικά. Ἐπιτρέποπνται μόνον οἱ ἐνισχύσεις ἀπὸ τὴν ἀπόφαση τῆς Δημογεροντίας.
\item Ἡ πανούκλα τῆς τρίτης χιλιετίας εἶναι ἐπιβολὴ τῆς διαστροφῆς ὡς δεδομένου καὶ μὴ κατακριτέου. Ἡ ἀνοχὴ τῆς διαστροφῆς σὲ πρώτη φάση μὲ τὴν ἐπιβολή της σὲ δεύτερη. Ἡ δημιουργία ὁμοφυλόφιλων, λεσβιῶν, πούστηδων καὶ λοιπὲς διαστροφές, πράγματα πρωτόγνωρα γιὰ μία \textbf{τίμια καὶ ἠθικὴ καὶ Ὀρθόδοξη} κοινωνία ὅπως Ἑλληνική. Ἡ νέα τάξη πραγμάτων ἡ ἀρχὴ τῆς ὁποίας ξεκινάει ἀπὸ τοὺς ρώτσιλτ, τὸ σιωνισμὸ καὶ κάθε εἴδους σατανικὰ καὶ φασιστικὰ περιβάλλοντα, ἔχει σκοπὸ μέσα ἀπὸ τὴν ἐκπαίδευση, τὴν κοπὴ τῶν τέκνων ἀπὸ τοὺς γονεῖς, ἀπὸ ἀνώμαλες τηλεοπτικὲς παραγωγές, ἀπὸ ταινίες ποὺ προβάλλονται τὴν ἀνοχὴ στὴ διαφορετικότητα καὶ ἄλλες ὡς τέτοιες προβοκατόρικες μποῦρδες καὶ προπαγάνδα τῶν σατανιστῶν δὲν ἔχουν καμία θέση στὸ ἑλληνικὸ κράτος καὶ ἔθνος. Εἴμαστε ἔθνος ἀνάδελφον καὶ ἀφοῦ ἀνακαλύψαμε τὴν ἀπάτη πού μας ἔχουν στήσει καὶ ποὺ τὸ πάνε τὸ πράγμα, γιὰ τὴν Πίστη μας, τὴν Πατρίδα μας, τὴν οἰκογένειά μας ἀξίζει νὰ δώσουμε μάχη μὲ κάθε ἐχθρό, χωρὶς νὰ φοβηθοῦμε γιὰ νὰ ὑποχωρήσουμε. Δὲν πρέπει νὰ ξεχνοῦμε ὅτι θέλει ἀρετὴ καὶ τόλμη ἡ ἐλευθερία.
\item Ἀπαγορεύονται παρελάσεις πούστηδων καὶ λεσβιῶν. Ἀπαγορεύεται νὰ ἀνήκουμε σὲ ἕνωση ποὺ προβάλλει τέτοια πράγματα. Κτηνοβασίες, ἀνωμαλίες κ.λπ. ἂν προβάλει κάποια ἕνωση ἀπαγορεύεται ἐμεῖς οἱ Ἕλληνες νὰ ἀνήκουμε. Ὅποιος λέει τὰ ἀντίθετα \textbf{νὰ ἐκτελεῖται στὸ Σύνταγμα ἐνώπιον ὅλων}, ὥστε οἱ βρωμεροὶ νὰ ἀντιληφθοῦν ὅτι τὴν Πατρίδα μας δὲν θὰ τὴν ἀφήσουμενα κυβερνᾶται ἀπὸ τὰ ὄργανα τῆς νέας τάξης μὲ τὶς κουφάλες τοὺς τραπεζίτες. Εἶναι γνωστὸ ὅτι ἡ νέα τάξη θέλει νὰ προχωρήσει τοὺς πούστηδες καὶ τὶς λεσβίες ὥστε νὰ μὴν ἔχει σθεναρὴ ἀντίδραση στὶς πολιτικές της ποὺ εἶναι ὁ πλήρης ἔλεγχός μας. Δὲν πρέπει νὰ τὸ ἐπιτρέψουμε.
\item Ἡ κεντρικὴ κυβέρνηση ὑποχρεοῦται νὰ χρησιμοποιεῖ τρίτες ὑπηρεσίες ἢ εἰδικοὺς ἀνθρώπους σὲ συμφωνία μὲ τὸν εἰσαγγελέα τοῦ Δήμου, χωρὶς τὴ συμφωνία τῶν ἄλλων αἱρετῶν του Δήμου προκειμένου νὰ καταστείλλει μία παρανομία. 
\item Τὰ μεταλλεῖα Χαλκιδικὴς παραδίδονται στοὺς κατοίκους τῆς Χαλκιδικῆς γιὰ οἰκολογικὴ καὶ ἐξόρυξη ἐνῶ ἐκδίδεται διεθνὲς ἔνταλμα συλλήψεως αὐτῶν ποὺ κρύβονται πίσω καὶ κατασχέσεως τῶν περιουσιῶν των.
\item Κάθε Δῆμος σὲ συνεργασία μὲ τοὺς δημογέροντες, τὰ πανεπιστήμια, τὸν εἰσαγγελέα, δημιουργεῖ καταλύματα καὶ ὑποδομὲς γιὰ τὴ διάσωση ἀνθρώπων ἐκπεσμένων(πόρνες, ἱερόδουλες, ναρκωμανεῖς, ἔνεργοι καὶ φτωχοὶ κλπ.).
\item Δημιουργεῖται ἡ Ἑλληνικὴ μυστικὴ ὑπηρεσία ἀντίστοιχη της μοσσάντ μὲ συμμετοχὴ εἰδικῶν ποὺ ἐγκρίνονται ἀπὸ τοὺς εἰσαγγελεῖς, τῆς ἀστυνομίας καὶ ἀσφαλείας κάθε Δήμου, καὶ ἔρχονται 13 ἀπὸ κάθε Δῆμο, αἱρετοὶ γιὰ 3 χρόνια. Ἀνάλογα μὲ τὶς ἐπιτυχίες τους νὰ ἐπιλέγονται οἱ ἄξιοι, διαβασμένοι, πατριῶτες, ὀρθόδοξοι, ἀμίληκτοι στὸν ἐχθρό, ἀπόλυτοι στὴ Δικαιοσύνη, μὲ ὀξυμένο τὸ αἴσθημα τοῦ Δικαίου, καλοπροαίρετοι στὶς γνῶμες τῶν ἄλλων, συνεπεῖς στὶς ὑποχρεώσεις τους.
\item Ἀπαγορεύεται ὁ σεξουαλικὸς τουρισμός.
\item Οἱ πολίτες νὰ ψάχνουν νὰ βρίσκουν μορφωμένους ἀνθρώπους καὶ μορφωμένοι ἄνθρωποι δὲν εἶναι αὐτοὶ ποὺ ξέρουν πολλά. Ἡ μόρφωση κατὰ τὸ Σωκράτη εἶναι θέμα συμπεριφορᾶς. Ὁ Σωκράτης θεωρεῖ μορφωμένους ἐκείνους οἱ ὁποῖοι: Ἐλέγχουν δυσάρεστες καταστάσεις, χωρὶς νὰ ἐλέγχονται ἀπὸ αὐτές. Αὐτοὶ ποὺ ἀντιμετωπίζουν ὅλα τὰ γεγονότα μὲ γενναιότητα καὶ λογική. Αὐτοὶ ποὺ εἶναι ἔντιμοι σὲ συναλλαγές. Αὐτοὶ ποὺ ἀντιμετωπίζουν γεγονότα δυσάρεστα καὶ ἀνθρώπους ἀντιπαθεῖς καλοπροαίρετα. Αὐτοὶ ποὺ ἀπολαμβάνουν τὴν κάθε ἡμέρα τους. Αὐτοὶ ποὺ δὲν ἡττήθηκαν ἀπὸ τὶς ἀτυχίες καὶ τὶς ἀποτυχίες τους. Αὐτοὶ ποὺ δὲν ἔχουν φθαρεῖ ἀπὸ τὶς ἐπιτυχίες καὶ τὴ δόξα τους. Παρακαλοῦνται οἱ πολίτες, ἂν θέλουν νὰ ζήσουν ἤρεμα καὶ εὐτυχισμένοι, νὰ ψάχνουν νὰ βρίσκουν τέτοιους ἀνθρώπους, νὰ τοὺς ψηφίζουν μὲ κάθε κόστος καὶ νὰ μὴν ψηφίζουν πολιτικάντηδες γιὰ νὰ γελᾶνε μαζί τους ἢ γιὰ τὴ βόλεψή τους.
\item Ἐὰν βρεῖτε ἕναν καλὸ μὴν τὸν ἀφήνετε, ἀλλά νὰ τὸν ὑποχρεώνετε νὰ παίρνει τὴν ἐξουσία καὶ ὁ Θεὸς θὰ εἶναι βοήθεια σὲ ὅλους.
\item Οἱ ὑποδομὲς τῆς χώρας ἐπανέρχονται ὅπως πρὶν τὴν Ε.Ε. Τὰ 7 ἐργοστάσια ζαχάρεως ἐπαναλλειτουργοῦν. Ὅτι ὑποδομὴ ἀπεβλήθη ἀπ᾿ τὴν Ε.Ε. ἐπανέρχεται . Οἱ προδότες ποὺ πράξανε τοῦτα περνοῦν δικαστήρια, κατάσχεται ἄμεσα κινητὴ καὶ ἀκίνητος περιουσία τῶν τέκνων τῶν προδοτῶν, ἡ ΕΕ πάει στὰ δικαστήρια γιὰ ἀπάτη, ἐκβιασμό, ἡ Ἀγγλία ἐπίσης καὶ κάθε τοκογλύφος ποὺ βρῆκε τρόπο νὰ δράσει. Τώρα μὲ τὸ νέο σύστημα δικαστηρίων τὰ πάντα εἶναι εὔκολα, γρήγορα, σωστά.
\item Ἀπὸ τὴ στιγμὴ ποὺ βγεῖ κυβέρνηση ἀπὸ τὸ κίνημα, ὅλα αὐτὰ ἰσχύουν ἄμεσα χωρὶς καμμία ἐπικύρωση ἀπὸ τὴ βουλή. 
\item Οἱ ἑπίτιμοι πολίτες ἔχουν δικαίωμα ψήφου.
\item Κάθε μέλος, φίλος κλπ του κινήματος μπορεί να βρίσκεται σε όποιο κόμμα ή κίνημα θέλει. Η Ελευθερία είναι επιβεβλημένη και κανείς δεν εξαναγκάζεται να κάνει κάτι που δεν θέλει. Σε περίπτωση όμως που έχει υπογράφει το παρόν, είναι υποχρεωμένος όπου ανήκει να το διακρατεί και να το ψηφίζει και να το προωθεί, προωθώντας την Ελευθερία μας και την Ελληνικότητά μας και την Ορθοδοξία μας.

\item  \textbf{ΠΑΙΔΕΙΑ}

\item Οι πανεπιστημιακοί καθηγητές γίνονται αιρετοί και προσλαμβάνονται μετά από έλεγχο γνώσεων από επιτροπή ειδικών, Ελλήνων και ξένων, που θα καθορίζει η κυβέρνηση, κάθε τρία χρόνια και από το έργο που θα έχουν παραδώσει και με δημόσιες εξετάσεις.
\item Ἡ ἐπίσημος γλώσσα τῆς Ἑλλάδος εἶναι ἡ ἀπλὴ καθαρεύουσα καὶ πολυτονική.
\item Ἡ Ἑλλὰς θὰ κάνει Ὀλυμπιακοὺς ἀγῶνες μόνο στὴν ἀρχαία Ὀλυμπία καὶ πουθενὰ ἀλλοῦ.
\item Τὸ ὑπουργεῖον παιδείας ἔχει αἱρετοὺς ὑπαλλήλους. Κάθε χρόνο ἀλλάζουν. Κάθε Δῆμος ἀποστέλλει ἕναν ἐκλεγμένο ὑπάλληλο, κατόχου ἀναλόγων πτυχίων καὶ ἱκανοτήτων. Κάθε ὑπάλληλος τοῦ ὑπουργείου παιδείας δύναται νὰ ἔχει τρεῖς ὑπαλλήλους βοηθούς. Υποχρεωτική η δημιουργία LGPL βιβλίων σε \XeLaTeX.
\item Τὰ μαθήματα τοῦ γυμνασίων καὶ λυκείων διακρίνονται σὲ βασικὰ καὶ ἐλεύθερα. Τὰ βασικὰ εἶναι τὰ ἀρχαία, γλώσσα, λογική, τὴ φιλοσοφία, ρητορική, μαθηματικά, γεωμετρία, φυσικὴ ἐνῶ ἐλεύθερα εἶναι ὅλα τὰ ὑπόλοιπα. Τὰ πέντε ὄγδοα τῶν ὡρῶν διδασκαλίας εἶναι στὰ βασικά. Καὶ πρῶτα ἀπ' ὅλα ἡ ἱστορία γιατί δημιουργεῖ χαρακτῆρες.
\item Τὰ γυμνάσια καὶ τὰ λύκεια ξεκινοῦν στὶς ὀκτὼ τὸ πρωὶ καταλήγουν σὲ 6 μ.μ. Προάριστον καὶ ἄριστον ἐνεργοῦνται στὰ σχολεῖα μὲ τὴν εὐθύνη τοῦ Δήμου. Τὸ ξεκίνημα ἔχει ὡς ἀρχὴ τὴν πρωϊνὴ προσευχή, στὴ συνέχεια τὴν ἔπαρση τῆς σημαίας καὶ τὸν ἐθνικὸ ὕμνο, καὶ ἀκολούθως ὁπωσδήποτε κατ᾿ ἐλάχιστον μία ὥρα ἀθλοπαιδειές.
\item Τὰ σχολεῖα λειτουργοῦν πέντε ἡμέρες τὴν ἑβδομάδα ἐνῶ ἡ ἕκτη, δηλαδὴ τὸ Σάββατο ἐκπαιδεύονται στὴν τοξοβολία, σκοποβολή, μὲ τὴν εὐθύνη τοῦ Δήμου.
\item Τὸ πρῶτο μάθημα τῶν γυμνασίων εἶναι ἡ ἀπόλυτη γνώση τοῦ Συντάγματος καὶ ἐν συνέχειᾳ τοῦ ἀστικοῦ καὶ ποινικοῦ κώδικος.
\item Ὅλοι οἱ Ἕλληνες πολίτες ὑποχρεοῦνται σὲ κατοχὴ τριῶν διαφορετικῶν ὅπλων καὶ ἐκπαίδευσή τους εἴκοσι μέρες τὸ χρόνο καὶ μία φορὰ τὸν μήνα. Ἀπαγορεύεται ἔποικοι καὶ ξένοι νὰ φέρουν τὸ οἱοδήποτε ὅπλο.
\item Τὰ πρῶτα καὶ βασικὰ μαθήματα στὰ γυμνάσια εἶναι ἡ μελέτη καὶ ἀπομνημόνευση τοῦ Συντάγματος.
\item Τὸ δεύτερο καὶ σημαντικότατο εἶναι ἡ ἀποστήθιση τοῦ ἐθνικοῦ ὕμνου καὶ ἡ μελέτη του ὅπως ἐπίσης καὶ τὰ κείμενα τῆς ἐθνοσυνελεύσεως τῆς Τροιζίνας, τὰ ἀπομνημονεύματά του Κολοκοτρώνη, τοῦ Μακρυγιάννη, καὶ ἐν τέλει ἡ τέλεια γνώση τῆς ἐπαναστάσεως τοῦ 21, τοῦ ἀθανάτου κρασιοῦ τοῦ 21.
\item Τη σημαία την κρατούν στις παρελάσεις μόνο Έλληνες στο γένος και οι πλέον άριστοι.
\item Ἡ ἱστορία στὸ Γυμνάσιο καὶ Λύκειο θὰ διδάσκονται στὸ πρωτότυπο καὶ ὄχι ἀποσπασματικά. Μὲ αὐτὸ τὸν τρόπο ἔχωμε νὰ δημιουργήσουμε σωστοὺς χαρακτῆρες στοὺς νέους. Το κόλπο που έχει παιχθεί με τους πράκτορες του Υπουργείου Παιδείας είναι ότι στοχευμένα δημιουργούν ύλη που δεν συνάδει με την πραγματική ανάγκη και γνώση καταργώντας την αριστεία και μάλιστα νομοθετώντας την.
\item Ὅλες μεγάλες γιορτὲς τῆς Ὀρθοδοξίας, τοῦ Σταυροῦ, τοῦ Ἁγίου Πνεύματος κ.λπ. εἶναι ἡμέρες ἀργίας ὅλων καὶ τὸ ἐορτολόγιον εἶναι τὸ Ἰουλιανό.
\item Ὅλοι οἱ φοιτητὲς τῶν Πολυτεχνείων νὰ ἐκπαιδεύονται σοβαρὰ στὶς C,C++,PHP,Python,MySQL κατ᾿ ἐλάχιστον.
\item Οἱ φοιτητὲς ὑπολογιστῶν υποχρεούνται σε γνώση του GCC.
\item Ἡ χρήση γλωσσῶν νὰ γίνεται μὲ open source projects ὅπως CodeBlocks, ἐνῶ ἀκόμη βασικὴ εἶναι ἡ γνώση τοῦ WxWidgets framework.
\item Οἱ φοιτητὲς τῶν Πολυτεχνείωv νὰ συμμετέχουν ὑποχρεωτικὰ σὲ open source projects ὡς διπλωματικές, ὅπως π.χ. inkcape, scribus, reactOS, gimp, openoffice, κλπ 9)Ἡ Ἑλλάδα νὰ συμμετέχει στὸ σχεδιασμὸ ἐπεξεργαστῶν καὶ νὰ φέρει τοὺς Ἕλληνες ποὺ ἔχουν σχεδιάσει τέτοιους.
\item Ἡ ἀμυντικὴ βιομηχανία νὰ χρησιμοποιεῖ φοιτητὲς γιὰ ἀνάπτυξη chip, τεχνικῶν, ἀποκωδικοποιητῶν, νέων πρωτοκόλλων μὲ τὴ βοήθεια τῶν καθηγητῶν πανεπιστημίων.
\item Οἱ διατριβὲς κατατίθενται σὲ \LaTeX, \XeLaTeX κλπ 
\item Στὰ πανεπιστήμια δημιουργεῖται ἕδρα λειτουργικῶν συστημάτων μὲ βάση τῶν open source projects, ὡς ἐπίσης καὶ τμῆμα πρωτοκόλλων καὶ δικτύων γιὰ στρατιωτικὲς ἐφαρμογές.
\item Κάθε καθηγητὴς Γυμνασίου-Λυκείου ἔχει τὴν αἴθουσά του στὴν ὁποίαν πηγαίνουν οἱ μαθητές.
\item Ἐπιτρέπεται οἱ καθηγητὲς γυμνασίων λυκείων νὰ δίδουν ἰδιωτικῶς μαθήματα.
\item Ἡ παρακολούθηση κατὰ 90\% τῶν μαθημάτων τῶν Πανεπιστημίων εἶναι ὑποχρεωτική. Οἱ φοιτητὲς χάνουν τὸ ἔτος διαφορετικά.
\item Ἀπαγορεύονται τὰ ἑξάμηνα. Μόνο ἔτη ὅπως παλαιά.
\item Οἱ ἐξετάσεις κάθε μαθήματος εἶναι σὲ ὀκτὼ ὧρες μὲ ὅλα τὰ μέσα τὰ ὁποῖο μπορεῖ νὰ ἔχει ἕνας φοιτητής.(ὑπολογιστές, βιβλία κλπ.).
\item Για την εισαγωγή στα πανεπιστήμια ο υποψήφιος δεν έχει συγκεκριμένη ύλη, όπως ήταν το σύστημα της δεκαετίας του 70 που έβγαιναν πολύ μορφωμένοι και σωστοί επιστήμονες.
\item Ο μέγιστος χρόνος αποφοιτήσεως από μια σχολή είναι 1,5 φορές τα χρόνια της, μετά ο φοιτητής διαγράφεται διά παντός, εκτός δικαστικής αποφάσεως από το ορκωτό δικαστήριο.
\item Ο πολίτης δύναται να παρακολουθεί όσες σχολές έχει επιτύχει και εγγραφεί.
\item Κάθε φοιτητὴς ποὺ παίρνει πτυχίο οἱασδήποτε σχολῆς πρέπει νὰ γνωρίζει νὰ δημιουργεῖ ἕναν ὑπολογιστή, νὰ ἐγκαθιστᾶ ἕνα λειτουργικὸ σύστημα καὶ νὰ χρησιμοποιεῖ κατ᾿ ἐλάχιστον ἕνα open source σουίτα γραφείου, ὅπως τὸ Libre Office, νὰ μπορεῖ νὰ ρυθμίσει ἕναν mail client, νὰ στείλει καὶ νὰ λάβει emails.
\item Τὰ σημερινὰ Λύκεια ἑτοιμάζουν παιδιὰ κιμά, στὸ δὲ σημερινὸ Πανεπιστήμιο τὰ ἀποτελειώνουν καὶ τὰ κάνουν βλαμμένα, συστημικά, ἀνθρώπους ἀνταλλακτικὰ γιὰ τὶς κωλοπολυεθνικὲς ποὺ ἔχουν στηθεῖ ἔξυπνα καὶ πονηρά. Τώρα ὅμως μὲ αὐτὸ τὸ ἀμεσοδημοκρατικὸ σύστημα τελειώνουν αὐτὲς οἱ ἱστορίες. Οἱ ἐφευρέτες, οἱ δημιουργοὶ πατεντῶν δικαιοῦνται αὐτοδικαίως μὲ ἕνα ἁπλὸ δικαστήριο καὶ μερίδα τοῦ λέοντος ἀπ᾿ τὶς κωλοπολυεθνικὲς ποὺ χρησιμοποίησαν τὴν τεχνογνωσία τους ἢ τὶς πατέντες τους, μὲ ὁποιαδήποτε σύμβαση. Τί ἔχει συμβεῖ στὴν πράξη; Ὅλες αὐτὲς οἱ κουφάλες τῶν πολυεθνικῶν στήσανε ἕνα ἔξυπνο σύστημα γιὰ τὴν ἐκμετάλευση αὐτῶν ποὺ πραγματικὰ μόχθησαν, τῶν ἐπιστημόνων καὶ ἐρευνητῶν. Τώρα τελειώνουν αὐτὲς οἱ πονηρὲς ἱστορίες στὴ βάση τοῦ πραγματικοῦ δικαίου καὶ τῆς ἀλήθειας.
\item Τὸ εἴκοσι-πέντε τοὶς ἑκατὸ κάθε Δήμου πάνω ἀπὸ 65 χρόνια ἀποτελεῖ τὴν δημογεροντία, τὸ θεσμὸ τῶν γερόντων. Αὐτὸς ὁ θεσμὸς ἀναλαμβάνει διάφορες δράσεις, καὶ ἡ βασική του εἶναι ἡ ψήφιση κατὰ πλειοψηφίαν τῶν ἐπιδοτήσεων, ἐνισχύσεων ἐπιστημόνων, ἐρευνητῶν, παραγωγῶν κλπ. Ἐποπτεύεται ἀπὸ τὸν εἰσαγγελέα τοῦ Δήμου καὶ ἐποπτεύει ὅλες αὐτὲς τὶς ἀναπτυξιακὲς δράσεις ἐκ τοῦ σύνεγγυς.
\item Κάθε ἐπιστήμων ποὺ τελειώνει ἕνα πολυτεχνεῖο, μία σχολὴ κλπ Ἀκόμη τὰ πανεπιστήμια εἶναι ὑποχρεωμένα στὴ δημιουργία ὅλων τῶν ἐπαγγελματικῶν προγραμμάτων τῶν ἐπιστημόνων ὥστε ὁ καθένας θὰ ἔχει πρόσβαση στὴ γνώση ἀλλὰ καὶ νὰ εἶναι δωρεὰν βοήθεια σὲ κάθε ἐρευνητικὴ ἢ παραγωγικὴ διαδικασία.
\item \textbf{Επανεξετάζονται όλα τα πτυχία των πολιτικών και των περί αυτών  από Παπανδρέου και εντεύθεν, με επιτροπές κατά νομό ειδικών που καθορίζει ο Δήμος έπειτα από γραπτές εξετάσεις}.
\item Θρησκεῖες ὅπως ὁ ἑβραϊσμός, προτεσταντισμός, παπισμός, μωαμεθανισμὸς παύουν νὰ εἶναι δημοσίου δικαίου. 
\item Ἀπαγορεύεται ἡ οὐνία. 
\item Οἱ μεγάλες ἑορτὲς τῆς Ὀρθοδοξίας σημαίνουν ἀποχὴ ἐργασίας ἐκτὸς τῶν ἀγροτικῶν καὶ κτηνοτροφικῶν καὶ εἰδικῆς ἀνάγκης ἐργασιῶν.
\item Ἀπαγορεύονται τὰ καζίνο τὰ σεξουαλικοῦ τύπου μαγαζιὰ ὅπου χορεύουν γυναῖκες γυμνές, τὰ προκλητικὰ μαγαζιά, ἀπαγορεύεται ἡ μαστροπεία, ἡ σεξουαλικὴ ἐκμετάλλευση γυναικῶν, τὰ μπουρδέλα. Οἱ μαστοποὶ νὰ ἐκτελοῦνται στὸ Σύνταγμα εἰς γνῶσιν ὅσων θέλουν νὰ πράξουν τὰ ἴδια. Ὅσον ἀφορὰ τὴν πορνεία, τοῦτο εἶναι προσωπικὸ ζήτημα καὶ ἰδιωτικό. Μὲ αὐτὸ τὸ καινούργιο σύστημα δὲν χρειάζονται οἱ δυστυχεῖς γυναῖκες νὰ ἐκπορνεύονται γιὰ τὸν τὸ πρὸς τὸ ζεῖν. Βρίσκουν εὔκολα ἐργασία, ἀλλὰ καὶ οἱ Δῆμοι ὑποχρεοῦνται νὰ τὶς στηρίξουν.
\item Σύζυγοι οἱ ὁποῖοι ἀπατοῦν τὶς/τοὺς συζύγους τους νὰ χάνουν κινητὴ καὶ ἀκίνητη περιουσία, ἐκτὸς ἐξαιρέσεων στὶς ὁποῖος μπορεῖ νὰ δεῖ ἕνα ὁρκωτὸ δικαστήριο.

\item  \textbf{ΠΟΛΙΤΙΚΟΙ}

\item Κατάσχεση κινητής και ακινήτου περιουσίας και δια βίου αποχή πολιτικών δικαιωμάτων όσων υπέγραψαν-ψήφισαν τα μνημόνια, όσων τα προέβαλαν και προέτρεψαν να πραγματοποιηθούν.
\item Κάθε Δήμος αποστέλλει 10 βουλευτές στη βουλή μόνο για ένα χρόνο. Εκεί δημιουργούν ομάδες ανάλογα με τα θέματα που υπάρχουν. Κάθε βουλευτής παίρνει μαζί του μέχρι 5 υπαλλήλους του για τη βουλή. Αυτοί οι υπάλληλοι είναι αιρετοί και ψηφίζονται και πληρώνονται από τους Δήμους. Αποτελούν τα εργαλεία των βουλευτών.
\item Η Βουλή διαθέτει μόνιμο προσωπικό το οποίο ενημερώνει τους νέους υπαλλήλους για τις υποχρεώσεις τους και για τα αρχεία. Το προσωπικό ελέγχεται από τους εισαγγελείς της χώρας.
\item Ψήφισμα μνημονίων είναι ίδιο με την υπογραφή τους.
\item Η ελευθερία του λόγου επιβάλλεται και ο καθένας υποχρεούται και επιβάλλεται να συμμετέχει γράφοντας στην εφημερίδα του Δήμου. Στην εφημερίδα του Δήμου που είναι υποχρεωμένος κάθε Δήμος να έχει, κάθε πολιτισμικός φορέας εκτός των ξένων πολιτισμικών φορέων, μόνον Ελληνικοί επιτρέπονται και κάθε φυσικό πρόσωπο μπορεί να γράφει τις απόψεις του και να τις βλέπει δημοσιευμένες ΧΩΡΙΣ ΚΑΜΜΙΑ ΑΛΛΑΓΗ Ή ΔΙΟΡΘΩΣΗ, όπως και το ίδιο γίνεται με τα κανάλια των Δήμων. Οτιδήποτε γράφει κανείς στην εφημερίδα του Δήμου, ΔΕΝ ΔΙΩΚΕΤΑΙ ΟΥΤΕ ΠΟΤΕ ΚΩΛΙΣΙΕΡΓΕΙΤΑΙ ΝΑ ΓΡΑΨΕΙ ΑΣΧΕΤΑ ΠΟΙΟΥΣ ΕΝΟΧΛΕΙ. Η ΕΛΕΥΘΕΡΙΑ ΤΟΥ ΛΟΓΟΥ ΓΡΑΠΤΟΥ ΚΑΙ ΔΗΜΟΣΙΟΥ ΕΠΙΒΑΛΛΕΤΑΙ.
\item Καταργοῦνται τὰ εἰδικὰ προνόμια τῶν Ἑβραίων ὅπως τὸ διπλὸ ΦΕΚ τοῦ 44, τὸ ΚΙΣ δημόσιου συμφέροντος καὶ ἄλλα θὰ βρεθοῦν στὴν πορεία.
\item Οἱ δημοσιογράφοι ψηφίζονται ἀπὸ τὴν τοπικὴ κοινωνία κάθε ἕνα ἔτος.
\item Παύει ἡ ὕπαρξη τῶν κομμάτων. Μόνο φυσικὰ πρόσωπα.
\item Καταργοῦνται οἱ συλλογικὲς συμβάσεις. Τώρα ὁ καθένας μπορεῖ εὔκολα, γρήγορα, οἰκονομικὰ νὰ κάνει ὅτι ἐργασία δύναται. Παύει ἡ ἀπάτη τῶν ἐφοριῶν μὲ τοὺς κωδικοὺς ἐπαγγελμάτων. Ἡ πιστοποίηση ἱκανότητος ἐργασίας ἐνεργεῖται ἀπὸ τὴν ἐπιτροπὴ τοῦ Δήμου καὶ γιὰ τὸ Δῆμο μόνο. Αὐτὴ ἐγγράφεται σὲ εἰδικὸ βιβλίο πιστοποιήσεων μὲ ἀντιστοίχως τὶς ὑπογραφὲς τῶν μελῶν τῆς ἐπιτροπῆς. Ἡ ἐπιτροπὴ αὐτὴ προεδρεύεται ἀπὸ τὴν εἰσαγγελία τοῦ Δήμου, τοῦ διευθυντοῦ τῆς ἀστυνομίας καὶ πανεπιστημιακὴ ὁμάδα εἰδικῶν ἐκλογῆς τοῦ συμβουλίου τῶν 7. Ἡ ὁμάδα αὐτὴ ἀποζημιώνεται ὡριαία ἀπὸ τοὺς πιστοποιούμενους ἢ σὲ περίπτωση ἀπορίας ἀπὸ τὸ Δῆμο.
\item Κάθε βουλευτὴς ψηφίζει δημόσια, δὲν ὑπάρχει παρών, μόνο ναὶ ἢ ὄχι.
\item Άμεσος και υποχρεωτικός Λογιστικός Έλεγχος του Χρέους.
\item Ὁ θεσμὸς τῶν γερόντων πέρα ἀπὸ τὶς χρηματοδοτήσεις βασικὸ στόχο ἔχουν τὴν ἀπὸ κοινοῦ δημιουργία νέου συντάγματος στὴ βάση τοῦ Συντάγματος τῶν 1911 καὶ τοῦ 1953, τοῦ 1975, τὸ δὲ νέο Σύνταγμα θὰ εἶναι βασισμένο στὰ θέσφατα, ἐπειδὴ ὅλος ὁ σκοπὸς εἶναι ἡ δημιουργία κατὰ τὸ δυνατὸν ἰδανικῆς πολιτείας στὴ βάση τῆς ἀμέσου Δημοκρατίας. Ἔτσι θὰ μπορέσουμε νὰ διώξουμε τοὺς ἐχθρούς μας ποὺ φέρουν τὴν ἀνεργία καὶ τὴν κυριαρχία τους καὶ τὴν καταστροφή μας.
\item Ἀπαγορεύεται ἐπὶ ποινῆς κατασχέσεως κινητῆς καὶ ἀκινήτου περιουσίας ἡ ἐκλογὴ σὲ δημόσια θέση αἱρετοῦ πάνω ἀπὸ 3 ἐν συνόλῳ φορές.
\item Ὁποιοσδήποτε πολιτικὸς ἀνεβαίνει στὴν ἐξουσία μὲ σκοπὸ νὰ βολευτεῖ ἢ νὰ βολέψει τοὺς δικούς του νὰ χάνει κινητὴ καὶ ἀκίνητο περιουσία.
\item Ἀναγορεύεται ἡ συνδιάλεξις καὶ συνδιαλαγὴ αἱρετῶν καὶ διορισμένων ἀρχόντων σὲ ἐπίπεδο ἑνώσεως.
\item Ἀπαγορεύεται οἱαδήποτε σχέσις αἱρετῶν ἀρχόντων καὶ Ἑλλήνων γενικά, μὲ τοὺς ρότσιλν ἢ ἐκπροσώπους ἢ ὀργάνων των.

\item  \textbf{ΥΓΕΙΑ}

\item Ἀπαγορεύονται τὰ λεγόμενα τεχνητὰ χρώματα καὶ οἱ γλυκαντικὲς ὕλες στὰ τρόφιμα.
\item Ἡ ζάχαρη ποὺ κυκλοφορεῖ παύει νὰ εἶναι χημική. Κυκλοφορεῖ μόνο ἀκατέργαστη συσκευασίες κενοῦ καὶ κυρίως φρουκτόζη. Ἀπαγορεύονται χημικὰ γλυκαντικὰ καὶ ἰδιαίτερα ἡ δημιουργία τοῦ σατανὰ ἡ ἀσπαρτάμη.
\item Ἀπαγορεύεται ο Codex Allimentarious, Ατζέντα 21 και οποιαδήποτε παρόμοια ή αντίστοιχης εμπνεύσεως \textbf{κλαπαρχιδιά σκεφτήκανε οι έρρωστοι εγκέφαλοι των γνωστών αλληταράδων λεγομένης της νέας τάξεως, δηλαδή των χιτλερικών με τους εβραίους.}
\item Ἀπαγορεύεται ὁ ψεκασμὸς τῶν ἀλεύρων τὴν ὥρα τῆς ἀλέσεώς των, ἀντ᾿ αὐτοῦ ἐπιβάλλεται τὴ συσκευασία κενοῦ.
\item Ἀπαγορεύεται ἡ χρήση χημικῶν λιπασμάτων, καθότι τὸ κίνημα διαθέτει πατέντα μετατροπῆς τῶν κατεστραμένων ἐδαφῶν τῶν σὲ βιολογικὰ καὶ μάλιστα ἄμεσα.
\item Ἱδρύεται Ἐθνικὸς Ὀργανισμὸς φαρμάκων. Ἀπαγορεύονται τὰ ἰδιωτικὰ φάρμακα. Στὸν ὀργανισμὸ συμμετέχουν πανεπιστήμια.
\item Ἱδρύεται ἰνστιτοῦτο συχνοτήτων γιὰ θεραπεία(zapper) καὶ ἀκουστικῶν ἢ μὴ συχνοτήτων(vibrate).
\item Τὰ νοσοκομεῖα δὲν εἶναι δωρεὰν ἄλλα κρατικά.
\item Οι ναρκωμανείς παίρνουν τη δόση τους στο νοσοκομείο.
\item Δημιουργεῖται ὑπὸ τῆς κυβερνήσεως καὶ τὴν ἐποπτεία τῶν εἰσαγγελέων καὶ Δήμου Ἀθηναίων τὸν ἐρευνητικὸ κέντρο ἀποκαταστάσεως ματιοῦ. Ὅλοι οἱ Ἕλληνες ὑποχρεοῦνται στὴ διόρθωση τῶν ματιῶν τῶν ἐφόσον θέλουν -σχεδὸν δωρεάν, ὥστε νὰ πάψει χρῆσις γυαλιῶν. Τὸ κέντρο αὐτὸ θὰ εἶναι παντελῶς ἀφορολόγητο καθότι μὴ κερδοσκοπικὸς ὀργανισμός, ἕνα μέρος θὰ ἀποκατασταθεῖ καὶ ξένους μὲ εἰδικὰ κριτήρια τὰ ὁποῖα καὶ θὰ χρηματοδοτοῦν τὸ κέντρο. Οἱ ἐργαζόμενοι στὴ διοίκηση εἶναι αἱρετοί.
\item Ἐπιτρέπεται καὶ ἐπιβάλλεται εἰδικὰ γιὰ τὸν καρκίνο ἡ καλλιέργεια τοῦ χασὶς γιὰ ἰατρικοὺς λόγους.
\item Οἱ ναρκομανεῖς μποροῦν νὰ πάρουν τὴ δόση τους στὰ νοσοκομεῖα ὡς ἀσθενεῖς ὄντες.
\item Ἐνισχύεται κατεπειγόντως ἀπὸ τὴν κεντρικὴ κυβέρνηση ἡ ἐκτροφὴ βουβαλιῶν καὶ τὰ σκληρὰ σιτηρὰ τύπου ΖΕΑΣ, ὥστε νὰ ξαναγίνει ἡ διατροφὴ τοῦ Ἕλληνος ὅπως στὴν ἀρχαία Ἑλλάδα. Ἡ Ἑλλάδα πρέπει νὰ ἔχει τουλάχιστον 250.000 βουβάλια.
\item Ἀπαγορεύονται τὰ χημικὰ ἀποῤῥυπαντικά. Ἔχει διαπιστωθεῖ ὅτι τὰ ἀποῤῥυπαντικὰ μὲ βάση τὴν ἐλιὰ δροῦν ἰσχυρότερα καὶ καλύτερα (καὶ κυρίως οἰκολογικά) σὲ σχέση μὲ τὰ χημικά.
\item Ἀπαγορεύεται ἡ χημικὴ διάβρωση γιὰ ἐξόρυξη μετάλλων, ὅπως σήμερα στὶς σκουριές. Ἀντικαθίσταται μὲ οἰκολογική.
\item Απαγορεύονται τα εμβόλια πάσης φύσεως. Όλες οι αρρώστιες λόγω παρασίτων θεραπεύονται με zapper και μέθοδο ταλαντώσεων.
\item Υποχρεωτική η πλήρης γνώση της παρασιτολογίας στα τμήματα ιατρικής. Οι περισσότερες αρρώστιες προέρχονατι από παράσιτα και ρετροϊούς. Αυτές δεν θαραπεύονται μέσω του ανοσοποιητικού αλλά με διάσπαση μέσω συντονισμού με zapper. Γι αυτό το λόγο και τα πανεπιστήμια ιατρικής είναι άμεσα συνδεδεμένα με τα τμήματα ηλεκτρολόγων μηχανικών και μηχανικών υπολογιστών για τη χρήση zapper στον πληθυσμό. ΤΕΛΟΣ ΚΑΤΑ 85\% ΣΤΗ ΦΑΡΜΑΚΕΥΤΙΚΗ ΠΡΟΣΕΓΓΙΣΗ ΤΩΝ ΠΟΛΥΕΘΝΙΚΩΝ.
\item ΥΠΟΧΡΕΩΤΙΚΗ Η ΕΠΑΝΑΛΕΙΤΟΥΡΓΙΑ Ή Η ΔΗΜΙΟΥΡΓΙΑ ΕΘΝΙΚΟΥ ΦΟΡΕΩΣ ΦΑΡΜΑΚΩΝ ΚΑΙ ΕΛΛΗΝΙΚΗΣ ΕΡΕΥΝΗΣ ΚΑΙ ΦΑΡΜΑΚΩΝ.

\item \textbf{ΑΜΥΝΑ}

\item Ἡ ἀεροπορία εἶναι ὑποχρεωμένη νὰ ρίχνει ὁτιδήποτε κυκλοφορεῖ στὸν ἐναέριο χῶρο τῆς Ἑλλάδος, σὲ ὁποιαδήποτε ὕψος, τὸ ὁποῖο ψεκάζει ἢ παρακολουθεῖ, ἀκόμη καὶ δορυφόρο χωρὶς καμιὰ προειδοποίηση.
\item Υποχρεωτική η παραγωγή όπλων στην Ελλάδα. Εισαγωγή μόνο με πανελλαδικό ορκωτό δικαστήριο με 10 ενόρκους από κάθε νομό.
\item Υποχρεωτική η αριστεία.

\end{itemize}

\textbf{ΦΙΛΟΙ ΤΟΥ ΚΙΝΗΜΑΤΟΣ}\\Για την νομιμοποίηση του κινήματος χρειαζόμαστε φίλοι και μέλη να υπερβαίνουν τα 5 εκατομμύρια πολίτες. Μπορεί να γίνει κάποιος φίλος με υπεύθυνη δήλωση με το γνήσιο της υπογραφής. Οι φίλοι δεν έχουν \textbf{καμμία απολύτως υποχρέωση}.


\textbf{{\Large ΣΕ ΟΛΑ ΤΑ ΑΝΩΤΕΡΩ ΣΕ ΠΕΡΙΠΤΩΣΗ ΥΠΕΡΚΑΛΥΨΕΩΣ ΙΣΧΥΟΥΝ ΤΑ ΒΑΡΥΤΕΡΑ ΑΠΟ ΤΗΝ ΑΠΟΨΗ ΤΩΝ ΠΟΙΝΩΝ.}}

ΚΑΙ ΕΝΑ ΚΕΙΜΕΝΟ ΓΙΑ ΤΗ ΓΛΩΣΣΑ ΚΑΙ ΓΙΑ ΠΟΙΟ ΛΟΓΟ ΘΑ ΚΑΤΑΡΓΗΘΕΙ ΤΟ ΥΠΟΥΡΓΕΙΟ ΠΑΙΔΕΙΑΣ ΚΑΙ ΤΟ ΑΧΡΕΙΟ ΠΑΙΔΑΓΩΓΙΚΟ ΙΝΣΤΙΤΟΥΤΟ
\verbatimfont{}
\begin{verbatim}
Κατ’ αρχάς να σου πω ότι εδώ στην Γαλλία οι Γάλλοι τώρα έχουν αρχίσει να αντιλαμβάνονται ότι έχουν χάσει το παιχνίδι με την κυριαρχία της γλώσσας και αρχίζουν να μαθαίνουν αγγλικά. Το όνειρο κάθε νέου είναι η Αμερική και όλων των εφήβων τα υπνοδωμάτια είναι διακοσμημένα με εικόνες από την Νέα Υόρκη. Επίσης στην γλώσσα έχουν περάσει αγγλικές λέξεις και επίσης σε λογότυπα επιχειρήσεων γίνονται λογοπαίγνια με αγγλικά και γαλλικά συγχρόνως. Στα Δημοτικά υποχρεώνουν τους δασκάλους να διδάσκουν αγγλικά χωρίς να είναι γνώστες της γλώσσας. Μην φανταστείς ότι κάνουν τίποτα φοβερό, μαθαίνουν τα παιδιά κάτι λέξεις κι αυτό το βάφτισαν μάθημα αγγλικών. Καταλαβαίνεις το επίπεδο. Δεν έχουν διορίσει καθηγητές αγγλικών, γιατί βέβαια δεν έχουν. Όσον αφορά εμάς είναι βέβαιο ότι όλοι οι πολιτικοί μας ήταν πουλημένα τομάρια. Η γλώσσα μας σε σχέση με τις άλλες ανοίγει δρόμους στον εγκέφαλο και μας κάνει εξυπνότερους και εφευρετικότερους. Αυτό δεν αρέσει στους ξένους οι οποίοι επιθυμούν να συνδιαλέγονται με κατώτερους, άρα πρέπει να χτυπηθεί η γλώσσα. Τώρα πια μπορώ να το πω. Συλλαβίζουν μαθητές της Γ’ Λυκείου σε καλά σχολεία και όχι στα σχολεία των περιχώρων με έντονο στοιχείο μεταναστών. Καταλαβαίνεις τι θα γίνεται εκεί. Δεν σου μιλώ για ορθογραφία. Εκεί να δεις το χάος. Έχει πολλούς κανόνες η γαλλική και δεν είναι αρκετά όσα κάνουν οι συνάδελφοι στο σχολείο. Εδώ δεν υπάρχει η έννοια της επανάληψης στο σπίτι ούτε της βοήθειας από τους γονείς ούτε του σχολικού βιβλίου. Τα βιβλία επιλέγονται από τους συναδέλφους από ό,τι υπάρχει στο εμπόριο, τα αγοράζει το σχολείο και τα παραχωρεί στους μαθητές, που είναι υποχρεωμένοι να τα παραδώσουν στο τέλος της χρονιάς στο σχολείο. Έτσι, για να μην καταστραφούν τα βιβλία και υποχρεωθεί το σχολείο να τα ξαναγοράσει, οι μαθητές δεν παίρνουν τα βιβλία στο σπίτι το απόγευμα. Άρα:
1. Ο δάσκαλος γράφει στον πίνακα και οι μαθητές αντιγράφουν μια περίληψη σε κάθε μάθημα και αυτήν πρέπει να μάθε ο μαθητής
2. Αν κάποιος μαθητής ήθελε να έχει πρόσβαση στο βιβλίο, δεν μπορεί και 
3. Δεν εξασκούν την μνήμη τους μαθαίνοντας απ’ έξω π.χ. Ιστορία

Όλα από την γλώσσα ξεκινάνε. Θέλουν λαούς κωθώνια για να τους κάνουν ό,τι θέλουν και όχι σκεπτόμενα άτομα. Το μόνο που ενδιαφέρει το μέσο πολίτη εδώ είναι το πού θα πάει διακοπές. Το έχουν βέβαια ανάγκη λόγω κλίματος, αλλά από κει και πέρα χαίρε βάθος αμέτρητο. Θα πρέπει να βρεθείς σε κύκλο υψηλού επιπέδου για να κουβεντιάσεις θέματα κοινά. Οι λαοί χάσαμε την αγωνιστικότητά μας βυθισμένοι στο κυνήγι της απόλαυσης και του κατέχειν.
Θεωρούμε εκσυγχρονισμό την καθημερινή μας συνεννόηση σε αμερικανο-ελληνικά ή γαλλο-ελληνικά κλπ, διεκδικώντας παγκόσμια πρωτοτυπία. 
 Ποιος άλλος λαός (του οποίου η γλώσσα διδάσκεται σε σχολεία και πανεπιστήμια του εξωτερικού) προσπαθεί να εξαφανίσει τον γλωσσικό του πλούτο; 
 Και σ΄ αυτό (δυστυχώς) είμαστε μοναδικοί . . . Προτιμούμε, για παράδειγμα, να λέμε (και το χειρότερο, να γράφουμε!): 
\end{verbatim}
\begin{multicols}{3}
 Πρεστίζ αντί κύρος \\
 Καριέρα αντί σταδιοδρομία \\
 Ίματζ αντί εικόνα \\
 Σαμποτάζ αντί δολιοφθορά \\
 Γκλάμουρ αντί αίγλη / λάμψη \\ 
 Μίτινγ αντί συνάντηση \\
 Κουλ αντί ψύχραιμος \\
 Τιμ αντί ομάδα \\
 Πρότζεκτ αντί σχέδιο / μελέτη \\
 Ντιζάιν αντί σχέδιο \\
 Τάιμινγκ αντί συγχρονισμός \\
 Σπόνσορα αντί χρηματοδότης / χορηγός \\
 Τσεκάρω αντί ελέγχω \\
 Μοντάρω αντί συναρμολογώ \\
 Ρισκάρω αντί διακινδυνεύω \\
 Σνομπάρω αντί περιφρονώ / υποτιμώ 
\end{multicols}
\begin{verbatim}
 Υπάρχουν εκατοντάδες παραδείγματα σαν κι αυτά. Τόσες υπέροχες ελληνικές λέξεις, μένουν στα γλωσσικά ερμάρια (ντουλάπια), γιατί ξεχάσαμε (κάποιοι ίσως δεν έμαθαν ποτέ), πώς και πού να τις χρησιμοποιούμε για να αποδώσουν τα
νοήματα όπως πρέπει . . . 
 Δυστυχώς το αρμόδιο Ελληνικό Υπουργείο, μαζί με το "Παιδαγωγικό Ινστιτούτο", όχι μόνο δεν ενδιαφέρονται για τη διατήρηση, με την κατάλληλη διδασκαλία της Ελληνικής γλώσσας, αλλά κάνουν τα πάντα για την διάβρωση της. 
 ΤΙ ΚΑΝΟΥΝ ΑΛΛΑ ΚΡΑΤΗ ΓΙΑ ΤΗΝ ΠΡΟΣΤΑΣΙΑ ΤΗΣ ΓΛΩΣΣΑΣ TΟΥΣ και τι κάνει η Ελλάδα... 
Η Ρωσία με διάταγμα του Προέδρου Πούτιν, το 2003, προβλέπει επιβολή αυστηρότατων προστίμων, σε όποιον τυχόν επιχειρήσει να χρησιμοποιήσει το Λατινικό αντί του Κυριλλικού αλφαβήτου.
Στην Ιαπωνία και Κορέα απαγορεύουν όχι μόνο την οποιαδήποτε αλλοίωση του αλφαβήτου τους, αλλά και την χρήση ξενόγλωσσων επιγραφών.
Η Κίνα ίδρυσε ειδική Υπηρεσία των Υπουργείων Παιδείας και Πολιτισμού, που επιβάλλει στους διαφημιστές να μην χρησιμοποιούν λογοπαίγνια που διαφθείρουν την γλώσσα. 
Η Ισπανία απείλησε με αποχώρηση από την Ευρωπαϊκή Ένωση, αν η ΕΕ επέμενε στην υπόδειξη για την κατάργηση του
 /"τήλδε",/ δηλαδή του τονικού τους σημείου, που μοιάζει με την δική μας (πρώην) περισπωμένη. 
Η Γαλλία ήδη προχωρά σε απο-αγγλισμό της γλώσσας της, διατηρεί μόνο της προερχόμενες από την Ελληνική γλώσσα και επιβάλλει διοικητικές κυρώσεις κατά των τυχόν χρησιμοποιούντων αγγλικούς κυρίως τίτλους σε πινακίδες, διαφημίσεις κ.λ.π. 
Το Ισραήλ ίδρυσε από το 1948, την Ακαδημία της Εβραϊκής γλώσσας και κατάφερε το1948 μετά από ….4.000!!! χρόνια να επαναφέρει και επιβάλει τη χρήση της αρχαίας - μέχρι τότε νεκρής - Εβραϊκής γλώσσας.
 - Για να εμπλουτίσουν το λεξιλόγιο τους με τις ονομασίες νέων φυτών, πραγμάτων, επιστημονικών όρων και δραστηριοτήτων, που δεν υπήρχαν στην αρχαία τους γλώσσα, δημιούργησαν νέες (εβραιογενείς) λέξεις. 
- Κατέστησαν την ανανεωμένη, εμπλουτισμένη και ενισχυμένη αυτή γλώσσα τους, ενιαίο γλωσσικό όργανο εκπαιδεύσεως, διοικήσεως, αλλά και συγγραφής βιβλίων και δημοσιογραφίας, με αυστηρότατους ελέγχους και διαρκή εποπτεία, επιμελούμενοι την ορθή χρήση της. 
Ενώ στην Ε Λ Λ Α Δ Α oι επεμβάσεις στην γλώσσα μας τα τελευταία 35 χρόνια. 
 α) Αν και δεν υλοποιήθηκε, αξίζει να καταγραφεί... Λέγεται ότι ο Κωνσταντίνος Καραμανλής, ως Πρωθυπουργός το 1975, είχε καλέσει στο γραφείο του, τον Υπουργό Πολιτισμού Κ.Τσάτσο και τον επικεφαλής του Παιδαγωγικού Ινστιτούτου Ευάγγ. Παπανούτσο και , μεταξύ άλλων, τους είπε ότι έπρεπε κάποια στιγμή να σκεφθούν το ενδεχόμενο να συνδυαστεί το Ελληνικό αλφάβητο με το λατινικό και να εξετασθεί και το θέμα της φωνητικής γραφής!!!! Τότε, όπως διέρρευσε αργότερα, σαν ελατήρια πετάχθηκαν επάνω οι δύο συνομιλητές του και του δήλωσαν, χωρίς περιστροφές , ότι παραιτούνται. Ο Καραμανλής απέσυρε το θέμα και ο Κ. Τσάτσος , αργότερα, αφηγήθηκε : "Δεν πίστευα στα αυτιά μου"... 
 β) Το 1976 ο Υπουργός Παιδείας Γεώργιος Ράλλης, με εγκύκλιό του, αυθαίρετα και αντιδημοκρατικά, καταργεί την καθομιλουμένη απλή καθαρεύουσα και επιβάλλει την Δημοτική. Δεν αρκείται, όμως, σε αυτό και εισάγει την παράλληλη διδασκαλία της Αρχαίας Ελληνικής γλώσσας με μετάφραση των διδασκομένων κειμένων εις την Δημοτική. Αργότερα, δηλώνει ότι αυτό ήταν το μεγαλύτερο και σοβαρότερο λάθος της πολιτικής του καριέρας. 
 γ) Το 1981 επιβάλλεται από την Κυβέρνηση Ανδρέα Παπανδρέου με Υπουργό Παιδείας τον Ελευθέριο Βερυβάκη, το μονοτονικό, με τροπολογία, που αιφνιδίως προτάθηκε κοντά στα μεσάνυχτα, όταν η Βουλή των Ελλήνων, κατά την συνεδρίαση της 11-1-82 είχε περατώσει την συζήτηση και είχε ψηφίσει το ένα και μοναδικό άρθρο του Νόμου 1228, που αποτελούσε κύρωση της από 11-11-1981 πράξεως του Προέδρου της Δημοκρατίας "Περί εγγραφής μαθητών στα Λύκεια της Γενικής και επαγγελματικής Εκπαιδεύσεως" (με ακριβώς αυτόν τον τίτλο δημοσιεύτηκε στο Α τεύχος της Εφημερίδος της Κυβερνήσεως). Από πόσους Βουλευτές ψηφίστηκε η τροπολογία αυτή; Κατά δήλωση του αείμνηστου Παν. Κανελλόπουλου, ο οποίος δεν ήταν παρών στην συνεδρίαση, αφού ουδείς γνώριζε πως η Κυβέρνηση θα εισήγαγε αιφνιδίως, προσκολλημένο σε άσχετο, επουσιώδη νόμο, τέτοιο μέγιστο εθνικό θέμα για συζήτηση, ψηφίστηκε από όχι από περισσότερους από τριάντα (30) βουλευτές... Αυτό επιβεβαιώνει και ο ποιητής και ακαδημαϊκός Νικηφόρος Βρεττάκος (στο ΕΘΝΟΣ την 8-5-1990). Παρατίθενται και οι αιτιολογίες της τροπολογίας. 1- Οι αρχαίοι έλληνες δεν χρησιμοποιούσαν στην γραφή τους τόνους και πνεύματα. 2- Οι μαθητές αφιερώνουν για την εκμάθηση τόνων και πνευμάτων περισσότερες από 6.000 διδακτικές ώρες. 3- Με την εφαρμογή του μονοτονικού θα προκύψει μείωση των δαπανών εκτυπώσεως και εκδόσεως των εφημερίδων και των βιβλίων, κατά 30\%. Όσον αφορά την πρώτη αιτιολογία είναι σωστή διότι οι αρχαίοι Έλληνες χρησιμοποιούσαν την μεγαλογράμματη γραφή, τα κεφαλαία. Όσον αφορά την δεύτερη αιτιολογία, στερείται σοβαρότητος, επειδή οι 6.000 ώρες διδασκαλίας (κατά τον εισηγητή), χρειάζονται 27,5 χρόνια διδασκαλίας! Η τρίτη αιτιολογία είναι βάσιμη, όσο και ειδεχθής. Οι μεγαλοεκδότες και οι παντοδύναμοι δημοσιογραφικοί οργανισμοί (τέταρτη εξουσία ή πρώτη;), αυξάνουν αισθητά τα κέρδη τους. 
\end{verbatim}

\textbf{ΕΝΤΑΞΗ ΣΤΟ ΚΙΝΗΜΑ}\\Για να ενταχθεί κάποιος στο κίνημα, κόμμα κλπ ως μέλος είναι απαραίτητο να κάνει τα εξής:

1. Καταθέτει το παρακάτω κείμενο στην πλησιέστερη στην οικία του Εισαγγελική αρχή.
\textbf{ΑΙΤΗΜΑ – ΔΗΛΩΣΗ ΠΡΟΣ ΤΗΝ ΕΙΣΑΓΓΕΛΙΚΗ ΑΡΧΗ} Κε εισαγγελεύ. Εμείς ως Ελληνικός λαός δεν αναγνωρίζουμε την κυβέρνηση του Σύριζα --ως και τις αποφάσεις των προηγουμένων -- ως νόμιμες Κυβερνήσεις, δεν απηχούν τη θέλησή μας και σας ζητούμε την σύλληψη όλων \textbf{των δωσιλόγων} -όσων έχουν υπογράψει ή φηφίσει τα μνημόνια- επί εσχάτη προδοσία. Παρακαλούμε κε εισαγγελεύ με βάση το άρθρο 4 του Συντάγματος επειδή όλοι οι Έλληνες πολίτες είμαστε ίσοι \textbf{να συλλάβετε επί εσχάτη προδοσία} όλους όσους \textbf{ψηφίσανε τα μνημόνια}, εκτός \textbf{κι αν θεωρείτε ότι όσοι είναι εβραίοι δεν υπόκεινται σε σύλληψη}. Μέχρι να γίνει αυτό σας δηλώνουμε ότι εφαρμόζοντας \textbf{το άρθρο 120 του Συντάγματος} απέχουμε και \textbf{δεν νομιμοποιούμε} οποιονδήποτε νόμο και αξίωση των πολιτικών και των συνεργών τους. Είναι σαφές ότι δεν τους δώσαμε καμία εξουσιοδότηση για όσα έχουνε ψηφίσει, ενώ είναι απόλυτα γνωστό ότι μας έχουν εξαπατήσει και θα έπρεπε ήδη να βρίσκονται πίσω από της φυλακής τα κάγκελα.
Μέχρι να επέλθη δικαιοσύνη δεν αναγνωρίζουμε καμία τους νομοθετική πράξη. Ο/η καταγγέλλων: ΟΝΟΜΑΤΕΠΩΝΥΜΟ ΔΙΕΥΘΥΝΣΗ Α.Δ. ΤΑΥΤΟΤΗΤΑΣ

2. Στη συνέχεια \textbf{καταθέτει το αίτημα-δικόγραφο} των 7000 περίπου σελίδων \textbf{στη ΔΟΥ}(τις πρώτες 46 σελίδες και το DVD με τα στοιχεία) που ανήκει -παύοντας πλέον να καταβάλλει οιαδήποτε απαιτήσεις των εφοριών, ταμείων κλπ-, παίρνοντας αντίγραφο με αριθμό πρωτοκόλλου, το φωτοτυπεί και το καταθέτει στην εισαγγελία της περιοχής προς τους αξιοτίμους κους εισαγγελείς προκειμένου να προβούν στα νόμιμα, παίρνει αντίγραφο επικυρωμένο το οποίο στη συνέχεια φωτοτυπεί και το καταθέτει και στο τοπικό Αστυνομικό Τμήμα ώστε να είναι ενήμεροι της απάτης και της προδοσίας που έχει στηθεί.

3. Πηγαίνει σε ένα \textbf{συμβολαιογράφο και καταθέτει την Συμβολαιογραφική Πράξη Καλλιελαίου Κινήματος} μαζί με τα δύο προηγούμενα έγγραφα και πλέον είναι μέλος της ομάδος του κινήματος. Τέλος αποστέλλει αντίγραφο της πρώτης σελίδος των καταθέσεων σε οποιονδήποτε γνωρίζει ότι έχει κάνει το ίδιο. Στις 200 καταθέσεις θα δημιουργηθεί κόμμα με καταστατικό όπου δεν υπάρχει πρόεδρος κλπ, παρά μόνο επιτροπή που αλλάξει τυχαία κάθε μήνα, ενώ άμεσα θα δημιουργηθεί και η πρώτη εφημερίδα προκειμένου να ενημερωθούν οι Έλληνες για το Αμεσοδημοκρατικό Κίνημα, την ανάγκη του και την απάτη που μας στήσανε οι εβραιομασσώνοι με του σιωνιστές και τους χιτλερικούς. Βασικό χαρακτηριστικό του καταστατικού θα είναι η \textbf{μηνιαία διαδικτυακή ή φυσική αν υπάρχει η δυνατότητα γενική συνέλευση μελών} και ο \textbf{πλήρη απολογισμός} των αιρετών και η αλλαγή τους όχι με ψήφο αλλά με τυχαία επιλογή. Δηλαδή \textbf{κάθε μήνα αλλάζει η σύσταση} των εκπροσώπων του φορέα \textbf{με τυχαία πρόσωπα}.

4. Παύει υποχρεωτικά να πληρώνει εφορίες, ΦΠΑ, κάθε είδους τέλος, να νομιμοποιεί πληρώνοντας διόδια και οτιδήποτε υπάρχει προς είσπραξη προς το παρα-κράτος που έχει δημιουργηθεί.

Υπ όψιν προς όλους ότι απαγορεύονται οι συγκεντρώσεις για το κίνημα και ο λόγος είναι προφανής. Όλα ελέγχονται και κατευθύνονται από τις μυστικές υπηρεσίες(CIA, DIA, Mossad, MI5 κλπ), οπότε δεν πρέπει να δίνουμε αφορμές για προβοκάτσιες και άλλα παρόμοια. Η λύση είναι απλή. Υπάρχουν οι εισαγγελίες και η αστυνομία στις οποίες κάθε φορά θα απευθυνόμαστε και κυρίως θα είμαστε παντελώς άοπλοι. Θα ενεργήσουμε όπως έκανε στην Ινδία ο Γκάντι. Είναι δικαίωμά μας η ελευθερία μας και δεν την παραδίδουμε για καμμία ψευτο-ψήφο  που έκανε ο κάθε δωσίλογος προδότης που πήρε φασιστικά δουλεύοντάς μας θέση στη βουλή και με τη συνέργεια των εχθρών μας και χωρίς τη θέλησή μας.

\textbf{Επίσης προσυπογράφω τα κάτωθι.}

Προς κάθε Δημόσια Αρχή Με ατομική μου ευθύνη και γνωρίζοντας τις κυρώσεις, που προβλέπονται από τις διατάξεις της παρ. 6 του άρθρου 22 του Ν. 1599/1986, δηλώνω ότι: Η παρούσα δήλωσή γίνεται εις γνώση μου και με τη θέλησή μου, ενώ αποδέχομαι από σήμερα και στο διηνεκές (είναι διαρκής και απαρέγκλιτη) να εκπέσω αυτομάτως και αμέσως από οποιαδήποτε θέση ή αξίωμα, εάν την παραβιάσω, είτε πρόκειται να συμπεριληφθώ στα μέλη του πολιτικού φορέα, είτε πρόκειται να διεκδικήσω θεσμική θέση στα όργανα του, είτε πρόκειται να είμαι υποψήφιος/α του εκλογικού συνδυασμού του πολιτικού φορέα που θα ενταχθεί το κίνημα ή στον πολιτικό φορέα που θα δημιουργήσει, όποτε ο φορέας συμμετέχει σε εκλογικές διαδικασίες και δηλώνω υπεύθυνα ότι:

– Μελέτησα με πολύ προσοχή την τα ανωτέρω και δηλώνω ότι πλήρως συντάσσομαι και αποδέχομαι όλα τα ανωτέρω και τα συνυπογράφω ανεπιφύλακτα.

– Δηλώνω δε, ότι έχω καθαρό ποινικό μητρώο, ούτε και εκκρεμεί στις αρχές καταγγελία, ή ποινική δίκη. 

– Ουδέποτε και με τον οποιοδήποτε άμεσο ή έμμεσο τρόπο, ΔΕΝ έχω καταχραστεί, υπεξαιρέσει ή κλέψει χρήματα άλλων και ειδικά του Δημοσίου, ούτε προτίθεμαι να πράξω τούτο. Και αν υποπέσει στην αντίληψή μου θα ενημερώσω άμεσα και αρμοδίως. Αυτό ισχύει και για κάθε νομικό πρόσωπο που ήταν/είναι υπό την εξουσία/κυριαρχία/επίβλεψη μου, όπως π.χ. κόμμα, επιχείρηση εμφανών ή αφανών συμφερόντων μου, κ.τ.λ. – Ουδέποτε και με τον οποιοδήποτε άμεσο ή έμμεσο τρόπο, δεν έχω εν γνώση μου συμμετάσχει, αποσιωπήσει, συγκαλύψει προδοσία κατά της Πατρίδος μου, ενώ κάτι τέτοιο δεν πρόκειται να το πράξω ποτέ και ούτε να το επιτρέψω, ενώ εάν υποπέσει κάτι τέτοιο στην αντίληψη μου, τουλάχιστον θα το καταγγείλω αμέσως και αρμοδίως. Αυτό ισχύει και για κάθε νομικό πρόσωπο που ήταν/είναι υπό την εξουσία/κυριαρχία/επίβλεψη μου, όπως π.χ. κόμμα, επιχείρηση εμφανών ή αφανών συμφερόντων μου, κ.τ.λ.

– Ουδέποτε έχω παρανόμως ή και δολίως χρηματιστεί, ή έχω λάβει την οποιαδήποτε έκνομη «προμήθεια». Ουδέποτε μεσολάβησα, προκειμένου να γίνει παρανόμως ή και δολίως, η ανάληψη του οποιουδήποτε δημόσιου ή ιδιωτικού έργου και γενικά για να αποκομίσω παράνομες «ωφέλειες».

– Ουδέποτε κατακυρώθηκε σε επιχείρηση εμφανών ή αφανών συμφερόντων μου δολίως προσχεδιασμένος μειοδοτικός διαγωνισμός (ούτε και εκκρεμεί επίσημη και κατατεθειμένη στις αρχές καταγγελία, ή δίκη επ’ αυτού) 

– Ουδέποτε έλαβα χορηγία από οποιονδήποτε αμφιβόλων συμφερόντων. για την προεκλογική μου εκστρατεία (για όσους έχουν συμμετάσχει ή θα συμμετάσχουν στον πολιτικό «στίβο»).

– Ουδέποτε παρενέβην στο έργο δημοσίου λειτουργού για τη συγκάλυψη ανομήματος είτε δικού μου, είτε άλλου προσώπου.

– Ουδέποτε είχα οικονομικές εμπλοκές (π.χ. πλαστών ή εικονικών τιμολογίων), ή κλοπή, ή υπεξαίρεση δημοσίου χρήματος γενικότερα και ουδέποτε χρημάτισα εφοριακό για να «τακτοποιήσει» φορολογικές μου εκκρεμότητες.

– Δεν διαθέτω και ουδέποτε διέθετα τραπεζικούς λογαριασμούς με αδήλωτα «μαύρα» χρήματα στο εξωτερικό είτε στο όνομά μου, είτε σε όνομα συγγενικού ή έμπιστου προσώπου, ακόμη δε και στο όνομα εταιρείας εμφανών ή αφανών συμφερόντων μου.

• Όλες ανεξαιρέτως οι πηγές της κινητής και ακινήτου περιουσίας μου, στο καλύτερο των γνώσεων μου (ειδικά για κληρονομιές και τις δωρεές), έχουν όλες νόμιμη προέλευση.

• Υπηρέτησα (για τον δημόσιο τομέα και το πολιτικό προσωπικό γενικότερα) στην τάδε και δείνα θέση του ευρύτερου Δημόσιου Τομέα και από τη θέση που κατείχα/κατέχω, διαχειρίστηκα το δημόσιο χρήμα με απόλυτη διαφάνεια και σεβασμό προς τον μόχθο του Ελληνικού λαού.

Τούτο ισχύει και για τις θητείες μου ως βουλευτή ή υπουργού ή υφυπουργού (στην περίπτωση πολιτικού προσωπικού).
Με την υπογραφή μου κατωτέρω, δηλώνω υπεύθυνα και εν γνώση όλων των συνεπειών του νόμου, ότι επιθυμώ με την ελεύθερη θέληση μου να γίνω Μέλος του Κινήματος δηλώνω δε, ότι δεν στερούμαι τα πολιτικά μου δικαιώματα και μπορώ νομίμως να συμμετέχω σαν μέλος του κινήματος και του πολιτικού φορέα.

\textbf{ΓΙΑ ΤΟΥΣ ΦΙΛΟΥΣ ΤΟΥ ΚΙΝΗΜΑΤΟΣ} αρκεί μια απλή βεβαίωση με το γνήσιο της υπογραφής η δήλωση φιλίας, δηλαδή νομιμοποιήσεως του κινήματος.

{
\noindent --Ονοματεπώνυμο--ΑΔΤ--Υπογραφή--........................................................................................................................\\

\noindent --Ονοματεπώνυμο--ΑΔΤ--Υπογραφή--........................................................................................................................\\

\noindent --Ονοματεπώνυμο--ΑΔΤ--Υπογραφή--........................................................................................................................\\

\noindent --Ονοματεπώνυμο--ΑΔΤ--Υπογραφή--........................................................................................................................\\

\noindent --Ονοματεπώνυμο--ΑΔΤ--Υπογραφή--........................................................................................................................\\

\noindent --Ονοματεπώνυμο--ΑΔΤ--Υπογραφή--........................................................................................................................\\

\noindent --Ονοματεπώνυμο--ΑΔΤ--Υπογραφή--........................................................................................................................\\

\noindent --Ονοματεπώνυμο--ΑΔΤ--Υπογραφή--........................................................................................................................\\

\noindent --Ονοματεπώνυμο--ΑΔΤ--Υπογραφή--........................................................................................................................\\

\noindent --Ονοματεπώνυμο--ΑΔΤ--Υπογραφή--........................................................................................................................\\

\noindent --Ονοματεπώνυμο--ΑΔΤ--Υπογραφή--........................................................................................................................\\

\noindent --Ονοματεπώνυμο--ΑΔΤ--Υπογραφή--........................................................................................................................\\

\noindent --Ονοματεπώνυμο--ΑΔΤ--Υπογραφή--........................................................................................................................\\

\noindent --Ονοματεπώνυμο--ΑΔΤ--Υπογραφή--........................................................................................................................\\

\noindent --Ονοματεπώνυμο--ΑΔΤ--Υπογραφή--........................................................................................................................\\

\noindent --Ονοματεπώνυμο--ΑΔΤ--Υπογραφή--........................................................................................................................\\

\noindent --Ονοματεπώνυμο--ΑΔΤ--Υπογραφή--........................................................................................................................\\

\noindent --Ονοματεπώνυμο--ΑΔΤ--Υπογραφή--........................................................................................................................\\

\noindent --Ονοματεπώνυμο--ΑΔΤ--Υπογραφή--........................................................................................................................\\

\noindent --Ονοματεπώνυμο--ΑΔΤ--Υπογραφή--........................................................................................................................\\

\noindent --Ονοματεπώνυμο--ΑΔΤ--Υπογραφή--........................................................................................................................\\

\noindent --Ονοματεπώνυμο--ΑΔΤ--Υπογραφή--........................................................................................................................\\

\noindent --Ονοματεπώνυμο--ΑΔΤ--Υπογραφή--........................................................................................................................\\

\noindent --Ονοματεπώνυμο--ΑΔΤ--Υπογραφή--........................................................................................................................\\

\noindent --Ονοματεπώνυμο--ΑΔΤ--Υπογραφή--........................................................................................................................\\

\noindent --Ονοματεπώνυμο--ΑΔΤ--Υπογραφή--........................................................................................................................\\

\noindent --Ονοματεπώνυμο--ΑΔΤ--Υπογραφή--........................................................................................................................\\

\noindent --Ονοματεπώνυμο--ΑΔΤ--Υπογραφή--........................................................................................................................\\

\noindent --Ονοματεπώνυμο--ΑΔΤ--Υπογραφή--........................................................................................................................\\

\noindent --Ονοματεπώνυμο--ΑΔΤ--Υπογραφή--........................................................................................................................\\

\noindent --Ονοματεπώνυμο--ΑΔΤ--Υπογραφή--........................................................................................................................\\

\noindent --Ονοματεπώνυμο--ΑΔΤ--Υπογραφή--........................................................................................................................\\

\noindent --Ονοματεπώνυμο--ΑΔΤ--Υπογραφή--........................................................................................................................\\

\noindent --Ονοματεπώνυμο--ΑΔΤ--Υπογραφή--........................................................................................................................\\

\noindent --Ονοματεπώνυμο--ΑΔΤ--Υπογραφή--........................................................................................................................\\

\noindent --Ονοματεπώνυμο--ΑΔΤ--Υπογραφή--........................................................................................................................\\

\noindent --Ονοματεπώνυμο--ΑΔΤ--Υπογραφή--........................................................................................................................\\

\noindent --Ονοματεπώνυμο--ΑΔΤ--Υπογραφή--........................................................................................................................\\

\noindent --Ονοματεπώνυμο--ΑΔΤ--Υπογραφή--........................................................................................................................\\

\noindent --Ονοματεπώνυμο--ΑΔΤ--Υπογραφή--........................................................................................................................\\

\noindent --Ονοματεπώνυμο--ΑΔΤ--Υπογραφή--........................................................................................................................\\

\noindent --Ονοματεπώνυμο--ΑΔΤ--Υπογραφή--........................................................................................................................\\

\noindent --Ονοματεπώνυμο--ΑΔΤ--Υπογραφή--........................................................................................................................\\

\noindent --Ονοματεπώνυμο--ΑΔΤ--Υπογραφή--........................................................................................................................\\

\noindent --Ονοματεπώνυμο--ΑΔΤ--Υπογραφή--........................................................................................................................\\

\noindent --Ονοματεπώνυμο--ΑΔΤ--Υπογραφή--........................................................................................................................\\

\noindent --Ονοματεπώνυμο--ΑΔΤ--Υπογραφή--........................................................................................................................\\

\noindent --Ονοματεπώνυμο--ΑΔΤ--Υπογραφή--........................................................................................................................\\

\noindent --Ονοματεπώνυμο--ΑΔΤ--Υπογραφή--........................................................................................................................\\

\noindent --Ονοματεπώνυμο--ΑΔΤ--Υπογραφή--........................................................................................................................\\

\noindent --Ονοματεπώνυμο--ΑΔΤ--Υπογραφή--........................................................................................................................\\

\noindent --Ονοματεπώνυμο--ΑΔΤ--Υπογραφή--........................................................................................................................\\


\end{document}





\item  ΥΠΕΥΘΥΝΗ ΔΗΛΩΣΗ ΟΛΩΝ ΤΩΝ ΣΥΜΜΕΤΕΧΟΝΤΩΝ ΣΤΟΝ ΠΑΤΡΙΩΤΙΚΟ ΧΩΡΟ, ΟΤΙ ΔΕΝ ΕΙΝΑΙ ΜΑΣΣΟΝΟΙ και  ΟΤΙ ΔΕΝ ΠΟΛΙΤΕΥΘΗΚΑΝ ΤΑ ΤΕΛΕΥΤΑΙΑ 40 ΧΡΟΝΙΑ. 




\makeatletter
\newcommand{\verbatimfont}[1]{\def\verbatim@font{#1}}%
\makeatother
\verbatimfont{pcr}
\verbatimfont{pcr}
\begin{verbatim}
This should be in pcr.
\end{verbatim}







ΣΑΝ ΣΗΜΕΡΑ ΑΥΤΟΚΤΟΝΗΣΕ Ο ΑΡΧΙΠΡΟΔΟΤΗΣ Ν. ΖΑΧΑΡΙΑΔΗΣ
01/08/2017
<https://ethnikismos.net/2017/08/01/%cf%83%ce%b1%ce%bd-%cf%83%ce%b7%ce%bc%ce%b5%cf%81%ce%b1-%ce%b1%cf%85%cf%84%ce%bf%ce%ba%cf%84%ce%bf%ce%bd%ce%b7%cf%83%ce%b5-%ce%bf-%ce%b1%cf%81%cf%87%ce%b9%cf%80%cf%81%ce%bf%ce%b4%ce%bf%cf%84%ce%b7/>

[image: zaxariadhs]
<https://ethnikismosblog.files.wordpress.com/2015/08/3701.jpg>

Σήμερα είναι  η επέτειος της αυτοκτονίας του αρχιπροδότη Ν. Ζαχαριάδη, του
μακροβιοτέρου Γ.Γ. του ΚΚΕ.

Ο Ν. Ζαχαριάδης ήταν ο άνθρωπος που ως αρχηγός του ΚΚΕ, εκτελώντας τις
διαταγές των εχθρών της Ελλάδος (Τίτο-Μπροζ, Στάλιν κ.λπ.), οργάνωσε και
εξετέλεσε το αιματοκύλισμα της Ελλάδος την περίοδο 1946-1949, γνωστό ως
«Τρίτο Γύρο», με σκοπό την απόσπαση της Μακεδονίας μας από τον εθνικό κορμό
και την ενσωμάτωσή της στην Γιουγκοσλαβική Ομοσπονδία του Εβραίου Τίτο. Θα
επικαλεσθούμε την μαρτυρία του Μάρκου Βαφειάδη, αρχιστρατήγου των
συμμοριτών: «Το άλλο είναι με τον Ζαχαριάδη, που το 1946 με τη συμφωνία
Τίτο-Πασχάλη-Ζαχαριάδη, τους δίνει τη Μακεδονία ολόκληρη μαζί με τη
Θεσσαλονίκη» (πηγή: «Απομνημονεύματα», 5ος τόμος, σελ. 381).

Εξάλλου, ο Ζαχαριάδης ήταν αρχηγός του ΚΚΕ τον Ιανουάριο του 1949, όταν το
κόμμα του εγκλήματος και της προδοσίας ανακοίνωσε την κατάπτυστη απόφαση
της 5ης Ολομελείας της Κεντρικής του Επιτροπής. Προς ενημέρωση των νέων
αναγνωστών μας, που διδάσκονται ΨΕΜΑΤΑ κατά συρροήν στα σχολεία περί της
«εθνικής μας αντίστασης», μεταφέρω τα κύρια σημεία της: «Στη Βόρεια Ελλάδα,
ο Μακεδονικός (Σλαβομακεδονικός) λαός τα ‘δωσε όλα για τον αγώνα και πολεμά
με μια ολοκλήρωση ηρωισμού και αυτοθυσίας, που προκαλούν τον θαυμασμό. Δεν
πρέπει να υπάρχει καμία αμφιβολία, ότι σαν αποτέλεσμα της νίκης του ΔΣΕ και
της λαϊκής επανάστασης, ο Μακεδονικός λαός θα βρει την πλήρη εθνική
αποκατάστασή του, έτσι όπως τη θέλει ο ίδιος, προσφέροντας σήμερα το αίμα
του για να την αποκτήσει».

Ξυλοκόπος για να αποκτήσει… προλεταριακή συνείδηση!

Μετά την συντριβή των συμμοριτών, ο Ζαχαριάδης κατέφυγε στο παραπέτασμα,
αφού απείλησε με σύντομη επιστροφή («Τέταρτο Γύρο»), με την δήλωσή του:
«Είμαστε με το όπλο παρά πόδα». «Βασίλευσε» στο ΚΚΕ έως το 1956. Το έτος
αυτό ξεκίνησε η «αποσταλινοποίηση», που σήμανε την καθαίρεση του Ζαχαριάδη
από την ηγεσία του ΚΚΕ. Το επόμενο έτος, διεγράφη από το κόμμα ως προδότης
και εξορίσθηκε στο Νοβγκορόντ της ΕΣΣΔ, όπου εργάστηκε στα δάση ως
ξυλοκόπος, σε πολικές θερμοκρασίες, για να αποκτήσει… «προλεταριακή
συνείδηση». Απελπισμένος, το 1962 επέστρεψε παράνομα στην Μόσχα και πήγε
στην ελληνική πρεσβεία ζητώντας να επιστρέψει στην Ελλάδα και να δικασθεί
για τα εγκλήματα και τις προδοσίες του!!! Συνελήφθη εκ νέου μετά την έξοδό
του από την πρεσβεία και οδηγήθηκε στο Σουργκούτ της Σιβηρίας. Δεν ήταν
ποτέ δυνατόν να τον αφήσουν οι Σοβιετικοί να επιστρέψει στην Ελλάδα, διότι
ο Ζαχαριάδης εγνώριζε τα πάντα και μ’ αυτά που θα απεκάλυπτε θα έβαζε
ταφόπλακα στο ΚΚΕ!

ΚΚΕ: Ο Ζαχαριάδης είναι βαμμένος εχθρός του κόμματος και της Σοβιετικής
Ένωσης!

Τον Φεβρουάριο του 1964 θα συνήλθε η 7η Ολομέλεια της Κεντρικής Επιτροπής
του ΚΚΕ στη Σοβιετική Ένωση, η οποία επιβεβαίωσε τα εγκλήματα και την
προδοσία του Ζαχαριάδη! Αυτός ο Σοβιετικός πολίτης, που υπέγραφε ως
«Νικολάγιεφ Νικολάι Νικολάγιεβιτς», με την ανάλογη ταυτότητα, θα
χαρακτηριζόταν «βαμμένος εχθρός της Σοβιετικής Ένωσης». Το 1964, η Κ.Ε. του
ΚΚΕ ανακοίνωσε τα εξής:

«Σαν συνέπεια όλου αυτού του «βίου και της πολιτείας» του Ζαχαριάδη έρχεται
φυσιολογικά η τελευταία φάση της αντικομματικής τυχοδιωκτικής,
αντισοβιετικής και ύποπτης δραστηριότητας του. Γράφει γράμματα στα οποία
ανάμεσα στις παλιές του συκοφαντίες διατυπώνει απειλές κατά της Κ. Ε. του
ΚΚΣΕ. Ταυτόχρονα έστειλε γράμμα στην Εισαγγελία της Αθήνας ζητώντας να πάει
στην Ελλάδα. Ένα μήνα ύστερα απ’ αυτό ο Ζαχαριάδης προσωπικά πηγαίνει στην
ελληνική πρεσβεία της Μόσχας, συζητάει με τον πρεσβευτή και δίνει προσωπικά
το γράμμα του για τον Εισαγγελέα Πλημμελειοδικών της Αθήνας. Ο Ζαχαριάδης
ζητάει με την βοήθεια του εχθρού να πάει στην Ελλάδα για να πολεμήσει το
ΚΚΕ και τη Σοβ. Ένωση. Ο Ζαχαριάδης αποκαλύφθηκε ανοιχτά εχθρικό στοιχείο.

Εκτός από τις πολιτικές ευθύνες, τους τυχοδιωκτισμούς, τη συνεργασία με την
Ασφάλεια, τις αντισοβιετικές προκλήσεις κ.λπ., ο Ζαχαριάδης φέρνει και
άμεσες ποινικές ευθύνες για μια σειρά πράξεις. Αναφέρουμε μερικές: Φέρνει
ποινική ευθύνη για τη φυσική εξόντωση στελεχών του κόμματος όπως του σ.
Κώστα Καραγιώργη. Ο ίδιος ο Ζαχαριάδης οργάνωσε τη δολοφονία του Πεχτασίδη
και ευθύνεται προσωπικά για τα εγκλήματα και τις δολοφονίες που έγιναν στο
Μπούλκες (στο νησί του Δούναβη).

Τέλος, σε προσωπική εντολή του Ζαχαριάδη οφείλεται η έκταση και η
αποτροπιαστική μορφή που πήραν οι συλλήψεις, τα βασανιστήρια και οι
δολοφονίες στην 7η Μεραρχία του ΔΣΕ.

Απ’ όλα τα παραπάνω βγαίνει ότι:

α) Ο Ζαχαριάδης με τη σεχταριστική – αντιφατική και τυχοδιωκτική γραμμή που
επέβαλε στο κόμμα, ιδιαίτερα μετά το 1945 (θέση για δύο πόλους, συμφωνία
για εισβολή στην Αλβανία, αποχή από τις εκλογές 1946, τυχοδιωκτική και
ανοργάνωτη έναρξη Δευτέρου ένοπλου αγώνα, θέση για μακεδονικό, αλλαγή του
χαρακτήρα της επανάστασης κ.λπ.), οδήγησε στα 1949 το κόμμα και το κίνημα
στην ήττα, στην απομόνωση και τη διάλυση και έκανε όλες εκείνες τις
ενέργειες που «συντέλεσαν στο να πραγματοποιηθεί το σχέδιο που κατάστρωσαν
η ελληνική αντίδραση και οι Αγγλοαμερικάνοι καταχτητές».

Μετά το 1949 εμπόδισε την ανασυγκρότηση και συσπείρωση του Κόμματος και του
Κινήματος με τις θέσεις για συνεχιζόμενη και μετά την ήττα επαναστατική
κρίση και «τα όπλα παρά πόδα», με την τυχοδιωκτική και εγκληματική οργάνωση
της παράνομης δουλειάς, με το ανώμαλο καθεστώς, τη διάλυση της Κ. Ε. και
του Π. Γ. και οδήγησε στη διάσπαση και το ανοιχτό ξέσπασμα της κρίσης του
κόμματος. Μετά την 6η Ολομέλεια της Κ. Ε. του 1956 ο Ζαχαριάδης στάθηκε ο
λυσσασμένος εχθρός της ενότητας του κόμματος.

β) 0 Ν. Ζαχαριάδης αποδείχτηκε βαμμένος εχθρός της Σοβιετικής ‘Ένωσης και
του ΚΚΣΕ. Η προβοκατόρικη θέση για στήριξη της Ελλάδας στην ιμπεριαλιστική
Αγγλία ενάντια στη Σοβ. Ένωση, η οργάνωση του πογκρόμ της Τασκένδης στα
1955 και μια σειρά άλλες προβοκατόρικης μορφής ενέργειες του.

γ) Ο Ν. Ζαχαριάδης αποδείχτηκε εχθρός της ενότητας του κόμματος. Και μετά
την 6η Ολομέλεια οργανώνει τους ομοϊδεάτες του και τους κατευθύνει με όλα
τα μέσα στην ψραξιονιστική και υπονομευτική δουλειά κατά του ΚΚΕ στο
εξωτερικό και στο εσωτερικό και κατά του ΚΚΣΕ και των άλλων αδελφών
κομμάτων»(πηγή: Πέτρου Ανταίου: «Ν. Ζαχαριάδης – θύτης και θύμα», εκδόσεις
Φυτράκη, σελ. 380-381).

«Το κουφάρι μου το κληροδοτώ στους Μπρέζνιεφ, Φλωράκη και Σία…».

Ώσπου, ήρθε το 1973. Ο απελπισμένος αρχηγός δεν άντεξε άλλο τα βασανιστήρια
και τις κακουχίες. Την 1η Αυγούστου βρέθηκε κρεμασμένος στο κελί του. Το
ΚΚΕ ανακοίνωσε ότι πέθανε από… ανακοπή καρδιάς. Βασικά, επρόκειτο για μια
προαναγγελθείσα αυτοκτονία. Τριάντα έξι ημέρες προ της αυτοκτονίας του,
στις 26/06/73, ο Ζαχαριάδης θα στείλει το τελευταίο του γράμμα στην Κ. Ε.
του Κ.Κ. της Σοβιετικής Ενώσεως:

«Το κουφάρι μου το κληροδοτώ στους Μπρέζνιεφ, Κολιγιάννη, Φλωράκη και Σία.
Χαλάλι τους…. Όσο κι αν προσπάθησαν δεκαεφτά ολόκληρα χρόνια τα όργανα της
Ασφάλειας της Σοβ. Ένωσης και οι διορισμένοι στο ΚΚΕ δεν απόδειξαν απόλυτα
τίποτα και αναγκάστηκαν να γλείψουν αφτού όπου φτύναν… Σήμερα, ύστερα απ’
όλα αυτά, δηλώνω ότι αν δεν αρθούν όλα τα μέτρα περιορισμού, εξορίας,
στέρησης ελευθερίας μετακίνησης και αναχώρησης απ’ τη Σοβ. Ένωση κ. τ. λ.
κ. τ. λ. που εφαρμόζονται ενάντια μου, τότε την 1η Αυγούστου 1973, σαν
έκφραση έσχατης διαμαρτυρίας, θ’ αυτοκτονήσω». (πηγή: ό. π., σελ. 491).
Συνεχίζουν τον πόλεμο κατά της Εκκλησίας οι άθεοι μπολσεβίκοι
Τρίτη, 1 Αυγούστου 2017 - 16:12

*Γράφει ο Ματθαιόπουλος Αρτέμης, Βουλευτής*

Η αδιαμφισβήτητη μεταστροφή του κυβερνητικού μορφώματος από ένα κατ’
επίφαση κίνημα κατά της λιτότητας σε πειθήνιο όργανο της διεθνούς
τοκογλυφίας, με σκοπό ύπαρξης του την επιβολή σκληρών αντιλαϊκών μέτρων ναι
μεν αποδόμησε το αφήγημα της αριστεράς ως το υποτιθέμενο μοναδικό αντίπαλο
δέος ενάντια στις δυνάμεις του καπιταλισμού, αλλά σε καμία περίπτωση δεν
εξάλειψε τα ανθελληνικά σύνδρομα που τη διακατέχουν.

*Μοιραία λοιπόν το κυβερνητικό μόρφωμα πρεσβεύει την αντίφαση και την
υποκρισία,* καθώς όσο απομακρυνόμαστε από το «σύγχρονο» κι «εξευρωπαϊσμένο»
πλαίσιο που έχει οριοθετήσει ως πεδίο πολιτικής του στα μεγάλα και
πρωτεύοντα ζητήματα, τόσο προσκρούουμε στις βαθιά ανθελληνικές και
διχαστικές θέσεις που υποστηρίζει ο εσωτερικός, σκληρός πυρήνας του στα
μικρότερα και δευτερεύοντα.

Ένα εξ αυτών και η εμμονή της κυβέρνησης για τον διαχωρισμό
εκκλησίας-κράτους. Αρχικά, για όσους αφελώς πίστεψαν ότι η απομάκρυνση του
ανιστόρητου και προκλητικού Φίλη από το Υπουργείο Παιδείας έγινε για λόγους
ευθιξίας των προϊσταμένων του κι εξαιτίας της αδιαλλαξίας του πάνω στο εν
λόγω θέμα, αξίζει να υπενθυμίσουμε πως *ο νυν υπουργός ουδέποτε αναίρεσε τα
λεγόμενα του προκατόχου του περί διαχωρισμού και ότι ουδέποτε διαφώνησε με
την προοπτική ρήξης που μεθοδικά προετοίμασε*. Ο άμεσος και χοντροκομμένος
τρόπος που επέλεξε ο πρώην υπουργός να διευθετήσει τις εκκρεμότητες του με
την εκκλησία μπορεί να επέφερε πολιτικό κόστος στην κυβέρνηση, αλλά οι
πρόσφατοι πανηγυρισμοί των κομματικών της φυλλάδων για το επόμενο χτύπημα
που ετοιμάζει εις βάρος της δεν αφήνει κανένα περιθώριο και αμφιβολίες για
το ποιες είναι οι προτεραιότητες που έχει ιεραρχήσει. Καμμία αναφορά στο
Σύνταγμα και της επίσημης Θρησκείας.

Με επιτροπή ειδικών εμπειρογνωμόνων, όπως μαθαίνουμε από τον έντυπο Τύπο,
προτίθεται να καταγράψει η κυβέρνηση την "μυθική περιουσία" της εκκλησίας
και το εκκλησιαστικό ακίνητο κεφάλαιο, με στόχο -όπως κομματικά στελέχη και
δημοσιογράφοι τονίζουν με έμφαση- την διευθέτηση του ιδιοκτησιακού
καθεστώτος και την αξιοποίηση του προς όφελος της κοινωνίας. Λίγους μήνες
μετά την κόντρα που ξέσπασε ανάμεσα σε κυβέρνηση και Eκκλησία για το μάθημα
των θρησκευτικών στα σχολεία και τη διαβεβαίωση του πρωθυπουργού για
ομαλοποίηση των μεταξύ τους σχέσεων, *ο ΣΥΡΙΖΑ προετοιμάζει τη δράση
ειδικών κλιμακίων για την χαρτογράφηση περιουσιακών στοιχείων, ανοίγοντας
(όπως είναι αναμενόμενο) νέα μέτωπα.*

Το τι αφήνουν να εννοηθεί όμως με όρους όπως «ιδιοκτησιακό καθεστώς» και
«αξιοποίηση προς κοινωνικό όφελος» γίνεται εύκολα αντιληπτό εάν αναλογιστεί
κανείς την ακραία νεοφιλελεύθερη και σκληρά καπιταλιστική πολιτική που έχει
ακολουθήσει η κυβέρνηση κατά τους δύο και πλέον χρόνους παραμονής της στην
εξουσία. Το ξεπούλημα της περιουσίας του ελληνικού λαού και η κάθετη
ιδιωτικοποίηση παντός περιουσιακού στοιχείου που μπορεί να αποτιμηθεί σε
χρήμα αποτελεί δεύτερη φύση του ΣΥΡΙΖΑ, παρόλα αυτά δεν έχει τον παραμικρό
ενδοιασμό να εμπαίζει τον λαό διαβεβαιώνοντας τον ότι αυτή η εθνοκτόνα
πολιτική αποσκοπεί στην εξασφάλιση του «κοινωνικού οφέλους», τα οποία έχουν
απολαύσει αφειδώς εδώ και τόσους μήνες εκατομμύρια Ελλήνων μέσω του ΕΝΦΙΑ,
της υψηλής φορολογίας, των υψηλών εισφορών και των χαμηλών μισθών.

*Η ίδια «κοινωφελής» πολιτική θα αφορά προφανώς και την Eκκλησία, στην
οποία η κυβέρνηση αναγνωρίζει ένα μεγάλο εμπόδιο για την πλήρη ισοπέδωση
των αξιών και την μετατροπή της χώρας σε ένα κράτος άθεων, εθνικά
«ουδέτερων» και βιολογικά αυτοπροσδιοριζόμενων πολιτών*. Εξ ου και οι νέες
επιθέσεις σε βάρος της με πρόσχημα την περιουσία της, που, πράγματι, αν και
μεγάλη, τίποτα δεν έχει να ζηλέψει από αυτή των στελεχών και των υπουργών
της κυβέρνησης, που συν τοις άλλοις πολλές φορές «παραλείπουν» να την...
«χαρτογραφήσουν» στις φορολογικές τους δηλώσεις.
ΜΑΣΟΝΙΑ …ΠΑΝΤΟΥ

   - at ΑΥΓΟΥΣΤΟΣ 01, 2017
   - απο ADMIN <http://egerssi.gr/author/ioulia/>
   - in ΤΑ ΑΡΘΡΑ ΜΟΥ <http://egerssi.gr/category/ta-arthra-mou/>
   - 0
   <http://egerssi.gr/%ce%bc%ce%b1%cf%83%ce%bf%ce%bd%ce%b9%ce%b1-%cf%80%ce%b1%ce%bd%cf%84%ce%bf%cf%85/#comments>



*Αλαφούζος, Τούρκοι, Survivor και στο βάθος …Παναθηναϊκός!  Σόρος, ΠΑΣΟΚ,
Ποτάμι, ΝΔ …βίοι παράλληλοι!*

*Γράφει ο Παναγιώτης Αποστόλου*

*Επικεφαλής της “Ελλήνων Πολιτείας”*

*egerssi@otenet.gr <egerssi@otenet.gr>          **www.egerssi.gr
<http://www.apostoloupanow.gr/>*

 Καλό, ευλογημένο, μήνα σε όλους σας!

Η λέσχη Rotary δημιουργήθηκε το 1905 από τον Αμερικανό δικηγόρο Paul Harris
στην πολιτεία του Ιλλινόις, στο εβραίο-κρατούμενο Σικάγο  των ΗΠΑ. Σ΄ αυτήν
την οργάνωση διείσδυσαν στη συνέχεια οι “Ροδόσταυροι” που θεωρούνται ότι
χειρότερο υπάρχει στη Μασονία!

Το 1917 ιδρύθηκε στο Σικάγο η λέσχη Lions (λέοντες)! Στην Ελλάδα
εμφανίστηκαν το 1954 και διατείνονται τους πατριώτες και τους
θρησκευόμενους ανθρώπους! Παρ΄ ότι οι κανονισμοί τους, τους απαγορεύουν να
ασχολούνται με θρησκευτικά ζητήματα!

Υπαινίσσονται πως είναι θεματοφύλακες των Αρχών, της νομιμοφροσύνης, της
αγαθοεργίας, της ευημερίας κλπ.

Ωστόσο, η Εκκλησία μας, θεωρεί πως έχουν στενή σχέση και συγγένεια με τον
Τεκτονισμό (Μασονία) και πως αποτελούν τον προθάλαμο και την κρησάρα
προκειμένου να εισέλθεις στην Μασονία! Από την στιγμή που το μεγαλύτερο
μέρος των Ροταριανών και των Lions είναι τέκτονες! Όπως, ο πρώτος
“κυβερνήτης” του εν Ελλάδι Rotary, ο οποίος ήταν διακεκριμένο μέλος της
ελληνικής Μασονίας!

Θα μου πείτε, γιατί ξεκινάω έτσι σήμερα;; Επειδή, από την πολύπαθη το
τελευταίο χρονικό διάστημα Κω, μου έρχεται η παρακάτω είδηση:

Παρουσία του Δημάρχου Κω κ. Γιώργου Κυρίτση, του Αντιδημάρχου κ. Ηλία
Σιφάκη, καθώς και προέδρων Κοινοτήτων του νησιού, πραγματοποιήθηκε το
Σάββατο 1ηΙουλίου 2017 στο ξενοδοχείο “Helona Resort” στην Καρδάμαινα, η
επίσημη παράδοση – παραλαβή του νέου ΔΣ της λέσχης Lion Κω!

Στο νέο ΔΣ της λέσχης Lion Κω, διακρίναμε στη θέση της Β΄ Αντιπροέδρου την
L Τλε Νιλούφερ! Η οποία σύμφωνα με πληροφορίες μας είναι νηπιαγωγός και
διατηρεί στη Κω, παιδικό σταθμό!

Αυτή η αναφορά μας γίνεται γιατί – όπως μαθαίνουμε – ο Πρόεδρος της ΝΔ κ.
Κυριάκος Μητσοτάκης επεξεργάζεται το ενδεχόμενο η κ. Νιλούφερ να βρίσκεται
στο ψηφοδέλτιο της ΝΔ στα Δωδεκάνησα!

Θα μου πείτε, που βρίσκεται το παράξενο! Αφού μέσα και στη σημερινή Βουλή,
αρκετοί είναι Μασόνοι και μάλιστα υψηλότατου βαθμού!

Το ζήτημα είναι πως, όταν είναι χαμηλού βαθμού στην Μασονική ιεραρχία, δεν
ντρέπονται να αποκαλύπτουν αυτή τη συμμετοχή τους στα Κοινά! Όταν όμως
ανεβαίνουν τα σκαλοπάτια της “μυστικής” ιεραρχίας τους, κρύβονται και δεν
αποκαλύπτονται!

Για την ιστορία – για πρώτη φορά το γράφω – όταν ήμουν υποψήφιος Βουλευτής
του ΛΑΟΣ στο νομό Ιωαννίνων στις εκλογές του 2007, λίγους μήνες πριν την
διεξαγωγή αυτών, αρνήθηκα την στήριξη της λέσχης Rotary Ιωαννίνων! Αν λέει
κάτι αυτό!!
*Αλαφούζος, Τούρκοι, Survivor και στο βάθος …Παναθηναϊκός!   *

Επί μήνες ο ελληνικός λαός, ο δυστυχισμένος και σκληρά πενόμενος ελληνικός
λαός – ευτυχώς όχι στο σύνολό του – αποχαυνωμένος παρακολουθούσε στην
τηλεόραση μια από τις προπαγανδιστικές εκπομπές, που λειτουργούν ως η
βαλβίδα εξαέρωσης της οργής του στη χύτρα που βράζει! Το Survivor, που
παρουσιαζόταν από τον τηλεοπτικό σταθμό του Γιάννη Αλαφούζου, του ΣΚΑΙ! Του
οποίου παραγωγός ήταν ο Τούρκος Acun Ilicali, που δεν σεβάστηκαν ούτε την
ημέρα Μνήμης για την Γενοκτονία των Ποντίων και πρόβαλαν την ίδια ημέρα
επεισόδια του Survivor για τη μάχη Ελλάδος – Τουρκίας!

Σύμφωνα με ακριβέστατες πληροφορίες μας – δεκτή η κάθε διάψευση – ο Acun
Ilicali συνεργάζεται στενά με τις Μυστικές Υπηρεσίες της Τουρκίας (ΜΙΤ)!
Αληθεύει αυτή η πληροφορία;;

Αληθεύει επίσης, ότι ο αδελφός του πρώην Υπουργού της ΝΔ κ. Μάκη Βορίδη,
Κώστας, ήταν ο αρχισυντάκτης του Sutvivor;;

Είναι επίσης αλήθεια, ότι μετά τους δεσμούς “αίματος” που αναπτύχθηκαν
μεταξύ Acun Ilicali και Γιάννη Αλαφούζου, ο ιδιοκτήτης της ιστορικής Ομάδος
του Παναθηναϊκού σκέπτεται να πουλήσει τον ΠΑΟ σε τούρκικα συμφέροντα;; Από
τη στιγμή που η Ελλάδα περνάει τη μεγαλύτερη οικονομική και αξιακή κρίση
της, οι Τούρκοι επιχειρηματίες βλέπουν ως μεγάλη ευκαιρία την αγορά της
Ελλάδος!

Ωστόσο, μας παραξένεψε και η μετάδοση από τον ΣΚΑΙ των συνεπειών που
προκάλεσε στην Κω ο πρόσφατος μεγάλος σεισμός! Ενώ, όλοι είχαν καταλήξει
πως η Κως άντεξε από τον σεισμό, ο ΣΚΑΙ παρουσίαζε μια άλλη πραγματικότητα!
Γιατί;;

Γιατί, ο ΣΚΑΙ δεν έδειξε πλάνα από το ισοπεδωμένο Bodrum, απέναντι στα
κατεχόμενα Μικρασιατικά παράλια;; γιατί;;

Ερωτήματα που αναμένουν απαντήσεων!!
*Σόρος, ΠΑΣΟΚ, Ποτάμι, ΝΔ …βίοι παράλληλοι!*

Μια από τις ισχυρές ΜΚΟ που χρηματοδοτείται από τον Σόρος, διά μέσω του
σκοπιανού παραρτήματος της εταιρείας του “Open Society Foundation” είναι η
“Solidarity Now”! Έχουν αναπτύξει μεγάλη δραστηριότητα και φήμες αναφέρουν
ότι χρηματοδοτείται και από το Υπουργείο Εξωτερικών της Ελλάδος (;;;;)!
Είναι αλήθεια ΥΠΕΞ κ. Νίκο Κοτζιά;;;

Η οποία ΜΚΟ αυτήν τη στιγμή ασχολείται με τη συγγραφή της “κοινής
βαλκανικής ιστορίας”! Θεός φυλάξοι!

Αλλά, ας δούμε μερικά από τα μέλη του ΔΣ αυτής της ΜΚΟ του Σόρος!

Στέλιος Ζαββός: Πρόεδρος της ΜΚΟ και κουμπάρος του Σόρος! Οικείος με τους
Λουκά Παπαδήμο, Γκίκα Χαρδούβελη, Νίκο Δένδια κα!

Χρήστος Ροζάκης: Νομικός, Υφυπουργός Εξωτερικών στην κυβέρνηση Σημίτη,
υπερασπιστής των μειονοτήτων και έχει “καταδικάσει” την Ελλάδα για την μη
αναγνώριση των “μακεδονικών” και τουρκικών συλλόγων!

Νίκος Αλιβιζάτος: Νομικός, υπηρεσιακός Υπουργός Εσωτερικών στην κυβέρνηση
Σημίτη (13/2-10/3/2004) και σύμβουλός του, υποψήφιος για την Προεδρεία της
Δημοκρατίας, τον Φεβρουάριο του 2015 με τη στήριξη του ΠΑΣΟΚ και του
“Ποταμιού”!

Αντιγόνη Λυμπεράκη: Καθηγήτρια της Παντείου, Βουλευτής της Β΄ Αθηνών με το
“Ποτάμι” (το κόμμα των Διεθνιστών). Πρώην αντιπρόεδρος του κόμματος “Δράση”
του νεοφιλελευθέρου Στέφανου Μάνου! Κόρη του Γεωργίου Λυμπεράκη και της
Αρτεμισίας Μητσοτάκη! Αδελφής του Κωνσταντίνου Μητσοτάκη!

Ευθύμιος Βιτάλης: Επιχειρηματίας, διετέλεσε αντιπρόεδρος του ΣΕΒ, σύμβουλος
του Αντώνη Σαμαρά! Η σύζυγός του βρέθηκε στη λίστα Λαγκάρτ!

Οι πληροφορίες μας λένε, όλους αυτούς τους συνδέει το λαθρομεταναστευτικό,
εξ ου και οι πολύ καλές σχέσεις με τον Ευρωπαίο Επίτροπο Δημήτρη
Αβραμόπουλο!

Τι είπατε;; Δεν άκουσα;;

Στις εκλογές, όποτε αυτές γίνουν, θα τρέξετε να ψηφίσετε ΝΔ, ΠΑΣΟΚ και
“Ποτάμι”; Με τις υγείες σας λοιπόν!!

Επανέρχομαι…





----------------------------------------------------


Η Άγνωστη Τοπική Αυτοδιοίκηση
                 του Παναγ. Παπαγαρυφάλλου
Συνέχεια 9η
Μέρος  Δεύτερον: Η Τοπική Αυτοδιοίκηση από  την Λήξη του Β’ Παγκοσμίου Πολέμου έως το Σύνταγμα του 1975
                                                 Κεφάλαιον  Πρώτον
Η λειτουργία του θεσμού υπό το καθεστώς του Συντάγματος του 1952 

1.    Πριν αναφερθώ στο Σύνταγμα του 1952,  σημειώνω την πιο δημοκρατική διάταξη του  άρθρου 107 του δημοκρατικού Συντάγματος του 1927,  το οποίο ορίζει:

«Το κράτος διαιρείται εις περιφερείας, εντός των οποίων οι πολίται διαχειρίζονται απ΄ ευθείαςτας τοπικάς υποθέσεις, ως νόμος θέλει ορίσει. Η κοινότης αποτελεί την πρώτη βαθμίδα των τοιούτων Οργανισμών Τοπικής Αυτοδιοικήσεως οι οποίοι πρέπει να είναι κατ’ ελάχιστον όριον δύο βαθμών, ανεξαρτήτως των Δήμων και των Συνδέσμων των κοινοτήτων. Το δικαίωμα του αποφασίζειν εις τους ανωτέρω οργανισμούς επί ζητημάτων εις την σφαίρα της Τοπική ς Αυτοδιοικήσεως ανήκει απαραιτήτως εις αιρετάόργανα εκλεγόμενα δια καθολικής ψηφοφορίας ή και αμέσως εις το σύνολον τωνεις έκαστον εξ’ αυτών ανηκόντων πολιτών. Το κράτος ασκεί, καθώς ο νόμος θέλει ορίσει, μόνον ανώτατην εποπτείαν επί των Οργανισμών Τοπικής Αυτοδιοικήσεως μη εμποδίζουσαν την πρωτοβουλίαν και την ελευθέραν δράσιν αυτών. Το κράτος δύναται να συντρέχει οικονομικώς τους ΟργανισμούςΤοπικής Αυτοδιοικήσεως». 

2.    Είναι προφανέστατο ότι το βραχύβιο Σύνταγμα του 1927 υπήρξε πολύ περισσότερο φιλοαυτοδιοικητικό από το μεταπολεμικό  Σύνταγμα του 1952,  το άρθρο 99 του οποίου περιορίστηκε σε μια λακωνική διάταξη  η οποία έλεγε «Η διοικητική οργάνωσις του κράτους βασίζεται στην αποκέντρωσιν και την τοπικήν αυτοδιοικήσιν, ως νόμος ορίζει. Η εκλογή των δημοτικών και κοινοτικών αρχών γίνεται δια καθολικής ψηφοφορίας». 

Πρόκειται περί δύο Συνταγμάτων και δύο αντιλήψεων για τον ρόλο και την αποστολή του θεσμού που απέχουν παρασάγγες. Το Σύνταγμα του 1927 του έδινε φτερά και βάση απογείωσης, ενώ τοΣύνταγμα του 1952 τον καθήλωνε σε στασιμότητα και στη μιζέρια και κατά την έγκυρη άποψη του Φ. Βαγλερή«έδειχνε δυσπιστία στην αυτοδιοικήση». 96Πρόκειται για ένα γεγονός το οποίο, κυρίως ανέδειξα στο έργο μου: «Το πρόβλημα της ουσιαστικοποιήσεως  των αρμοδιοτήτων  της Τοπικής Αυτοδιοικήσεως εν Ελλάδι και οι δικαιολογητικαί  βάσεις της διευρύνσεως τούτων». 97  
Καρπός του αντιαυτοδιοικητικου Συντάγματος του 1952 υπήρξε το Ν.Δ.2888/1954 με το οποίο θεσπίστηκε ο τότε Δημοτικός Κώδικας ο οποίος επέβαλε ασφυχτικούς ελέγχους κάθε λογής στη δράση του θεσμού. 98  
4. Είναι αυτονόητον ότι το Κοίλον,  περιεχόμενον του θεσμού, εξ απόψεως αρμοδιοτήτων , και το πλέγμα των δεσμεύσεων στην δράση του, έχει δυσμενέστατο αντίκτυπο και στα έσοδα του,  τα οποία δεν επαρκούσαν  στοιχειωδώς στην εκπλήρωση της αποστολής του. (βλ το έργο μου «Η μεταπολεμική εξέλιξις και διάρθρωσις των οικονομικών της Τ.Α. στην Ελλάδα – μετά συγκριτικών στοιχείων των Ευρωπαϊκών Χωρών»),   στην  «Επιθεώρηση  Τοπικής Αυτοδιοικήσεως » τεύχη (Δ-Ε Απριλίου- Μαΐου 1975 και ειδικότερα το Δεύτερο Κεφάλαιο).Το γεγονός αυτό οδηγούσε  ευθέως στην απόλυτη εξάρτηση των ΟΤΑ από το Κράτος- Κόμμα το οποίο δια της  μεθόδου των επιλεκτικών χορηγήσεων, ασκούσε κηδεμονία στη δράση τους.99
5.  Των πράγματων ούτως εχόντων, στην πιο πάνω έ­ρευνά μου, κατέληγα στο συμπέρασμα ότι: «Το οικονομικό πρόβλημα των ΟΤΑ εκκρεμεί εις την χώραν μας από της συστάσεως του νοελληνικού Κράτους».100
΄Ενα Κράτος στο οποίο κατά τη χρονική περίοδο 1958-1965, «τα συνολικά έσοδα της Τ.Α. ανήρχοντο εις το 50% της εθνικής δαπάνης δια ποτά καί καπνόν και εις το 27% της εθνικής δαπάνης δια κρέας» κατά μέσον όρον.101
Είναι αυτονόητο ότι υπό τίς προϋποθέσεις αυτές, κάθε άσκηση δημοτικής καί κοινοτικής πολιτικής και εκπλήρωση στοιχειωδών καθηκόντων, τα οποία απορρέουν από τη φύση καί την αποστολή της αυτοδιοικήσεως ήταν αδύνατη.
6.  Πρόκειται για ένα γεγονός, το οποίο καθιστούσε δυσμενέστερη η πληθώρα των ΟΤΑ στην Ελλάδα, οι οποίοι περισσότερο έμοιαζαν με φαντάσματα παρά με αυτοδιοι­κούμενους λειτουργικούς οργανισμούς, ικανούς να ικα­νοποιήσουν στοιχειωδώς τίς τοπικές ανάγκες των κατοί­κων.
Εντελώς ενδεικτικά παραθέτω τα έξης στοιχεία από την Απογραφή του 1951, σύμφωνα με την οποία ο συνολικός αριθμός των κοινοτήτων ανερχόταν σε 5.756, από τίς όποιες 5.453 δηλαδή ποσοστό 94,7% είχαν πληθυσμό κάτω των 2.000 κατοίκων ενώ παραλλήλως υπήρχαν καί 10.900 οικι­σμοί, από τους οποίους οι έχοντες αριθμό κατοίκων κάτω των 2.000 ήταν 10.489 δηλαδή το 96% του συνόλου τους.
7. Αξίζει να σημειωθεί ότι ο κύριος όγκος των κοινοτή­των πού είχαν πληθυσμό μεταξύ 1.000-1.999 ανερχόταν σε 906, πού αντιπροσώπευαν 1.220.000 κατοίκους δηλαδή το 15,7%.102
 
8. Υπό το φως αυτών των δεδομένων καί της διαπιστώ­σεως της εδώ Αμερικανικής Αποστολής ότι «ή Ελλάς ήτο πι­θανώς ή πλέον συγκεντρωτική χώρα μεταξύ των ελεύθερων εθνών», γίνεται πρόδηλον ότι ή ελληνική αυτοδιοίκηση αυτής της δεκαετίας (1951-1961) αποτελούσε σκιά αυτοδιοικήσε­ως, η οποία όπως εγκύρως γράφτηκε, οδηγούσε «σε ολοσχε­ρή απώλεια πάσης εξουσίας επί των τοπικών υποθέσε­ων».103
Αξίζει, επίσης να σημειωθεί, ότι σύμφωνα με τα απο­γραφικά στοιχεία του 1951 επί συνολικού πληθυσμού 7.632.000 της Ελλάδος, τα 4.310.000 διαβιούσαν σε Κοινότητες δηλαδή το 78,4%, ενώ οι  δήμοι άνω των 20.000 κατοίκων ανέρχονταν σε 35 αντιπροσωπεύοντας το 27% του πληθυ­σμού.
9.  Επρόκειτο για μία τοπική αυτοδιοίκηση με γυάλινα πόδια την οποία δεν ισχυροποίησε ο νέος Δημοτικός καί Κοινοτικός Κώδικας, ό οποίος κυρώθηκε με το Ν.Δ. 2888/ 1954, ό οποίος κάθε άλλο παρά ήρθε ν' αντιμετωπίσει τα πολλαπλά λειτουργικά αδιέξοδα του θεσμού, μη απομακρυνόμενος ουσιαστικώς από τις ρυθμίσεις του Κώδικα ΔΝΖ/1912.
Το γεγονός αυτό προκάλεσε την αντίδραση των εκπροσώπων του θεσμού, οι οποίοι στα Συνέδριά τους ζητούσαν  την ισχυροποίηση-συγχώνευση των μικρών κοινοτήτων, ενώ   γνωστός καθηγητής του διοικητικού δικαίου, διατύπωσε τη δυσμενή κριτική του γράφοντας: «Λόγω ενός ουτοπικού φιλελευθερισμού του νόμου καί προσέτι της αρνήσεως  της κεντρικής διοικήσεως να επιχειρίση ανασύστασιν των κοινοτήτων-φαντασμάτων, αι δυνατότητες οιασδήποτε τοπικής δράσεως θυσιάζονται είτε χάριν αφηρημένων αρχών, είτε  χάριν της εκλογικής σκοπιμότητας».104
Την ίδια περίοδο, ο γνωστός Κοινοτιστής Κ. Καραβίδας, ασκώντας οξύτατη κριτική στην Κυβερνητική Πολιτική έγραφε καί τα έξης: «Η σημερινή εν Ελλάδι Τοπική, ως λέγεται, Αυτοδιοίκησις, άνωθεν δοτή... πράγματι ως μία τυπική γραφειοκρατική υπηρεσία, ως π.χ. δια την τήρησιν των ληξιαρχικών βιβλίων κ.λπ. ευρίσκεται εις πολύ μεγάλην απόστασιν από την πανάρχαιαν κοινοτικήν μας παράδοσιν, η οποία διεμορφώθη αυτοκαθοριστικά δια μέσου των αιώνων, ως ουσιαστικόν βίωμα, κατ' ακριβείαν ως όρος υπάρξεως....»105
10. Ήταν η εποχή κατά την οποία υπήρχε έκδηλη αντίφα­ση των προθέσεων καί των πραγματώσεων του ελληνικού κράτους, το οποίο άλλα διεκήρυσσε στην Αιτιολογική έκθεση του δημοτικού καί κοινοτικού Κώδικα (ΝΔ 2888/ 1954) καί άλλα έκανε στην πράξη. Ανεξαρτήτως όμως αυτού του γεγονότος, ο Κώδικας προέβαινε στην καταγραφή των τοπικών υποθέσεων, τις οποίες ανέθετε στους ΟΤΑ και μάλιστα κατ' αποκλειστικότητα. Ανάμεσα σ’ αυτές σημειώ­νω:
α) Κατασκευή, συντήρησις καί λειτουργία συστημάτων υδρεύσεως.
β) Η κατασκευή, συντήρησις καί λειτουργία συστημά­των υπονόμων καί αποχετεύσεων.
…
στ) Η κατασκευή, συντήρησις καί λειτουργία συστη­μάτων άδρεύσεως καί εγγειοβελτιωτικών έργων,
ζ) Η κατασκευή, συντήρησις καί λειτουργία δημοτικών και κοινοτικών αλσών, παιδικών κήπων, υπαιθρίων κοινο­χρήστων χώρων αναψυχής...106
Καταλήγοντας, ως προς το νομοθετικό καθεστώς του Δημοτικού Κωδικός του 1954, σημειώνω ότι αυτό ήταν τόσο ασφυκτικό στη δράση των ΟΤΑ ώστε αδυνατούσαν να προ­βούν στη χορήγηση έστω καί μίας γαλοπούλας σε άπορες οικογένειες τα Χριστούγεννα χωρίς την έγκριση του Νομάρχου, όπως έγραφα στίς πιο πάνω εργασίες μου.
Στό σημείο όμως αυτό είναι χρήσιμο να επισημανθούν οι έγκυρες παρατηρήσεις του Φ. Βεγλερή, ο οποίος αναφερό­μενος στη σημασία των νομοθετικών κειμένων, έγραφε ότι από πλευράς εφαρμοστέας δημοτικής καί κοινοτικής πολι­τικής έχουν μεγαλύτερη σημασία όχι αυτές καθ' αυτές οι διατάξεις πού διέπουν τη λειτουργία καί τη δράση των ΟΤΑ αλλά η δυνατότητα να εκτελέσουν τις αρμοδιότητες οι οποίες τους ανατίθενται.
Άλλο τί ορίζει η νομοθεσία καί εντελώς διαφορετική είναι  στην πράξη η δραστηριότητα των ΟΤΑ, των οποίων οί  καταγραφόμενες αρμοδιότητες ασκούνται ταυτοχρόνως καί  «από ειδικά νομικά πρόσωπα δημοσίου δικαίου, δημοσίων επιχειρήσεων καί ιδιωτικές εταιρείες»107.
11. Από τα απογραφικά στοιχεία του 1961, ως προς τη δομή του κοινοτικο-δημοτικού μας συστήματος  προκύπτει ότι τα πράγματα παρέμειναν στην ίδια περίπου κατάσταση  με την απογραφή του 1951.
 Αυτό το αντιαναπτυξιακό γεγονός αποτέλεσε ανασχε­τικό παράγοντα για κάθε κοινωνική, οικονομική καί πολιτιστική άνοδο των τοπικών κοινωνιών, των οποίων οι ΟΤΑ « όχι  μόνο εστερούντο Κοινοτικών Γραφείων αλλά καί αυτής της δυνατότητας να τοποθετηθούν τα μέλη του συμβουλίου πέριξ μίας τράπεζας» καί αυτό «αφορούσε το 98% των Κοινοτήτων».108
Αυτή η υπολειτουργία ή «ανυπαρξία» της αυτοδιοική­σεως συνεχιζόταν παρά το γεγονός ότι ο δημοτικός κώδικας ( Ν.Δ. 2888/1954)  προέβλεπε  τη  δυνατότητα καταργήσεως των μικρών καί αναιμικών κοινοτήτων και της συγχωνεύσεώς τους «με συνεχόμενον δήμον ή κοινότητα» ενώ προς την ιδία κατεύθυνση είχε ήδη κινηθεί καί η ΚΕΔΚΕ, ήδη από το  1953 με υποβολή σχετικού υπομνήματος στο Υπουργείο των Εσωτερικών.109
12. Εν αναφορά προς τα οικονομικά των ΟΤΑ κατά τη χρονική περίοδο 1954-1967 πρέπει να σημειωθεί ότι αυτά υπήρξαν -όπως πάντα- πενιχρά τόσο εξ ιδίων φορολογικών εσόδων όσο καί από κρατικές επιχορηγήσεις, οι οποίες έβαι­ναν μειούμενες εν συγκρίσει προς τα έσοδα του Κρατικού Προϋπολογισμού.
Συγκεκριμένα, από την ανάλυση σχετικού πίνακος αυ­τής της περιόδου προέκυψε ότι κατά το διάστημα αυτό τα έσοδα των ΟΤΑ εν σχέσει προς τα έσοδα του Κράτους μειώ­θηκαν από 5% σε 4,8%.110  Υπό το κράτος των γλίσχρων εσόδων των ΟΤΑ ήτο επόμενον, η αδύνατη  συμβολή τους στον τομέα των επενδύσεων.
Αντιγράφω το αναλυτικό συμπέρασμα στο οποίο είχα καταλήξει το 1975: «Ήτοι, εκ της αναλύσεως των ως άνω στοιχείων προκύπτει ότι το σύνολον των δημοσίων επεν­δύσεων κατά το διάστημα 1960-1967, ήταν άνωτερον των αντιστοίχων επενδύσεων των ΟΤΑ, κατά 11.500 φορές περί­που, ενώ το σύνολον των ιδιωτικών επενδύσεων ήταν άνω­τερον κατά 25.200 φορές», καί καταλήγοντας έγραφα: «Αυτό σημαίνει ότι οι ΟΤΑ, αδυνατούν να επηρεάσουν καί να δια­μορφώσουν την τοπικήν ζωήν καί το περιβάλλον εντός του οποίου διαβιούν οι δημόται των».111
(Συνεχίζεται)
 
 ΣΗΜΕΙΩΣΕΙΣ
96 Βλ.  άρθρο του «Μια πέτρα του Σίσυφου» στο περιοδ. «Τοπική Αυτοδιοικήση» τευχ. 1/1978
97 Βλ. περιοδικό της ΚΕΔΚΕ, «Επιθεώρησις  Τοπικής Αυτοδιοικήσεως » τεύχη Ζ-Η ( Ιούλιος- Αύγουστος 1974 )
98 Βλ. οπ. π. σελ.1101-1127
99Βλ. όπ. π. σελ. 68 του ανατύπου.
100 Βλ. όπ. π. σελ. 111.
 101Βλ. . όπ. π.   σελ. 88.
102 Περισσότερα αναλυτικά στοιχεία για τις δομές των κοινοτήτων από την απογραφή του 1951 βλ. την εργασία μου: « Η συγχώνευση των Κοι­νοτήτων στην Ελλάδα -από άποψι θεωρίας, τάσεως και πραγματικότητος», στην «Επιθεώρηση Τοπικής  Αυτοδιοικήσεως» τεύχ. Οκτωβρί­ου 1975 ή το ανάταπο στο οποίο παραπέμπει ο γράφων και εν προκειμέ­νω σελ. 25-27 καί τις εκεί υποσημειώσεις.
103 Βλ. Εισήγηση του Δ. Ψαρού στο Α' Συνέδριο των πόλεων της Ελλά­δος», το οποίο συνήλθε στην Αθήνα το 1951 και τα σχετικά του πρακτι­κά που εκδόθηκαν τον ίδιο χρόνο από το Δήμο Αθηναίων, και τα εδώ παραθέματα σελ. 265.
104 Για τα ζητήματα αυτά βλ. το έργο μου: «Η Συγχώνευση των Κοινοτήτων», σελ. 28-31 καί τις αυτόθι σημειώσεις.
105 Βλ. Κ. Καραβίδα: «Η αρχαία καί η Κοινοτική παράδοσις των Ελλήνων», στην «Επιθεώρηση Τοπικής Αυτοδιοικήσεως», έτους 1958, τεύχ. Ζ', σελ.505-512
 106 Περισσότερα για τις ρυθμίσεις αυτού του Κώδικα και Κριτικές Παρατηρήσεις της επιστήμης και των εκπροσώπων των ΟΤΑ βλ. στα έργα μου: «Το πρόβλημα τής ουσιαστικοποίησεως των αρμοδιοτήτων τής Τ.Α. εν Ελλάδι», οπ.π. σελ. 1107-1110, «Η συγχώνευσι των Κοινοτή­των», οπ. π. σελ. 30-35 και «Αι καταβολαί του θεσμού τής Τ.Α. εν Ελλάδι και αι ποοσπάθειαι διοικητικής οργανώσεως του», στην  «Επιθ. Τοπ. Αυτοδιοικήσεως» έτους 1974, τευχ. Α'
107 Βλ. το άρθρο του: «Μία πέτρα του Σισύφου» στο μηνιαίο περιοδικό «Τοπική Αυτοδιοίκηση», τεύχ. 1/1978, συνεκδότης-συνδιευθυντής ο γρα­φών.
108 Περισσότερα για τα όσα σημειώνονται εδώ Βλ. στο έργο μου: «Η συγ­χώνευση των Κοινοτήτων στην Ελλάδα», σελ. 20-24 και τα εκεί αδιά­σειστα ντοκουμέντα.
109  Περισσότερα στο έργο μου: «Αι καταβολαί του Θεσμού της Τ.Α. εν Ελλάδι και αί προσπάθειαι διοικητικής οργανώσεώς του» σελ. 37-38 και τις εκεί  πηγές.
110 Περισσότερα αναλυτικά στοιχεία έσοδων-δαπανών κατά τη χρονική περίοδο 1954-1965, βλ. στο έργο μου: «Η μεταπολεμική εξέλιξις των οικονομικών των ΟΤΑ» κ.τ.λ. όπ. π. 71-87.
111  Βλ. οπ. π. σελ. 93.
ΣΥΝΤΑΞΙΟΔΟΤΙΚΗ Μάρτιος 2015
 
 
 
 
 



	

	

 


	

 




Copyright © 2017 Papagarifallou, All rights reserved.
Του πολίτη Π.Λ.Παπαγαρυφάλλου

Η λίστα των email είναι:

Papagarifallou

Aristeidou

Athens, GR 10559 

Greece


 <http://tweebit.us13.list-manage.com/vcard?u=96f2fa5f16adaeb761684afbf&id=d3449866f2> Add us to your address book



Θέλετε να αλλάξετε τη διεύθυνσή σας;
Μπορείτε να  <http://tweebit.us13.list-manage.com/profile?u=96f2fa5f16adaeb761684afbf&id=d3449866f2&e=a4c420fea6> ανανεώσετε τα στοιχεία σας ή  <http://tweebit.us13.list-manage.com/unsubscribe?u=96f2fa5f16adaeb761684afbf&id=d3449866f2&e=a4c420fea6&c=c62712c7e9> να κάνετε ΔΙΑΓΡΑΦΗ από τη λίστα παραληπτών

 <http://www.mailchimp.com/monkey-rewards/?utm_source=freemium_newsletter&utm_medium=email&utm_campaign=monkey_rewards&aid=96f2fa5f16adaeb761684afbf&afl=1> Email Marketing Powered by MailChimp

  <http://tweebit.us13.list-manage.com/track/open.php?u=96f2fa5f16adaeb761684afbf&id=c62712c7e9&e=a4c420fea6> 








Η σημαία επανέρχεται στην παλαιά μορφή με το σταυρό στη μέση.
Στα σχολεία έχουμε έπαρση της σημαίας, Πρωϊνή προσευχή και τον Εθνικό Ύμνο, και στην αποχώρηση υποστολή αυτής.










